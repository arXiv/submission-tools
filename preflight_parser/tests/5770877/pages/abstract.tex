\thispagestyle{plain}
\begin{center}
    \Large
    \phantomsection
    \addcontentsline{toc}{section}{Abstract}
    \textbf{Abstract}
\end{center}

% In this project, we explore the effectiveness of graph neural networks (GNNs) for predicting transactions in the Ethereum network. Specifically, we formulate the graph as a dynamic graph and perform the prediction task snapshot by snapshot, using the transaction data of erc721 tokens.

% The project report details the methodology, data preprocessing steps, and experimental setup used in this study. We utilize a dataset of ERC721 token transactions, which includes the sender and receiver addresses, the transaction timestamp, and other related information. The data is preprocessed to create a dynamic graph where each snapshot represents transactions that occur within a specific time window. Our experimental results show that the GNN-based model effectively predicts new token transactions in the Ethereum network with high accuracy. Additionally, we develop a real-time API based on the model we trained, which processes real-time data and updates its prediction periodically. This shows that the GNN-based model is suitable for real-life applications.

% The proposed GNN-based model has potential applications in various fields, including fraud detection, price prediction, address clustering, etc. We conclude that GNNs can be used to predict new erc721 token transactions in a dynamic graph context with high accuracy and recommend further research on this topic.

Phishing detection is a critical cybersecurity task that involves the identification and neutralization of fraudulent attempts to obtain sensitive information, thereby safeguarding individuals and organizations from data breaches and financial loss. In this project, we address the constraints of traditional reference-based phishing detection by developing an LLM agent framework. This agent harnesses Large Language Models to actively fetch and utilize online information, thus providing a dynamic reference system for more accurate phishing detection. This innovation circumvents the need for a static knowledge base, offering a significant enhancement in adaptability and efficiency for automated security measures.

The project report includes an initial study and problem analysis of existing solutions, which motivated us to develop a new framework. We demonstrate the framework with LLMs simulated as agents and detail the techniques required for construction, followed by a complete implementation with a proof-of-concept as well as experiments to evaluate our solution's performance against other similar solutions. The results show that our approach has achieved with accuracy of 0.945, significantly outperforms the existing solution(DynaPhish) by 0.445. Furthermore, we discuss the limitations of our approach and suggest improvements that could make it more effective.

Overall, the proposed framework has the potential to enhance the effectiveness of current reference-based phishing detection approaches and could be adapted for real-world applications.


\vspace{2cm}
Subject Descriptors:

\begin{tabular}{c c}
I.2.4 & Knowledge Representation Formalisms and Methods \\
I.2.7 & Natural Language Processing \\
I.4 & Image Processing and Computer Vision \\
H.3.3 & Information Search and Retrieval\\
\end{tabular}

\vspace{2cm}
\noindent Keywords:
Generative Agent, Phishing Detection, Knowledge Discovery \\
\indent 

\pagebreak