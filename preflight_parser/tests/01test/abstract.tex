% abstract
%
\bgroup
\color{abs}
\hrule
\egroup



%
%%  DO NOT EDIT  LINES ABOVE                     %

\begin{abstract}
\noindent \textit{In mechanics, common energy principles are based on fixed boundary conditions. However, in bridge engineering structures, it is usually necessary to adjust the boundary conditions to make the structure's internal force reasonable and save materials. However, there is currently little theoretical research in this area. To solve this problem, this paper proposes the principle of minimum virtual work for movable boundaries in mechanics through theoretical derivation such as variation method and tensor analysis. It reveals that the exact solution of the mechanical system minimizes the total virtual work of the system among all possible displacements, and the conclusion that the principle of minimum potential energy is a special case of this principle is obtained. At the same time, proposed virtual work boundaries and control conditions, which added to the fundamental equations of mechanics. The general formula of multidimensional variation method for movable boundaries is also proposed, which can be used to easily derive the basic control equations of the mechanical system. The incremental method is used to prove the theory of minimum value in multidimensional space, which extends the Pontryagin's minimum value principle. Multiple bridge examples were listed to demonstrate the extensive practical value of the theory presented in this article. The theory proposed in this article enriches the energy principle and variation method, establishes fundamental equations of mechanics for the structural optimization of movable boundary, and provides a path for active control of mechanical structures, which has important theoretical and engineering practical significance.
}\\

\noindent\textbf{Keywords}:  Principle of Minimum Virtual Work, Variation Method, Tensor Analysis, Bridges, Structural Optimization


\end{abstract}%

\bgroup
\color{abs}
\hrule
\egroup


