\documentclass[12pt]{article}
\usepackage[a4paper,left=1.0in,right=1.0in,top=1.1in,bottom=1.3in]{geometry}

\usepackage{setspace}
\usepackage{microtype}%if unwanted, comment out or use option "draft"

%\graphicspath{{./graphics/}}%helpful if your graphic files are in another directory
%\usepackage[style=alphabetic]{biblatex}
\usepackage{authblk}
\usepackage{xcolor}
\usepackage{etex,ifthen}
\usepackage{etoolbox}
\usepackage{graphicx}
\usepackage{rotating}
\usepackage{tikz}
\usepackage{pgfplots}\pgfplotsset{compat=1.14}
\usepackage{tikz-cd}
\usepackage{url}
\usepackage{calc}
\usepackage[font={small,it}]{caption}
\usepackage{complexity}
\usepackage{alphalph}
%\usepackage{enumitem}
\usepackage{amsthm}
\usepackage{thmtools}
\usepackage{eqparbox}
\usepackage{afterpage}
\usepackage{amsfonts,amsmath,amssymb}
\usepackage{mathrsfs}
%\usepackage{program}
\usepackage{algorithm}
\usepackage{algpseudocode}
%\usepackage{mathabx}
%\usepackage{txfonts}
\usepackage{bm}
\usepackage{mdframed}
\usepackage{listings}
\usepackage[english]{babel}
\usepackage{hyperref}
%\usepackage[noabbrev,capitalize]{cleveref}
%\usepackage{mathtools}
\usepackage{verbatim}
%\usepackage{mathpazo}
%\usepackage{MnSymbol}
\usepackage{appendix}
\usepackage[all]{xy}

\usetikzlibrary{chains,fit}
\usetikzlibrary{shapes.geometric}
\usetikzlibrary{matrix,positioning,calc}
\usetikzlibrary{decorations.markings}
\usetikzlibrary{decorations.pathmorphing}
\tikzstyle{vertex}=[circle, draw, inner sep=0pt, minimum size=6pt]
\tikzstyle{svertex}=[circle, draw, inner sep=0pt, minimum size=3pt]
\tikzstyle{dvertex}=[circle, draw, inner sep=0pt, minimum size=9pt]
\tikzstyle{vertbox}=[draw, inner sep=0pt, minimum size=8pt]
\newcommand{\vertex}{\node[vertex]}
\newcommand{\svertex}{\node[svertex]}
\newcommand{\dvertex}{\node[dvertex]}
\newcommand{\vertbox}{\node[vertbox]}

\newcommand{\oset}[3][0ex]{
    \mathrel{\mathop{#3}\limits^{
		\vbox to#1{\kern-2\ex@
		\hbox{$\scriptstyle#2$}\vss}}}}

\pagenumbering{gobble}

\newcommand{\tbd}[1]{\textbf{\textcolor{magenta}{#1}}}
\newcommand{\cupdot}{\mathbin{\mathaccent\cdot\cup}}
\newcommand{\floor}[1]{\left\lfloor #1 \right\rfloor}
\newcommand{\ceil}[1]{\left\lceil #1 \right\rceil}
\newcommand{\interval}{\mathcal{I}}
\newcommand{\Cost}{\mathcal{C}}
\newcommand{\paths}{\mathcal{P}}
\newcommand{\cA}{\mathcal{A}}
\newcommand{\cB}{\mathcal{B}}
\newcommand{\cF}{\mathcal{F}}
\newcommand{\cX}{\mathcal{X}}
\newcommand{\fI}{\mathfrak{I}}
\newcommand{\rev}{\operatorname{rev}}
\newcommand{\Gmn}[2][]{G_{m#1,n,#2}}
\newcommand{\GL}{G^L}
\newcommand{\GM}{G^M}
\newcommand{\GR}{G^R}
\newcommand{\linkgadget}{\mathbf{L}}
\newcommand{\wt}{\mathsf{wt}}
\newcommand{\num}{\mathsf{num}}
\newcommand{\den}{\mathsf{den}}
\newcommand{\linkedge}{\mathtt{link}}
\newcommand{\tS}{\mathtt{S}}
\newcommand{\sols}{\mathfrak{L}}
\newcommand{\compl}{\varphi}
\newcommand{\complpoly}[1]{\varphi^{\operatorname{deg}(#1)}}
\newcommand{\complmult}[1]{\varphi^{\operatorname{lin}(#1)}}
\newcommand{\pl}{\operatorname{pl}}
\newcommand{\npl}{\operatorname{npl}}
\newcommand{\gr}{\operatorname{gr}}
\newcommand{\tw}{\operatorname{tw}}
\newcommand{\Zn}{\mathbb{Z}_n}
\newcommand{\Znhat}{\hat{\mathbb{Z}}_n}
\newcommand{\Expt}{\mathbb{E}}
\newcommand{\var}{\mathbf{var}}
\newcommand{\conv}{\mathbf{conv}}
\newcommand{\Grid}{\Upsilon}
\newcommand{\pred}{\mathsf{pred}}
\newcommand{\suc}{\mathsf{succ}}
\newcommand{\Coeff}{\Delta}
\newcommand{\vlambda}{\vec{\lambda}}
\newcommand{\va}{\vec{a}}
\newcommand{\vx}{\vec{x}}
\newcommand{\vy}{\vec{y}}
\newcommand{\vz}{\vec{z}}

\declaretheorem[numberlike=equation]{Theorem}
\declaretheorem[numberlike=equation]{Lemma}
\declaretheorem[numberlike=equation]{Corollary}
\declaretheorem[numberlike=equation]{Conjecture}
\declaretheorem[numberlike=equation]{Proposition}
\declaretheoremstyle[bodyfont=\it,qed=$\lozenge$]{defstyle}
\declaretheorem[numberlike=equation,style=defstyle]{Definition}
\declaretheorem[numberlike=equation]{Claim}
\declaretheorem[numberlike=equation]{Fact}

\makeatletter
\patchcmd{\ALG@step}{\addtocounter{ALG@line}{1}}{\refstepcounter{ALG@line}}{}{}
\newcommand{\ALG@lineautorefname}{Line}
\makeatother
\newcommand{\algorithmautorefname}{Algorithm}
\addto\extrasenglish{%
  \def\chapterautorefname{Chapter}%
  \def\sectionautorefname{Section}%
  \def\subsectionautorefname{Subsection}%
  \def\subsubsectionautorefname{Subsubsection}%
  \def\paragraphautorefname{Paragraph}%
  \def\subparagraphautorefname{Subparagraph}%
}

\newcommand\blankpage{%
	\null
	\thispagestyle{empty}%
	\addtocounter{page}{-1}%
	\newpage}

\setcounter{tocdepth}{3}
\setcounter{secnumdepth}{3}

%\setcounter{chapter}{2} % If you are doing your chapter as chapter one,
%\setcounter{section}{3} % comment these two lines out.

\hypersetup{
	colorlinks,
	linkcolor=red!75!black,
	citecolor=blue!75!black,
	urlcolor=green!75!black
}

%\title{Treewidth Characterizes Parametric Shortest Path Complexity}

\title{Insertion Sort and Spearman's Footrule}

\author[1]{Kshitij Gajjar\thanks{\texttt{kshitijgajjar@gmail.com}}}
\author[2]{Jaikumar Radhakrishnan\thanks{\texttt{jaikumar@tifr.res.in}}}

\affil[1]{National University of Singapore, Singapore}
\affil[2]{Tata Institute of Fundamental Research, India}

\renewcommand\Authands{ and }

%\setstretch{1.1}

\begin{document}
	
	%\onehalfspacing
	
	\maketitle
	
	%%%%%%%%%%%%%%%%%%%%%%%%%%%%%%%%%%%%%%%%%%%%%%%%%%%%%%%%%%%%%%%%%%%%%%%%%
	
	%%%%%%%%%%%%%%%%%%%%% A  B  S  T  R  A  C  T %%%%%%%%%%%%%%%%%%%%%%%%%%%%
	
	%%%%%%%%%%%%%%%%%%%%%%%%%%%%%%%%%%%%%%%%%%%%%%%%%%%%%%%%%%%%%%%%%%%%%%%%%
	
	\begin{abstract}
	In this short note, we exhibit a simple connection between Spearman's footrule (a measure of disorder in a permutation) and the insertion sort algorithm. The results regarding Spearman's measure were already proved by Diaconis \& Graham~\cite{diaconisgraham1977}. We provide an alternate proof using an accounting method based on insertion sort.
	\end{abstract}
	
	%%%%%%%%%%%%%%%%%%%%%%%%%%%%%%%%%%%%%%%%%%%%%%%%%%%%%%%%%%%%%%%%%%%%%%%%%
	
	%%%%%%%%%%%%%%%%% I  N  T  R  O  D  U  C  T  I  O  N %%%%%%%%%%%%%%%%%%%%
	
	%%%%%%%%%%%%%%%%%%%%%%%%%%%%%%%%%%%%%%%%%%%%%%%%%%%%%%%%%%%%%%%%%%%%%%%%%
	
	\section{Introduction}
	
	Insertion sort is a popular and elementary sorting algorithm. %Although its worst-case and average-case running time is $\Omega(n^2)$\footnote{Gapped insertion sort runs in $O(n\log n)$ time with high probability~\cite{bender2006}.} and there are several well-known sorting algorithms that run in $O(n\log n)$ time (merge sort, heap sort), insertion sort
	It is taught in almost every basic Algorithms course as a simple and natural approach to sorting.
	
	%Given an unsorted array as input, insertion sort partitions the array into a sorted part and an unsorted part. Initially the sorted part is empty. %The algorithm works by growing the size of its sorted part by one until the entire array is sorted.
	%At each iteration, insertion sort removes one element from the unsorted part by repeatedly swapping it with adjacent elements, finds its location in the sorted part, and inserts the element there. Once the unsorted part is empty, the algorithm terminates. For full details, see~\cite[Page 80]{knuth1998} or~\cite[Page 16]{cormen2022}.
	
	Given an unsorted array, insertion sort works by swapping adjacent elements in a systematic manner until the array is completely sorted. For full details and analysis of the insertion sort algorithm, see~\cite[Page 80]{knuth1998},~\cite[Page 16]{cormen2022}, or~\cite{enwiki:ainsertionsort}.
	
	Insertion sort (or any sorting algorithm) can be thought of as transforming a disordered (unsorted) input permutation to the fully ordered (identity) permutation in a step-by-step manner. Since insertion sort operates locally (it can only swap adjacent elements), the more the disorder in the input permutation, more the number of swaps needed to sort it.
	
	Let us formalize this notion. Let $\sigma$ be an input permutation on $n$ elements. Now, if an element $i$ is in position $j$ in $\sigma$, then clearly it needs to participate in at least $|j-i|$ adjacent swaps to reach its correct position $i$ in the sorted (identity) permutation. We denote this quantity by $l_i(\sigma)$ in our proof below. \emph{Spearman's footrule} is defined as $\sum_{i=1}^n l_i(\sigma)$. We ask: %Let $\textsc{Swaps}(\sigma)$ be the total number of swaps performed by insertion sort on $\sigma$.
	
	%\begin{itshape}\label{ques:spearman} Is Spearman's footrule related to the number of swaps performed by insertion sort?
	%\end{itshape}
	
	\vspace{0.3cm}
	\begin{mdframed}[backgroundcolor=blue!5]
	\begin{itshape} Are the number of swaps performed by insertion sort bounded by Spearman's footrule?
	\end{itshape}
	\end{mdframed}
	
	\section{Main Theorem and Proof}
	
	The above question was already answered affirmatively by Diaconis \& Graham 45 years ago~\cite{diaconisgraham1977}. The rest of our write-up is devoted to re-proving their result using insertion sort and interval graphs. Let us now commence our proof.
    
    %\begin{shadequote}[r]{Neha Sangwan putri Ashok Kumar}
        %Please, let's go for lunch.
    %\end{shadequote}
    
    \section{Concluding Remarks}
    
    Let us restate our main theorem.
    
    \begin{Theorem}
    If $\textsc{Swaps}(\sigma)$ is the number of swaps performed by insertion sort on $\sigma$, then
    $$\displaystyle{\frac{1}{2}\sum_{i=1}^n l_i(\sigma) \leq \textsc{Swaps}(\sigma) \leq \sum_{i=1}^n l_i(\sigma) - |X(\sigma)| + |Y(\sigma)|}$$
    \end{Theorem}
    
    It is easy to see that $|X(\sigma)|\geq |Y(\sigma)|$. Hence, if we denote Spearman's footrule by $\rho(\sigma)=\sum_{i=1}^n l_i(\sigma)$, then our main theorem implies that $$\frac{\rho(\sigma)}{2} \leq \textsc{Swaps}(\sigma) \leq \rho(\sigma),$$
    
    thereby confirming that Spearman's footrule bounds the number of swaps performed by insertion sort from both above and below. Showing the lower bound of $\rho(\sigma)/2$ is easy. However, even proving an upper bound of $c\rho(\sigma)$ for a large constant $c$ does not seem easy at first, which was one of the motivating reasons behind this write-up.
    
    It is easy to come up with several examples where the lower bound or the upper bound are tight, and we invite the reader to do so. For the even more inquisitive reader, we leave them with the following challenge.
    
    Catalan numbers are a sequence of natural numbers ubiquitous in combinatorics. Denoted by $C_n$, the $n$th Catalan number is defined (for all positive integers $n$) as follows. $$C_n = \frac{1}{n+1}\binom{2n}{n} = \frac{(2n)!}{(n+1)!\,n!} = \prod\limits_{k=2}^{n}\frac{n+k}{k}$$
    
    $C_n$ is equal to the number of ways of arranging $n$ pairs of correctly matched parentheses, the number of full binary trees with $n + 1$ leaves, the number of ways of triangulating a regular $(n + 2)$-sided convex polygon, and the number of permutations of $[n]$ with no three-term increasing subsequence. Catalan numbers also show up in many other (sometimes unexpected) places in the world of mathematics and computer science. See~\cite{koshy2008catalan,stanley2015catalan,enwiki:bcatalannumbers} for more on Catalan numbers. We are now ready to formally state our challenge.
    
    \vspace{0.3cm}
    \begin{mdframed}[backgroundcolor=blue!5]
	\begin{itshape} Show that the number of permutations $\sigma\in\mathcal{S}_n$ for which $\displaystyle{\textsc{Swaps}(\sigma)=\rho(\sigma)/2}$ is $C_n$.
	\end{itshape}
	\end{mdframed}
    \vspace{0.3cm}
    
    Finally, we are not the first to use Spearman's footrule as a metric in the context of insertion sort and other sorting algorithms. Some earlier works that might be worth exploring are~\cite{spin1,spin2,spin3,spin4}. For a fun\footnote{This paper appeared at the 3rd International Conference on FUN with Algorithms (FUN 2004).} way of handling the number of swaps in insertion sort, see ``Insertion sort is $O(n \log n)$''~\cite{bender2006}.
    
    \subsection*{Acknowledgements} We are grateful to Jatin Batra for coming up with the idea of representing the movement of each element across the array (or permutation) by an interval, using which he proved an upper bound of $2\rho(\sigma)$ on $\textsc{Swaps}(\sigma)$. We thank Meena Mahajan for informing us about the paper of Diaconis \& Graham~\cite{diaconisgraham1977}. We also thank Nikhil Mande and Suhail Sherif for the initial discussions on insertion sort.
	\bibliographystyle{alpha}
	\bibliography{ThirdPaper.bib}
	
\end{document}