%
% introduction.tex
%

\section{Introduction}
\label{sec:introduction}

What is our way of proving? Which ways of demonstrating the truth of a
statement are accepted as proof? What is the connection between
informal descriptions and formal proofs, is there any at all? 

Over the time various proof methods have emerged and fallen back into
oblivion. A lot of these lost methods were used to guide
mathematicians into new areas of knowledge, assure them of the truth
of their conjectures. 
To cite from \cite{polya1}:
% How do we set up new conjectures? How do we prove? This are questions
% which are relevant for any serious mathematician. The final outlook of
% a theorem, the final proof does rarely resemble the way how the idea
% for this theorem was found and how the proof was established. To cite
%from \cite{polya1}:
\begin{quotation}
  We secure our mathematical knowledge by \emph{demonstrative
  reasoning}, but we support our conjectures by \emph{plausible
  reasoning}. A mathematical proof is demonstrative reasoning, but the
  inductive evidence of the physicist, the circumstantial evidence of
  the lawyer, the documentary evidence of the historian, and the
  statistical evidence of the economist belong to plausible reasoning.

  The difference between the two kinds of reasoning is great and
  manifold. Demonstrative reasoning is safe, beyond controversy, and
  final. Plausible reasoning is hazardous, controversial, and
  provisional. Demonstrative reasoning penetrates the sciences just as
  far as mathematics does, but it is in itself (as mathematics is in
  itself) incapable of yielding essentially new knowledge about the
  world around us. Anything new that we learn about the world involves
  plausible reasoning [\ldots]
\end{quotation}

Accompanying the introduction of formal methods within mathematics and
logic less and less methods of proving where founded on a formal base
and therefore accepted as valid. One particular interesting example, and
at the same time one of the last ones, since it is from~1953, is
the use of physical notions in a proof of
Sch�tte~\citep{SchuetteWaerden:MA-125-325}, which would be impossible 
today. Nevertheless it is of fundamental proof theoretic interest to 
understand the relation between formal proofs and informal
descriptions. If we improve our understanding of these connections we
could on the one hand better understand old proofs using these methods
and on the other hand really use this methods in new proofs.

The use of sketches in geometry resembles exactly the description
given above. They were used for plausible reasoning \citep{polya1,polya2}.
Ever since ancient civilizations like the Chinese, Arian or Greek
cultures started to prove geometrical theorems there 
where different forms of proving, different ways of assuring the
truth of statements. All these different forms where accepted as
complete proving tools. The distinction between plausible and
demonstrative reasoning has not been made.
With the beginning of strict formalization of
mathematics and the use of formal methods within mathematics, some of
these proving techniques have vanished or have changed the way they
are used,
because they couldn't stand the
requirements of current science for a valid proving method. Some of
these methods still exist but have lost there importance because they
where not considered ``timely'' or formal enough to be used.

One of these methods is the use of sketches to prove facts of various
geometries. This method can look back to a long history as we will
point out below. We will undertake a small excursion in the history of
sketches and try to investigate this very old method of proving. In
this paper we
will analyze sketches within the framework of proof theory to
understand the connection between sketches and formal proofs.


Sketches are known as very useful to illustrate the facts of a proof
and to make the idea of a proof transparent. But sketches need not
only be just a hint, they can, in certain cases, be regarded as a
proof by themselves. \Pgq\footnote{We will understand under ``\pg'' the
\emph{plane} \pg and will loose the ``plane'' for simplicity.}
 is best for analyzing this relation between
sketches and proofs.

In newer times sketches have never been accepted as a proving tool.
The purpose of this paper is to show that sketches can be regarded as
proofs, i.e.\ that we can translate sketches into proofs and vice
versa. Additionally we will show interesting results concerning the
length of sketches compared to the length of proofs using standard
techniques like cut elimination. For this purpose we will use Herbrand
disjunctions, which will be analyzed in detail in sec.~\ref{sec:herbrand}.

% The purpose of this paper is to bring an idea of what a sketch can do
% and to explain the relations between sketches and proofs. Therefore
% we extend Gentzen's \lk to \pelk and use the properties of \pelk
% to formalize the concept of a sketch. We will then present a result on
% the equivalence of sketches and proofs and a result on the length of
% proofs with sketches and with proofs in \lk.

% In the proof of the equivalence of sketches and proofs we will use
% \HD. The reason for using this complicated and not well known concept
% of mathematical logic is the fact, that in this way proofs of speedup
% results are independent from any calculus.

\subsection{Historic Examples}
\label{sec:introduction:history}

Proofs of geometric propositions are found in the earliest known
mathematical texts in India and China as well as in Greece. Let us
give some examples (taken from \cite{waerden-ancient}):

In the \=Apastamba \'Sulvas\=sutra, an altar is described in the form
of an isosceles trapezium with its eastern base 24~units, its
western~30, its width~36. The text says that the area is 972~square
units. This is proved as follows (c.f.~fig.~\ref{fig:trapezium}):

\begin{figure}[hbt!]
  \[\epsfbox{diagonal.1}\]
  \caption{Area of a trapezium}
  \label{fig:trapezium}
\end{figure}

\begin{quotation}
{\small One draws (a line) from the southern \emph{amsa} ($D$ in Fig.)
toward the southern sr\=oni ($C$), (namely) to (the point~$E$ which
is) 12 (\emph{padas} from the point~$L$ of the
\emph{prsthya}). Thereupon one turns the piece cut off (i.e.\ the
triangle~$DEC$) around and carries it to the other side (i.e.\ the
north). Thus the \emph{vedi} obtains the form of a rectangle. In this
form ($FBED$) one computes its area.}
\end{quotation}

The translation is taken from \cite{seidenberg}, p.~332, where there
is also an interesting comment:

\begin{quotation}
        The striking thing is that here we have a proof.
\end{quotation}

As another example let us present a passage from the earliest Chinese
text on astronomy and mathematics, the \emph{Chou Pei Suan Ching},
which may be translated to ``Arithmetical Classic of the Gnomon and
the Circular Paths of Heaven''. In this classical text from the
Han-period a proof of the ``Theorem of Pythagoras'' is presented. The
proof is only worked out for the $(3,4,5)$ triangle, but the idea of
the proof is perfectly general.  Figure~\ref{fig:chou} and the
translation are taken from \cite{needham}:

\begin{figure}[ht]
  \[\epsfxsize=6cm
  \epsfbox{chinabild.eps}
  \]
  \caption{The proof of the Pythagorean Theorem in the Chou Pei Suan
  Ching. From J.~Needham: Science and Civilization in China, Vol.~3}
  \label{fig:chou}
\end{figure}

{\small
\begin{quotation}
  Thus, let us cut a rectangle (diagonally), and make the width
  3~(units) wide, and the length 4~(units) long. The diagonal between
  the (two) corners will then be 5~(units) long. Now after drawing a
  square on this diagonal, circumscribe it by half-rectangles like
  that, which has been left outside, so as to form a (square)
  plate. Thus the (four) outer half-rectangles of width~3, length~4,
  and diagonal~5, together make two rectangles (of area~24); then
  (when this is subtracted from the square plate of area~49) the
  remainder is of area~25. This (process) is called ``piling up the
  rectangles''. 
\end{quotation}
}

These examples are of course not the only ones. All the proofs in
Euclid's ``Elements'' can be viewed as sketches. But sketches as
proving tools are not limited to ancient times, there are a lot
theorems in geometry which are obvious from a sketch, take for example
Dandelin's theorem \citep{ogilvy} on the intersections of a plane with
a cone or Robert's theorem. 

\subsection{Present state}
\label{sec:introduction:present}

Today sketches lost most of their importance \emph{as proving tools by
themselves}
and are more or less only
used to explain the ideas of a proof and to help understanding, but
all the proofs have to be formalized in a strict sense. On the other
hand there are some approaches to these area which try to reintroduce
sketches as proofs: For example Gr�nbaum introduced a
notion of provability in elementary geometry which is based on
sketches made with \textsc{Mathematica} \citep{mathematica}. Other
approaches of proving with the help of sketches can be found in
programs like \emph{Geometer's Sketchpad} \citep{sketchpad} or
\emph{DrGeo} \citep{drgeo}.

\subsection{Why Projective Geometry}
\label{sec:introduction:whypg}

We choose projective geometry out of various reasons: The main one is
the fact, that projective geometry is the most simple one, since the
only concept is incidence, i.e.\ ``point lying on a line'', while
other geometries, like affine or even higher geometries introduce more
and more concepts like parallel, distance, measurement. But although
projective geometry lacks concepts like distance and angle, it has great
expressive power (see~sec.~\ref{sec:orevkov}).

One step of sketching is to assume points or lines in a ``general''
position. This means that this new point should not be placed in some
special relation to the other ones. If you analyzed how many
``special'' points there are in a triangle, it is obvious that it is
difficult to put a new point into it which is ``for sure'' not one of
these special points. In projective geometry the assumption of general
position is easier, just because of this lack of higher concepts.

Finally we want to mention that projective geometry is also easily
axiomatized, it is in fact defined by only three axioms.  

\subsection{Layout of the article}
\label{sec:introduction:layout}

Section~\ref{sec:herbrand} presents an analysis of Herbrand disjunction.\\
Section~\ref{sec:sketch} gives an introduction to projective geometry,
defines sketches and gives a simple example of a sketch.\\
Section~\ref{sec:herbrandsketch} gives a proof that sketches and
Herbrand disjunctions are equivalent by giving a translation from one
to the other.\\
Section~\ref{sec:orevkov} analyzes special cases where  sketches are much
worse to be used as proving tool than formal proofs due to the fact
that there can be a high speedup using cut elimination. 

%%% Local Variables: 
%%% mode: latex
%%% TeX-master: "paper"
%%% End: 
