%
% recursive.tex
%

\section{G�del, recursive functions and the theory of $\N$}
\label{sec:goedel}

According to G�del in every theory, which contains the operation
``+'', ``$\cdot$'' 
and the relation ``<'' and in which the natural numbers can be defined, all
the recursive functions can be represented (see~\cite{shoenfield} for a
extensive discussion). G�del gives also a transformation from recursive
functions to the representing formula.

\subsection{Recursive functions}
\label{sec:goedel:recursive}

We will give a crash-course on recursive functions and therein we follow~\cite{shoenfield}:

\begin{definition}
If $P$ is an $n$-ary predicate, the representing function $K_P$ is defined by
        $$K_P(\ger a)=\left\{%
          \begin{array}{ll}
                0 & \textrm{if $P(\ger a)$}\\
                1 & \textrm{if $\lnot P(\ger a)$}\\
          \end{array}\right.$$

        $$I_i^n(a_1,\ldots,a_n)=a_i$$
\end{definition}

\begin{definition}
\emph{Recursive functions} are defined by inductive definition
\begin{enumerate}
\item[R1] The $I_i^n$, $+$, $\cdot$, and $K_<$ are recursive.
\item[R2] If $G, H_1, \ldots, H_k$ are recursive, and $F$ is defined by
        $$F(\ger a)=G(H_1(\ger a),\ldots,H_k(\ger a)),$$
        then $F$ is recursive.
\item[R3] If $G$ is recursive and $\qa{\ger a}\qe x(G(\ger a,x)=0)$, and $F$
        is defined by
        $$ F(\ger a) = \qm x(G(\ger a,x)=0),$$
        then $F$ is recursive.
\end{enumerate}
\end{definition}

\begin{definition}
A predicate $P$ is recursive is its representing function is recursive.
\end{definition}

\begin{proposition}
Every constant function is recursive. If $P$ and $Q$ are recursive, then
 $\lnot P$, $P\limp Q$, $P\land Q$, $P\lor Q$ are recursive. The predicates
 $<$, $\le$, $>$, $\ge$, and $=$ are recursive.
\end{proposition}

G�del gives a binary recursive function for converting sequences into
sequence-numbers:
\begin{proposition}
There is a binary recursive function $\beta$ such that\footnote{In fact the
minus below should be the minus in the natural numbers, where $0-1=0$.}
        $$ \beta(a,i)\le a-1$$
for all $a$ and $i$, and such that for any numbers $a_0,\ldots,a_{n-1}$ there
is a number $a$ such that $\beta(a,i)=a_i$ for all $i<n$.
\end{proposition}


% \subsection{Representability}
% \label{sec:goedel:representability}

% Every recursive function or predicate can, in a suitable sense, be calculated
% in the theory $N$, consisting only from +, $\cdot$ and $<$.
% \begin{definition}
% We use $\knum_n$ as a name for the numeral for $n$\footnote{In $N$ this is
% $SS\cdots S0$, in projective geometry we can take the terms defined as $0$,
% $1$, \ldots}.

% Let $F$ be an $n$-ary function, $A$ a formula of $N$, $x_1,\ldots, x_n, y$
% distinct variables. We say that \emph{$A$ with $x_1,\ldots,x_n,y$
% represents~$F$} if for every $a_1,\ldots,a_n$
%         $$\provable A_{x_1,\ldots,x_n}[\knum_{a_1},\ldots,\knum_{a_n}]
%                 \lequ y = \knum_b,$$
% where $b=F(a_1,\ldots,a_n)$.

% Let $P$ be an $n$-ary predicate, $A$ a formula of $N$, $x_1,\ldots, x_n, y$
% distinct variables. We say that \emph{$A$ with $x_1,\ldots,x_n,y$
% represents~$P$} if for every $a_1,\ldots,a_n$
%         $$P(a_1,\ldots,a_n) \to
%                 \provable A_{x_1,\ldots,x_n}[\knum_{a_1},\ldots,\knum_{a_n}]$$
% and
%         $$\lnot P(a_1,\ldots,a_n) \to
%                 \provable \lnot
%                 A_{x_1,\ldots,x_n}[\knum_{a_1},\ldots,\knum_{a_n}].$$
% \end{definition}

% \begin{proposition}
% Every recursive function is representable.
% \end{proposition}

% \beginproof (Sketch)
% The basic function $I_i^n$ is represented by $x_i$ with $x_1,\ldots,x_n$. The
% functions $+$ and $\cdot$ are trivially representable, $<$ too.

% Let $x_1,\ldots, x_n, y_1,\ldots, y_k,z$ be distinct variables. Choose $A_i$
% such that $A_i$ with $x_1,\ldots,x_n,y_i$ represents $H_i$ and choose~$B$ such
% that $B$ with $y_1,\ldots,y_k,z$ represents $G$. Then
%         $$\qei y1\ldots\qei yk(A_1\land\cdots\land A_k\land B)$$
% with $x_1,\ldots,x_n,z$ represents $F$ defined as
%         $$F(\ger a)=G(H_1(\ger a),\ldots,H_k(\ger a)).$$

% Suppose that $F$ is defined by
%         $$F(\ger a)=\qm x(G(\ger a,x)=0),$$
% where $G$ is representable. Let $A$ with $x_1,\ldots,x_n,y,z$ represent
%  $G$. Let $w$ a new variable and let $B$
%         $$A_x[0] \land \qa w(w<y\limp\lnot A_{y,z}[w,0]).$$
% Then $B$ with $x_1,\ldots,x_n,y$ represents $F$. \endproof

% \beginexample As an example we will show that the function~$2^x$ is
% representable in the theorie of natural numbers. We are looking for a
% predicate~$G$ such that
%   $$\provable G_x[\ka_a] \equiv \ka_b \qquad \mbox{if~$b=2^a$}$$
% With G�del's~$\beta$-function we can compute sequence numbers and
% select entries out of a sequence number. Therefore, we look at the
% following sequence number:
%   $$u = \lceil 2^0,2^1,\ldots,2^a\rceil.$$
% This can be accomplished by setting~$u$ to
% \begin{multline*}
% u = \qm u (\Seq u \land \lh u = \ka_a+1 \land \goi u0 = 1 \land\\
%   \land \qai i{\ka a} (i\not=0\limp\goi k{i+1}=2\goi ki))
% \end{multline*}
% and computing
%   $$\beta(u,\ka_a)$$
% which is the sought value. To reduce the resulting formula we have to
% transform the use of~$\mu$, $\Seq u$, $\lh u$ and the index
% operations~$\goi ui$ into predicates, which indeed is possible, since
% the $\beta$-function can be represented.

\endinput
%%% Local Variables: 
%%% mode: latex
%%% TeX-master: "paper"
%%% End: 
