\documentclass[matlog,final]{svjour}
\usepackage{current-macros}
\def\atom{\mbox{Atom}}
\let\true\perp
\begin{document}

\begin{center}
\Large\bfseries
Minimization and Reskolemization
\end{center}

\bigskip

\begin{definition} 
For every syntactic component of a formula (function symbols,
variables, predicate symbols, conjunctors, quantors) there is a unique
signature called the \emph{position} which describes the position
within the formula. For a position~$p$ corresponding to a predicate
symbol, the operator~$\atom(A,p)$ yields the atomic formula for this
predicate symbol-
\end{definition}

Our aim is the following: For a given formula~$A'$ and a given Herbrand
Disjunction~$H$ for the skolemization of the formula~$A'$ we want to
find another, possibly different, Herbrand disjunction, which has
terms with smaller depth and which is, in some sense, minimal. We will
explain this concept of minimality later on.

Let this Herbrand Disjunction be the formula~$H$:
        $$H = A(\vec t_1, \vec S_1) \lor \ldots \lor A(\vec t_n, \vec S_n)$$
where $S_i$ are Skolem terms and~$t_i$ are (regular) terms.

Now we substitute new variables into the formula~$H$ for the~$t_i$ and
obtain~$H^\#$:
        $$H^\# = A(\vec x_1, \vec S_1) \lor \ldots \lor A(\vec x_n, \vec S_n)$$
Since we only changed terms the positions of atomic formulas
within~$H^\#$ and~$H$ are the same. We construct the set~$M$ of all
pairs of positions, whose respective atomic formulas are equal in~$H$,
i.e.: 
        $$M = \{ (p_i, p_j) | \atom(H,p_i) = \atom(H,p_j)\}$$
And finally we construct the equality system~$G$ consisting of all the
atomic formulas in~$H^\#$, which are equal in~$H$:
        $$G = \{ \atom(H^\#,p_i) = \atom(H^\#, p_j)\quad |\quad (p_i,p_j) \in M\}$$
This equality system has a solution, namely the original substitution
which we used to obtain the skeleton~$H^\#$ from~$H$.

This equality system describes the connections between the atomic
formulas within~$H$ which render~$H$ true, i.e. make a tautology out
of it. 

But a tautology can be true out of various reasons, for example take a
long formula with vary complicated parts which is a tautology and then
add an~$\lor\true$. If we `compute' the truth of this new formula with
the complicated first part it will be very difficult, if we take
the~$\lor\true$ part, it will be very easy. And exactely the same
situation can occur with the above tautologies. Therefore we try to
find a minimal set of connections which render the Herbrand
Disjunction true. As a result of this we also obtain terms which are
for sure not longer than the original ones.

We now take one element~$g$ from the power set of~$G$, ie.\ a
subsystem of~$G$. Every element of the power set also has at least one
solution, namely the projection of the original substitution, but
there may be more. Now we take a look at all the solutions of~$g$ and
check for the property, that the solution produces a tautology
from~$H^\#$. If this is the case we have obtained a more simple
connection set within the Herbrand disjunction.
%\begin{itemize}
%\item Die L�sungssubstitution macht aus~$H^\#$ eine Tautologie.
%\item Von dieser Tautologie l�sst sich durch Reskolemisierung die
%        urspr�ngliche Formel erstellen. 
%\end{itemize}

We will call an equality system~$g$ with such a solution
\emph{alternative}.

All these alternative equality systems make up a tree where the root
is the original~$G$ and going down the tree gives smaller and smaller
systems. In each branch there is a~$g$ which is the smallest one. From
all these~$g$ we can take one of those, which have the smallest number
of equalities, i.e.\ the smallest number of necessary connections to
render the Herbrand Disjunction true.

Finally we want to show how it is possible to obtain the original
formula~$A'$ from the modified Herbrand Disjunction, ie.\ from a
Herbrand Disjunction obtained by apllying a solution of an alternative
equality system to the skeleton~$A^\#$. The process is called
\emph{reskolemization}.

Let~$G$ be this modified Herbrand Disjunction and~$A'$ the original
formula. Every position of a quantified variable in~$A'$ matches with
more (depended on the number of disjunctions in~$G$) positions
in~$G$. We call a position~$p$ of a quantified variable in~$G$
\emph{weak} or \emph{strong} dependend on the status of the matching
variable in~$A'$.

\begin{algorithm}[Reskolemization of infix formulas]
\begin{enumerate}
\item Fix a sequence of all Skolem terms with decreasing term depths
\item Take the first term and search all weak positions where this
        term occurs. 
\item Introduce an existential quantifier for this term and shift it
        inward as long as possible.
\item Introduce the strong quantifier for this term and shift it
        inward as long as possible. If this is not possible start with
        a new sequence. 
\item Restart with one of the deepest terms.
\end{enumerate}
\end{algorithm}

This procedure is non-deterministic because there may be more valid
sequences of Skolem terms. The case that it is not possible to
introduce the strong quantifier because of the eigenvariable condition
can happend when we have choosen to start with the wrong Skolem terms,
ie.\ we tried to introduce `wrong' quantifiers first.


\end{document}



%%% Local Variables: 
%%% mode: latex
%%% TeX-master: t
%%% End: 
