%
% File:         closing.tex
% Date:         Thu Oct  1 13:55:34 1998
% Author:        (norbert)      
%


\section{Closing Comments}
\label{sec:closing}

As a conclusion we would like to mention some points which seem to be
relevant for us:

\textbf{Sketches lead from special cases to general ones} 
This analysis presents sketches as tools to progress from specialized
problems to general statements. We described a method to obtain
general results from special instances. This is similar to the
generalization of Eulers calculation in~\cite{Baaz99THEC}.

\smallskip
\textbf{Sketches are not constructive} As can be seen from the way
sketches are drawn and the way they are translated to Herbrand
disjunctions it is easy to see that sketches are \emph{not}, as the
name suggests, constructive. All this ``un-constructiveness'' arises
from the concept of \emph{general position}, which indeed becomes more
difficult to be explained when geometries with higher concepts
(distance, angle, \ldots) are introduced. Since Projective Geometry
only makes use of the notion of 
incidence the assumption of a Point in general position is easier to
realize than elsewhere in geometry. The concept of general position
lacks constructiveness due to atomic case distinctions which have to
be made. It is the principle of the excluded middle on atomic level,
which make constructions not constructive.

\smallskip
\textbf{Theoretical foundations for computer tools for geometric
reasoning} As 
the use and speed of computers increases they become more and more
in use as proving tools with the help of programs like
Mathematica etc. These programs are already applied to prove
theorems and to draw sketches. This article provides a theoretical
framework for such applications. \citep{sketchpad,drgeo,mathematica}

%\paragraph{Sketches can be considered as proofs} Although the ``proof
% contents'' is well hidden within the sketches, the simplicity of
% Projective Geometry allows to extract informations from specific
% examples, i.e.\ sketches and obtain proofs providing more abstract
% result. This is the sort of proofs which have been used in earlier
% times.


\endinput

%%% Local Variables: 
%%% mode: latex
%%% TeX-master: "paper"
%%% End: 
