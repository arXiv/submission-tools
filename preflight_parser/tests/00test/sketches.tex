%
% sketches.tex
%

\section{Sketches in Projective Geometry}
\label{sec:sketch}

Most of the proofs in \pg are illustrated by a sketch. But this method
of a graphical representation of the maybe abstract facts is not only
used in areas like \pg, but also in other fields like algebra,
analysis and I have even seen sketches to support understanding in a
lecture about large ordinals, which is highly abstract!

The difference between these sketches and the sketches used in \pg
(and similar fields) is the fact, that the proofs in \pg deal with
geometric objects like Points and Lines, which are indeed objects we
can imagine and draw on a piece of paper (which is not necessary true
for large ordinals).

So the sketch in \pg has a more concrete task than only illustrating
the facts, since it exhibits the incidences, which is the only
predicate constant besides equality really needed in the normalization
of \pg. It is a sort of proof by itself and so potentially interesting
for a proof-theoretic analysis.


\subsection{Introduction to Projective Geometry}
\label{sec:sketch:intro}

\subsubsection{Historical Background\label{:projective:history}}
The earliest systematic method used in the study of geometry was the
deductive axiomatic method introduced by the Greek. Thales
(640-546 {\scshape b.c.}) is generally considered to be the first to
treat geometry as a logical structure. In the next 300~years much
geometric knowledge was developed. Then Euclid (\textit{c.}~300
{\scshape b.c.}) collected and systematized all the geometry previously
created. He did this by starting out with a set of axioms, statements
to be accepted as ``true'', from which all theorems were deduced as
logical consequences.

Though the Greek realized the need for axioms, they did not seem to
find a logical need for undefined terms. Euclid therefore attempted to
define everything. The assumptions which
Euclid used in his proofs were not all stated explicitly. For example,
there is nothing in Euclid's axioms from which we can deduce that an
angle bisector of a triangle will intersect the opposite side. 
Further, where constructions demand the intersection of two circles, or of
a line and a circle, Euclid simply assumed the existence of the needed
points of intersection. Though there were attempts to improve on
Euclid's definitions and axioms, nevertheless,
Euclid reigned supreme  until the 19th century. Then came the
discovery of non-Euclidean 
geometry and with it a re-examination of the foundations of Euclidean
geometry.

%
% subsubsection Euclidean Axiom of Parallelism
\paragraph{The Euclidean Axiom of Parallelism}
The first four of Euclid's axioms were
accepted as simple and ``obvious''. The fifth, however, was
not. Euclid proved his 28 propositions without using the fifth
axiom. For 2000~years mathematicians tried to prove this axiom; i.e.,
tried to deduce it from the other axioms and the first
28~propositions. But they only succeeded in replacing it by various
equivalent assumptions.

In the 19th century the conclusion was reached that not only could the
parallel postulate not be proved, but that a logical system of
geometry could be constructed without its use. Up to this point no one
thought of arguing against the ``truth'' of Euclid's  parallel
postulate. But in the 19th~century the founders of non-Euclidean
geometry---Carl Friedrich Gauss (1777--1855),
Nicolai Ivanovitch Lobachevsky (1793--1856),
and Johann Bolyai (1802--1860)---concluded
independently that a consistent geometry denying Euclid's parallel
postulate could be set up.

Gauss, from 1792 to 1813, tried to prove Euclid's parallel
postulate, but after 1813 his letters show that he had overcome the
usual prejudice and developed a non-Euclidean geometry. But, fearing
ridicule and controversy, he kept these revolutionary ideas to
himself, except for letters to his friends. Lobachevsky and Bolyai
were the first to publish expositions of the new geometry, Lobachevsky
in 1829 and Bolyai in 1832. This geometry is known today as
Hyperbolic Geometry. In 1854 Bernhard Riemann
(1826--1866) developed another non-Euclidean
geometry, known as Elliptic Geometry.

%
% subsubsection Hilbert
\paragraph{Hilbert and the new approach to Geometry}
The next great step in the development of the logic of geometry was
the break with the long hold tradition of defining everything
mathematics or geometry speak about.
It's not easy to see the great step, but the fact of treating
Points and Lines as primitives dispenses you of the
awful duty to define Points as ``impartable objects'', ``the entity
in space'' and all the other interesting definitions Greek
philosophers invented. 
Only a few undefined objects and relations are assumed as primitives
and the axioms determine the ``behavior'' of them.

Though defects in Euclid's logical structure were pointed out
earlier, it was not until after the discovery of non-Euclidean
geometry that mathematicians began carefully scrutinizing the
foundations of Euclidean geometry and formulating precise sets of
axioms for it. The problem was to erect the entire structure of
Euclidean geometry upon the simplest foundation possible; i.e., to
choose a minimum number of undefined elements and relations and a set
of axioms concerning them, with the property that all of Euclidean
geometry can be logically deduced from these without any further
appeal to intuition. There were many such axiom sets formulated at the
end of the 19th century beginning with the work of Pasch (1882), known
for the Pasch Axiom, Peano (1889) and Pieri (1899) and culminating
with the famous set by David Hilbert (1899, cf.~\cite{hilbert-geometrie}).

The roots of \pg can be traced back to the ancient Greek who knew some of
the theorems as part of Euclidean geometry. Its formal development
probably started in the 15th century by artists who were
looking for a theory of perspective drawing; i.e., the laws of
constructing the projections of three-dimensional  objects on a
two-dimensional plane. The theory was extended by Desargues
(1593--1662), an engineer and architect who, in 1639, published a
treatise on conic sections using the concept of projection. It was
here that Desargues used the idea of adding one point ``at infinity''
to each line with the locus of these ``ideal points'' forming an
``ideal line'', added to the Euclidean plane, where parallel lines
were to intersect. However, it was not until Monge (1746--1818), with
his co-workers at the Ecole Polytechnique in Paris, developed his
descriptive geometry---the analysis and representation of
three-dimensional objects by means of their projections on different
planes---that the study of \pg began to flourish.

Mathematicians classified geometric properties into two categories:
\emph{metric} properties, which are those concerned with measurements
of distances, angles, and areas, and \emph{descriptive} properties,
which are those concerned with the positional relations of geometric
figures to one another. For example, the length of a line segment and
the congruence of three lines are metric properties, but the
collinearity of three points and the concurrence of three lines are
descriptive properties. In the case of plane figures, descriptive
properties are preserved when a figure is projected from one plane
onto another (provided we consider parallel lines as intersecting at
an ``ideal point''), while metric properties may not be preserved.
Thus the property of a given curve being a circle is a metric property
but that of its being a conic is a descriptive or projective property.

The beginning of the modern period of the development of \pg is
usually placed at 1822 when Poncelet (1788--1867), a pupil of Monge,
published his great treatise on the projective properties of figures,
written while he was a prisoner in Russia. Throughout the 19th~century,
the subject was developed rapidly by Gergonne, Brianchon, Pl\"ucker,
Steiner, Von Staudt and others.

For the most part, however, \pg was developed as an extension of
Euclidean geometry (cf.~\ref{:projective:examples:piae}); e.g., the
parallel postulate was still used and a line was added to the
Euclidean plane to contain the ``ideal points'' mentioned above. It
was only at the end of the 19th~century and the beginning of the
20th~century, through the work of Felix Klein (1849--1925), Oswald
Veblen (1880--1960), David Hilbert, and others, that \pg was seen to be
independent of the theory of parallels. Projective geometry was then
developed as an abstract science based on its own set of axioms.

%%%%%%%%%%%%%%%%%%%%%%%%%%%%%%%%%%%%%%%%%%%%%%%%%%%%%%%%%%%%%%%%%%%%%%%
% DEFINITION
%
\subsubsection{What is Projective Geometry\label{:projective:definition}}
We will now give an axiomatization of \pg in the sense of the previous
sections.
The \pg deals, like the Euclidian geometry, with points and
lines. These two elements are primitives, which aren't further
defined. Only the axioms tell us about their properties. The
axioms for the \pg are very simple, the reason why I chose this
geometry for a proof-theoretic analysis. 

Now let me begin with the definition of the \pg: There are two classes
of objects, 
called {\em Points\/} and {\em Lines}\footnote{We will use the
expression ``Point'' (note the capital P) for the objects of \pg and
``points'' as usual for e.g.\ a point in a plane. The same applies to
``Line'' and ``line''.}, and one
predicate, that puts up a
relation between Points and Lines, called {\em Incidence}.

Furthermore we must give some axioms to express
certain properties of Points and Lines and to specify the behavior of
the incidence on Points and Lines:
\begin{itemize}
\item \pei For every two distinct Points there is one and only
one Line, so that these two Points incide with this Line.
\item \peii For every two distinct Lines there is one and only
one Point, so that this Point incides with the two
Lines\footnote{``one and only one'' can be replaced by ``one'',
because the fact that there is not more than one Point can be proven from
axiom (PE1).}.
\item \peiii There are four Points, which never incide with a
Line defined by any of the three other Points.
\end{itemize}

The next chapter will present some examples of Projective Planes.


%%%%%%%%%%%%%%%%%%%%%%%%%%%%%%%%%%%%%%%%%%%%%%%%%%%%%%%%%%%%%%%%%%%%%%%
% examples for projective planes
%
\subsubsection{Examples for Projective Planes%
\label{:projective:examples}}

%
%
\paragraph{The projective closed Euclidean plane $\piae$%
\label{:projective:examples:piae}}
The easiest approach to  \pg is via the Euclidean plane. If we add
one Point ``at infinity'' to each line and one ``ideal Line'',
consisting of all these ``ideal Points'', it follows that two
Points determine exactly one Line and two distinct Lines determine
exactly one Point\footnote{More precise: The ``ideal Points'' are the
congruence classes with respect to the parallel relation and the
``ideal Line'' is the class of these congruence classes.} 
(cf.~\ref{skizze:piae}).
So the axioms are satisfied.

This projective plane is called $\piae$ and has a lot of other
interesting properties, especially that it is a classical projective
plane.

\begin{figure}[ht] \[
   \left.\hbox to 6cm{$\vcenter{\epsfbox{diagonal.2}}$}\right\}\]
  \caption{``Ideal Points'' in $\piae$\label{skizze:piae}}
\end{figure}

\paragraph{The projective Desargues-Plane%
\label{:projective:examples:desargues}}
A very well known property of $\piae$\ is the Desargues' Theorem. To
understand it, some definitions (cf.~fig.~\ref{skizze:desargues}):

\begin{definition} Two triangles are said to be \emph{perspective from
a Point $O$} if there is a one-to-one correspondence between the
vertices so that Lines joining corresponding vertices all go through
$O$. Dually, two triangles are said to be \emph{perspective from a
Line} $o$ if there is a one-to-one correspondence between the sides of
the triangles such that the Points of intersection of corresponding
sides all lie on $o$.
\end{definition}

\begin{theorem}[Desargues' Theorem] If two triangles are perspective
from a Point, then they are perspective from a Line.
\end{theorem}

\begin{figure}[ht]
  \[\epsfbox{diagonal.9}\]
  \caption{Desargues' Theorem}
  \label{skizze:desargues}
\end{figure}

Finally an example for a finite Projective Plane:
\paragraph{The minimal Projective Plane%
\label{:projective:examples:minimal}}
One of the basic properties of projective planes is the fact, that
there are seven distinct Points. Four Points satisfying axiom \peiii and the 
three diagonal Points $([\ao\bo][\co\deo])=: D_1$, $([\ao\co][\bo\deo])=:
D_2$ and $([\ao\deo][\bo\co])=:D_3$. If we can set up a relation of
incidence on these Points such as that the axioms \pei and \peii are
satisfied, then we have a minimal projective plane. 
Fig.~\ref{skizze:minimal} defines such an incidence-table. In this
table not only the usual lines are Lines for the projective geometry,
but also the circle. 

\begin{figure}[ht] 
  \[\epsfbox{diagonal.10}\]
  \caption{Incidence Table for the minimal Projective Plane%
    \label{skizze:minimal}}
\end{figure}

We could attribute a number called ``order'',
which is the number of Points on a Line minus one, to every finite
projective plane. Then there is the
question for which number $n$ there is a projective plane with order $n$.
One partial solution for this problem depends on the existence of finite
fields. If we have a finite field, we can construct a finite
projective plane with the same order. Since there are finite fields
for every power of a prime\footnote{The so called Galois Field GF($n$).},
for any such number there is also a projective plane. 

Principally these questions can be answered simply by trying all 
possible relation tables with $n$ Points and $n$ Lines and look, whether
there is one satisfying the axioms. But this method is much too
difficult to do, because the number of such tables rises exponentially.

%%%%%%%%%%%%%%%%%%%%%%%%%%%%%%%%%%%%%%%%%%%%%%%%%%%%%%%%%%%%%
% some consequences
%
\subsubsection{Some Consequences of the Axioms%
\label{:projective:consequences}}

\begin{itemize}
\item There are seven distinct Points in a projective plane, namely
        the four constants $\ao,\dots,\deo$ and the three diagonal
        Points\\ $D_1=([\ao\bo][\co\deo]), D_1=([\ao\co][\bo\deo]),
        D_3=([\ao\deo][\bo\co])$.
\item For each Line there are three distinct Points which incide with
        this Line.
\item For distinct Lines $g$ and $h$ there is a Point $P$ such
        that $P\notcI g$ and $P\notcI h$.
\item There is a one-to-one mapping from the set of Points to the set
        of Lines in a projective plane.
\item If there are exactly $n+1$ distinct Points on a Line, then on
        every Line there are $n+1$ distinct Points, for each Point there
        are exactly $n+1$ different Lines passing through it and there
        are exactly $n^2+n+1$ Points and exactly that much Lines.
\end{itemize}

%%%%%%%%%%%%%%%%%%%%%%%%%%%%%%%%%%%%%%%%%%%%%%%%%%%%%%%%%
%%%%%%%%%%%%%%%%%%%%%%%%%%%%%%%%%%%%%%%%%%%%%%%%%%%%%%%%%%

\subsection{An example for a sketch}

As a first example I want to demonstrate a proof of \pg, which is
supported by a sketch. It deals with a special sort of mappings, the
so called ``collineation''. This is a bijective mapping from the set
of Points to the set of Points, which preserves collinearity. In a
formula:
$$\koll(R,S,T)\limp\koll(R\kappa,S\kappa,T\kappa)$$
The fact we want to proof is
$$\lnot\koll(R,S,T)\limp\lnot\koll(R\kappa, S\kappa, T\kappa)$$
That means, that not only collinearity but non-collinearity is preserved
under a collineation as well.

The proof is relatively easy and is depicted in
fig.~\ref{skizze:proof1}: If $R\kappa$, $S\kappa$ and $T\kappa$ are
collinear, then there exists a Point $X'$ not incident with the Line
defined by $R\kappa$, $S\kappa$, $T\kappa$. There exists a Point $X$,
such that $X\kappa = X'$. This Point $X$ doesn't lie on any of the
Lines defined by $R$, $S$, $T$. Let $Q=([RT][XS])$ then $Q\kappa\cI
[R\kappa S\kappa]$ and $Q\kappa\cI [S\kappa X\kappa]$, that is
$Q\kappa\cI [S\kappa X']$ (since collinearity is preserved). So
$Q\kappa = S\kappa$ (since $Q\kappa = ([R\kappa S\kappa][S\kappa X']) =
S\kappa$), which is together with $Q\not=S$ a contradiction to the
injectivity of $\kappa$.

\begin{figure}[ht]
  \[\epsfbox{diagonal.3}\]
  \caption{Sketch of the proof
        $\lnot\koll(R,S,T)\limp\lnot\koll(R\kappa, S\kappa, T\kappa)$}
  \label{skizze:proof1}
\end{figure}

This sketch helps us to understand the relation of the geometric
objects and you can follow the single steps of the verbal proof.

If we are interested in the concept of the sketch in mathematics in
general and in \pg in special then we must set up a formal description
of what we mean by a sketch. This is necessary if we want to be more
concrete on facts on sketches. So we come to $\ldots$

%
% subsection
% formalization of sketches
\subsection{A Formalization of Sketches in
Projective Geometry\label{:sketch:formalisation}}

In this part we want to give a formalization of the
sketch\index{sketch!formalization} in \pg and
want to explain our motivation behind some of these concepts.

All Points and Lines are combined in the sets called
$\typpo$ and $\typli$, respectively. So if we say, that $x\in\typpo$,
then we mean that $x$ is from type Point, means it's any term which
describes a Point.

Sketches speak about geometric objects, that are Points and Lines. So
the first logical objects necessary for the formalization are
constants or variables for Points and Lines. From these constants we
could build more and more complex objects by connecting and
intersection. This step is described in

\begin{definition}[Set of Terms over $\cC$]\label{definition:terms}
Let $\cC$ be a set of constants of type $\typpo$ or $\typli$, than
$\cT_n$ is inductively defined
%\begin{align*}
%\qquad &\bullet\quad \cT_{0}(\cC) = \cC\\
%&\bullet\quad \cT_{n+1}(\cC) = \cT_n(\cC)
% &\cup \{[tu] : t,u\in\cT_n(\cC); t,u\in\typpo\}\\
%&&\cup \{(tu) : t,u\in\cT_n(\cC); t,u\in\typli\}
%\end{align*}
\begin{tabbing}
\qquad\=$\bullet$\quad\=$\cT_{n+1}(\cC) = \cT_n(\cC)\ $\=\kill
\> $\bullet$ \> $\cT_0(\cC) = \cC$\\
\> $\bullet$ \> $\cT_{n+1}(\cC) = \cT_n(\cC)$ \>
         $\cup\, \{\,[tu] : t,u\in\cT_n(\cC); t,u\in\typpo\}$\\
\>\>\>     $\cup\, \{\,(tu) : t,u\in\cT_n(\cC); t,u\in\typli\}$
\end{tabbing}
\end{definition}

\begin{definition}[Depth] The \emph{depth} of a term
$t$\index{term!depth} is defined as the number $n$, at which $t$ is
added (or constructed) in the process given above.
\end{definition}

The expression ``depth'' describes how deep a term is nested.

To ensure consistency inside a set
of starting objects, they must obey one rule, namely that if a
compound term is in the set, than also the subterms are. That's the
reason for the next definition.

\begin{definition}[admissible set of terms] Let $\cM$  be a
subset of $\cT(\cC)$, $\cC$ a set of constants, then $\cM$ is called
admissible\index{termset!admissible} if it obeys the
following rules:
\begin{tabbing} \qquad\=$\bullet$\quad\=\kill
\> $\bullet$ \> $(\forall [XY]\in\cM)(X,Y\in\cM)$\\
\> $\bullet$ \> $(\forall (gh)\in\cM)(g,h\in\cM)$
\end{tabbing}
\end{definition}

The idea is to define a set of Points, Lines and certain combinations
of them (the intersection points and connection lines) and let the
sketch be a subset of all possible atomic formulas over these terms.

\begin{definition}[Universe of Formulas] Let $\cM$ be an
admissible termset and $\cP$ a set of predicates, then the
universe of formulas over $\cM$ with regard to $\cP$ is
defined as
$$\fu_\cP(\cM) = \{P(t_1,t_2) : P\in\cP; t_i \text{ of the right
                                                types}\}$$
\end{definition}

$\cP$ will only be $\{\cI,=\}$ or $\{\cI\}$. The set $\fu$ contains
all the possible positive statements which can be made over the
termset $\cM$.

If we bear in mind that we want to do something proof-theoretic with
the formalization of the sketch, we must ensure that nothing evil
happens when a simple procedure without any knowledge about geometry
is performed. And one of the evils that could happen is a

\begin{definition}[Critical Constellation] Let $P$ and $Q$ be terms in
$\typpo$ and $g$ and $h$ terms in $\typli$. Than we call the
appearance of the following four formulas a critical
constellation:
\[\begin{tabular}[c]{c|c}
\strut$P\cI g$ & $P\cI h$\\\hline
\strut$Q\cI g$ & $Q\cI h$\\
\end{tabular}\]
We will denote such critical constellations by $(P,Q;g,h)$.
\end{definition}

Such a constellation is called critical, because from these four
formulas it follows that either $P=Q$ or $g=h$ (or both), but we
cannot determine which one of these alternatives without supplementary
information (see fig.~\ref{skizze:critical}).

\begin{figure}[ht]
  \[\epsfbox{diagonal.4}\]
  \caption{The two solutions for a critical constellation}
  \label{skizze:critical}
\end{figure}

When constructing any sketch we start from some assumptions over a set
of constants and then construct new objects and deduce new relations.
From a proof-theoretic point of view these first assumptions are the
left side of the deduced sequent, i.e.\ the assumptions from which we
deduce the fact. In the proof given at the beginning of this section
the assumptions are that $R,S,T$ are not collinear and that $R\kappa,
S\kappa, T\kappa$ are collinear. Then we tried to deduce a
contradiction to show, that one of the assumptions is wrong, i.e.\ that
from $\lnot\koll(R,S,T)$ $\lnot\koll(R\kappa,S\kappa,T\kappa)$ can be
deduced.

We now come to the final definition of the sketch. We want a
sketch to be a set describing all the incidences in the
sketch\footnote{This one is on the paper!}. But we want also that this
subset is closed under trivial incidences, which means that if we talk
about a Line which is the connection of Points, then we want that the
trivial formulas express that these two Points lie on the
corresponding Line.

Further we don't want to have a critical constellation in a sketch.
That arises from the fact that we want that every geometric object is
described only by one logical object, i.e.\ one term. Since a critical
constellation implies the equality of two logical objects, which we
cannot determine automatically, we want to exclude such cases.

\begin{definition}[Sketch] Let $\cM$ be a admissible termset
over a set of constants $\cC$, $\{\ao,\bo,\co,\deo,
[\ao\bo],\dots,[\co\deo]\}\subset\cM$,\\
let $\cE$ be a subset of
 $\fu_{\{\cI\}}(\cM) \cup \overline{\fu_{\{\cI,=\}}(\cM)}$
with $\ao\not=\bo$, \ldots,
 $\co\not=\deo$, $\ao\notcI[\bo\co]$,\ldots, $\deo\notcI[\ao\bo]\in\cE$,\\
let $Q$ be a set of equalities and let the triple
$(\cM,\cE,Q)$ obey the following requirements:
\begin{align}
\qquad & (\forall X,Y\in\cM,\typpo)([XY]\in\cM\limp(X\cI[XY])\in\cE
                                \land (Y\cI[XY])\in\cE)\notag\\
 & (\forall g,h\in\cM,\typli)((gh)\in\cM\limp((gh)\cI g)\in\cE\land
                                    ((gh)\cI h)\in\cE)
                        \label{eq:s1}\tag{S.1}\\
 & (\lnot\exists x,y\in\cM)(P(x,y)\in\cE \land
 \lnot P(x,y)\in\cE)\notag\\
 & (\lnot\exists x\in\cM)((x\not=x)\in\cE)
                \label{eq:s2}\tag{S.2}\\
 & \text{there are no critical constellations in $\cE$}
                 \label{eq:s3}\tag{S.3}\\
 & (\forall x\in\cM)((x=x)\in Q) \label{eq:s4}\tag{S.4}
\end{align}
Then we call the triple $\cS=(\cM,\cE,Q)$ a sketch.
\end{definition}

We will call the violation of~\ref{eq:s2} also a direct
contradiction.

A small example should help to understand the concepts:

\begin{figure}[ht]
  \[\epsfbox{diagonal.5}\]
  \caption{A sample sketch}
  \label{skizze:sample-sketch}
\end{figure}

In the sketch depicted in fig.~\ref{skizze:sample-sketch} the
different sets are (where the incidences of the constants are lost!): 
\begin{tabbing}
        \qquad\= $\fu_{\{\cI,=\}}=\{$\=\kill
        \> $\cC = \{P,Q,R,X,g\}$\\
        \> $\cM = \{P,Q,R,X,g,[RQ]\}$\\
        \> $\fu_{\{\cI,=\}}=\{$\>$P\cI g, Q\cI g, R\cI g, X\cI g,$\\
        \>            \>$P\cI[RQ], Q\cI[RQ], R\cI[RQ], X\cI[RQ],$\\
        \>            \>$P=Q, P=R, P=X, Q=R, Q=X, R=X,$\\
        \>            \>$g=[RQ]\}$\\
        \>$\cE = \{ Q\cI[RQ], R\cI[RQ], X\cI g, P\notcI g, Q\notcI g,
                           R\notcI g, $\\ 
        \> \qquad\qquad $ P\notcI[QR], X\notcI[QR]\}$
\end{tabbing}

A few words to the habit of writing: If we are writing
expressions like $P\in\cS, (P\cI g)\in\cS, (P\notcI h)\in \cS\ldots$,
then $P\in\cM; (P\cI g), (P\notcI h)\in \cE$,
respectively is meant. Any other similar expression has to be
interpreted accordingly.

Why should the set $\cE$ only contain a subset of $\fu_{\{\cI\}}(\cM)$
and not of $\fu_{\{\cI,=\}}(\cM)$? The reason is, that in a sketch
every geometric object should have one and only one name and should
also be described by one logical object. The same idea lies behind the
introduction of the concept of the critical constellation.

Note that one sketch is only one stage in the process of a
construction, which starting from some initial assumptions forming a
sketch deduces more and more facts and so constructs more and more
complex sketches.

The set $Q$ in the definition of the sketch initially was absent, but
investigations in the equality of proofs and constructions showed,
that this set is important for the proof, although it is not used in
the sketch. This depends on the usage of the equality: in the sketch it is
a strict one, i.e., there is only one name for an object allowed,
while in a proof you can use one time one name, the other another
name. In the sketch, as we will later see, there is not a local
substitution of a term, but a global, therefore only one name is
``actual'' at a time for an object. But if we want to translate a
proof into a construction, which is one of the aims of this work, we
need informations on all the name-changes that are possible.

%
% subsection
% actions on sketches
%
\subsection{Actions on Sketches\label{:sketch:actions}}
Till now a sketch is only a static concept, nothing could happen, you
cannot ``construct''. So we want to give some actions on a sketch,
which construct a new sketch with more information. This new sketch
must not obey the requirements~\ref{eq:s1}--\ref{eq:s3}, but it will
be a \ldots

\begin{definition}[Semisketch] A \emph{semisketch} is a sketch that
need not obey to~\ref{eq:s2} and~\ref{eq:s3}.
\end{definition}

These actions should correspond to similar actions in the real
constructing. After 
these actions are defined we can explain what we mean by a
construction in this calculus for construction.

The following list defines the allowed actions and what controls have
to be executed.
The following list describes the changes that have to be done on the
quadruple of a sketch when we want to carry out the corresponding
action. 

In the following listing we will use the function $\text{closure}(Q)$
on a set of equalities $Q$. This function deduces all equalities which
are consequences of the set $Q$. This is a relatively easy
computation. If we have $Q=\{x=x,y=y,z=z,x=y,y=z\}$, then the
procedure returns $Q\cup\{x=z\}$. This function is used to update the
set $Q$ of a sketch after a substitution.

\begin{description}
\item\textbf{Joining of two Points $X,Y$; Symbol: $[XY]$}
        \begin{compactitem}
        \item $\cM'=\cM+[XY]$
        \item $\cE'=\cE+\{X\cI[XY], Y\cI[XY]\}$
        \item $Q'=Q+([XY]=[XY])$
        \end{compactitem}
        The requirement \eqref{eq:s1} and \eqref{eq:s4} is fulfilled
        since the necessary formulas are added to $\cE$ and
        $Q$. This action can produce a semisketch from a sketch.
\item\textbf{Intersection of two Lines $g,h$; Symbol $(gh)$}\\
        Dual to the joining of two points.
\item\textbf{Assuming a  new Object $C$ in general position, Symbol $\{C\}$}
        \begin{compactitem}
        \item $\cM'=\cM+C$
        \item $\cE'=\cE$
        \item $Q'=Q+(C=C)$
        \end{compactitem}
        That $\cS'$ is a sketch is trivial, since $C$ is a completely
        new constant. $C$ must be a constant of type \ttyppo\ or
        \ttypli.
\item\textbf{Giving the Line $[XY]$ a name $g:=[XY]$; Symbol $g:=[XY]$}
        \begin{compactitem}
        \item $\cM'=\cM[[XY]/g]$
        \item $\cE'=\cE[[XY]/g]$
        \item $Q'=Q[[XY]/g]$
        \end{compactitem}
        $\cS'$ is a sketch since this operation is only a
        name-change. 
\item\textbf{Giving the Point $(gh)$ a name $P:=(gh)$; Symbol
        $P:=(gh)$}\\
        Dual to giving an intersection-point a name.
\item\textbf{Identifying two Points $u$ and $t$; Symbol $u=t$}
        \begin{compactitem}
        \item $\cM'=\cM\setminus\{u\}$
        \item $\cE'=\cE[u/t]$
        \item $Q'=\text{closure}(Q\cup\{u=t\})$
        \end{compactitem}
        Note that the set $Q'$ can contain terms $t$ not in $\cM'$.
        This action can produce a semisketch from a sketch.
%        For an example c.f.~fig.~\ref{skizze:gleichsetzen}.
\item\textbf{Identifying two Lines $l$ and $m$; Symbol $l=m$}\\
        Dual to identifying two Points.
\item\textbf{Using a ``Lemma'': Adding $t\cI u$; Symbol $t\cI u$}
        \begin{compactitem}
        \item $\cM'=\cM$
        \item $\cE'=\cE+(t\cI u)$
        \item $Q'=Q$
        \end{compactitem}
        This action can produce a semisketch from a sketch. 
\item\textbf{Adding a negative literal $t\notcI u$; Symbol $t\notcI u$}
        \begin{compactitem}
        \item $\cM'=\cM$
        \item $\cE'=\cE+(t\notcI u)$
        \item $Q'=Q$
        \end{compactitem}
\item\textbf{Adding a negative literal $t\not=u$; Symbol $t\not=u$}
        \begin{compactitem}
        \item $\cM'=\cM$
        \item $\cE'=\cE+(t\not=u)$
        \item $Q'=Q$
        \end{compactitem}
\end{description}

\begin{figure}[ht]
  \[\epsfbox{diagonal.6}\]
  \caption{Identifying two objects $t$ and $u$}
  \label{skizze:gleichsetzen}
\end{figure}

To deduce a fact with sketches we connect the concept of the sketch
and the concept of the actions into a new concept called
construction. This construction will deduce the facts.


\begin{definition}[Construction] A \emph{construction} is a rooted and
directed 
tree with a semisketch attached to each node and an action
attached to each vertex and satisfying the following conditions:
If a vertex with action $A$ leads from node $N_1$ to node $N_2$, then $N_2$
is obtained from $N_1$ by carrying out the action on $N_1$. If from a
node $N$ there is a vertex labeled \ldots
\begin{compactitem}
  \item with $[XY]$ or $(gh)$, then $X\not=Y$ or $g\not=h$ is contained
  in $\cE^N$.
  \item with $\{C\}$, $g:=[XY]$, $P:=(gh)$, then there is no other
  vertex from $N$.
  \item with $P\cI g$, $P\notcI g$, $X=Y$, $X\not=Y$, then there is
  exactly one other vertex labeled with the negation of a
  formula. This is called a case-distinction. 
\end{compactitem}
\noindent Furthermore if $\cE$ attached to a node \ldots 
\begin{compactitem}
\item yields a direct contradiction, then it has no successor,
\item is a semisketch but not a sketch, i.e.\ that there are critical
constellations, let $(P,Q;g,h)$ be one of them, then there are exactly
two successors, one labeled with the action $P=Q$ and one labeled
with the action $g=h$.
\end{compactitem}
\end{definition}


What is deduced by a construction: A formula is true when it is true
in all the models of the given calculus. The distinct models in a
construction are achieved by case-distinctions. So if a formula 
is deduced by a construction, it must be present in all the leafs of
the tree. Since some leaves end with contradictions and from the
logical principle ``ex falso quodlibet'' we only require that a
formula, which should be deduced, has to be present in all leaves
which are not contradictious. 

We also have to pay attention to the way a construction handles
identities. Since in a construction an identity is carried out in the
way that all occurrences of one term are substituted for the other, we
not only prove an atomic formula, but also all the formulas which are
variants with respect to the corresponding set $Q$. This notion will
now be defined.

\begin{definition} Two atomic formulas $P(t_1,u_1)$ and $P(t_2,u_2)$
are said to be \emph{equivalent with respect to $Q$}, where $Q$ is a
set of equalities, in symbols $P(t_1,u_1)\equiv_{Q_N} P(t_2,u_2)$, when
$(t_1=t_2), (u_1=u_2) \in Q_N$ (or the symmetric one).
\end{definition}

Now we can define the notion of what a construction deduces:


\begin{definition} A construction deduces a set of atomic formulas
$\Delta$ iff for all $A\in\Delta$ there is a not contradictious leaf,
where either $A\in Q_N$ or
$(\exists B\in\cE(N))A\equiv_{Q(N)} B$.
\end{definition}

The meaning of this definition is that if a construction
deduces~$\Delta$ then the disjunction of all formulas in~$\Delta$ is
proved by this construction.


%??DRINLASSEN??
%For an example see section~\ref{:example}.\\
%For an example see \cite{Prei96MT}.
%??DRINLASSEN??


%In the approach to formalize the sketch in \pg, one of the early
%approaches was influenced by the idea of a closed world assumption: In
%a sketch everything what is drawn unequal should be unequal and only
%those incidences drawn are valid, all the others are wrong. It is
%similar to someone at the airport asking for a connection from A to B,
%not finding such a connection, deducing that there is none. But the
%negation of all not explicitly stated atomic formulas led to serious
%problems, since \pg is not complete, i.e., there are theorems which
%are true in one model and false in another. Take for an example the
%formula $D_1\cI[D_2D_3]$, where the $D_i$ are the diagonal-points of
%the points $\ao,\dots,\deo$. As shown in
%sec.~\ref{:projective:examples} there is \pg, e.g.\ the minimal \pg,
%where this incidence is true, and other ones, e.g. $\piae$, where this
%incidence is wrong. So if we negate this axiom we would leave the
%generality of models and restrict ourselves to special models. But
%this is possible, if we assume this incidence in the initial sketch.
%For a detailed discussion on different closed world assumptions
%compare to \cite{cadoli-cwa} and \cite{minker-cwa}.

\endinput

%%% Local Variables: 
%%% mode: latex
%%% TeX-master: "paper.tex"
%%% End: 

