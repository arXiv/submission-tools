%
% abstract.tex
%

\begin{abstract}
% This paper presents a prooftheoretic analysis of axiomatic plane
% projective geometry together with an analysis of the strength of
% sketches as proof. We will introduce a calculus extending Gentzen's
% \lk and give some theorems for this calculus, especially a
% cut-elimination theorem is presented.

% Furthermore we formalize the concept of sketches as they are used in
% projective geometry to support the understanding of (formal)
% proofs. Going on we show that sketches by themselves can be viewed as
% proofs and give a proof of the proof-theoretic equivalence of sketches
% and proofs in \lk.

% We will exhibit the connection between sketches and cut-free proofs
% and starting from this we will show that proofs are non-elementary
% faster than sketches.

This paper introduces sketches in projective geometry as valid proving
tools. Although sketches are used to support the understanding of
proofs in projective geometry, they are not considered as proofs by
themselves. We will show that sketches can indeed be viewed as
proofs. For this purpose we will analyze some properties of Herbrand
disjunctions (sec.~\ref{sec:herbrand}), then we will present our
formalization of sketches (sec.~\ref{sec:sketch}). Going on we will
show that sketches and Herbrand disjunctions are exchangeable, 
i.e.\ that sketches are proofs
(sec.~\ref{sec:herbrandsketch}). Finally we will present results
comparing the length of sketches and proofs using standard techniques
(sec.~\ref{sec:orevkov}). As an interesting consequence of these
analyses we will see that sketches are not constructive in the logical
sense. 


\end{abstract}

%%% Local Variables: 
%%% mode: latex
%%% TeX-master: "paper"
%%% End: 
