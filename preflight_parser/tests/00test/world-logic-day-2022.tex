\documentclass[aspectratio=169]%[t]
  {beamer}
\usepackage{etex}
\usefonttheme{serif}
\usefonttheme{professionalfonts}
\usepackage[T1]{fontenc}
\usepackage[utf8x]{inputenc}
\usepackage{hhline,bm,xspace}
\usepackage{bussproofs,bpextra}
%
\renewcommand{\rmdefault}{pasj}
\usepackage[small,euler-digits]{eulervm}
\usepackage{bm}
%
%\let\digamma\relax
%\usepackage[romanfamily=bright-osf,seriftt=true,stdmathdigits=true,lucidascale]{lucimatx}
%
\usepackage{amsmath}
\usepackage{natbib}
\usepackage{mathrsfs,proof,prooftree}
\usepackage{tikz,pgfheaps}
\usetikzlibrary{shapes,arrows,positioning,decorations,calc,snakes}
\usepackage{ctable}
\newif\ifoldstyle
\oldstyletrue
\newcommand{\nscosf}[1]{%
  \ifoldstyle\textsc{\MakeLowercase{#1}}\else#1\fi}
\usetheme{default}
\setbeamercolor*{black on white}{bg=white,fg=black}
\setbeamertemplate{navigation symbols}{}
\setbeamertemplate{theorems}[unnumbered]
\setbeamersize{text margin left=3em,text margin right=3em}
\setbeamertemplate{frametitle}
{
  %\begin{centering}
    %\hbox{~}
    \vspace{0.5em}
    %\par
    \textsc{\insertframetitle}
    \par
    %\medskip
  %\end{centering}
}
\setbeamersize{text margin left=14pt,text margin right=14pt}

\newtheorem{conjecture}[theorem]{Conjecture}

\bibpunct{(}{)}{; }{a}{,}{,}
\newcommand{\cutin}[1]{%
\frame[c]{\begin{center}{\Large\bf\color{blue}#1}\end{center}}}
\def\mis{\\[\medskipamount]}
\def\bis{\\[\bigskipamount]}

\newcommand{\kO}{\ensuremath 0_K}
\newcommand{\kI}{\ensuremath 1_K}
\newcommand{\Up}{\mathrm{Up}}
\newcommand\HS{{\,\,|\,\,}}
\newcommand\ProofLabel[1]{\LeftLabel{\emph{#1}}}
\newcommand\SplitR[4]{{\ensuremath{}#1{:}{}^{#2}\,{\mathrm{Spt}_{#3,#4}}}}
\newcommand\ComR[3]{{\ensuremath\mathrm{com}^{#1}_{#2,#3}}}
\newcommand\ConR[1]{{\ensuremath{}#1{:}\,\mathrm{Ctr}}}
\newcommand\Ga{\Gamma}
\newcommand\De{\Delta}
\newcommand\si{\sigma}
\newcommand{\HND}{Hyper Natural Deduction\xspace}
\newcommand{\HLK}{Hypersequent Calculus\xspace}
\newcommand{\hnd}{GLHN\xspace}
\newcommand{\hlk}{HLK\xspace}

\newcommand\iSAT{\mathnormal{1}\hbox{\upshape -SAT}}
\newcommand\iSATd{\iSAT^\bd}
\newcommand\oSAT{\mathnormal{0^*}\hbox{\upshape -SAT}}
\newcommand\oSATd{\oSAT^\bd}
\newcommand\VAL{\hbox{\upshape VAL}}
\newcommand\VALd{\VAL^\bd}
\newcommand{\al}{\alpha}
\newcommand{\be}{\beta}
\newcommand{\ga}{\gamma}
\newcommand{\om}[1]{\omega^{#1}}
\newcommand{\up}[1]{{#1}^{\uparrow}}
\def\ordinal#1{%
  \ordord#1::}
\def\ordord#1,#2,#3::{
  \omega^2 #1 + \omega^1 #2 + \omega^0 #3}

\def\zerord#1#2{%
  \draw [line width=0.5] (#1-1pt,#2) -- (#1+1pt,#2) ;
  %\draw (0,#1) node[anchor=east] {#2} ;
}
\def\zerohor#1#2{%
  \draw [line width=0.5] (#1,#2-1pt) -- (#1,#2+1pt) ;
  %\draw (0,#1) node[anchor=east] {#2} ;
}

\def\oneord#1#2#3{%
  %\draw (0,0) node {#1 -- #2 -- #3};
  \draw[-stealth, line width=0.5pt] (#1,#2) -- (#1,#3) ;
  \draw [line width=0.5] (#1-1pt,#2) -- (#1+1pt,#2) ;
}
\def\onehor#1#2#3{%
  %\draw (0,0) node {#1 -- #2 -- #3};
  \draw[-stealth, line width=0.5pt] (#1,#3) -- (#2,#3) ;
  \draw [line width=0.5] (#1,#3-1pt) -- (#1,#3+1pt) ;
}
\def\tword#1#2#3{%
  \draw [line width=0.5] (#1-4pt,#2) -- (#1+1pt,#2);
  \oneord{#1}{#2}{#3}
  \pgfmathparse{(#3-#2)/3}
  \pgfmathtruncatemacro\u\pgfmathresult
  \oneord{#1-3pt}{#2}{#2 + \u pt}
  \oneord{#1-3pt}{#2+\u pt}{#2+2*\u pt}
  \draw[dotted, line width=0.5] (#1-3pt,#2+2*\u pt) -- (#1-3pt,#3) ;
  %\oneord{#1-2pt}{#2+2*\u pt}{#2+3*\u pt}
}
\def\twohor#1#2#3{%
  \draw [line width=0.5] (#1,#3-4pt) -- (#1,#3+1pt); gut
  \onehor{#1}{#2}{#3} %gut
  \pgfmathparse{((#1)-(#2))/3}
  \pgfmathtruncatemacro\u\pgfmathresult
  \onehor{#1}{#1-\u pt}{#3-3pt}
  \onehor{#1-\u pt}{#1-2*\u pt}{#3-3pt}
  \draw[dotted, line width=0.5] (#1-2*\u pt,#3-3pt) -- (#2,#3-3pt) ;
  %\oneord{#1-2pt}{#2+2*\u pt}{#2+3*\u pt}
}

\def\axdelta{\mathnormal{\bd}}
\def\axdi{{\ensuremath{\axdelta1}}}
\def\axdii{{\ensuremath{\axdelta2}}}
\def\axdiii{{\ensuremath{\axdelta3}}}
\def\axdiv{{\ensuremath{\axdelta4}}}
\def\axdv{{\ensuremath{\axdelta5}}}
\def\axdr{{\ensuremath{\axdelta R}}}
\def\axdel{{\ensuremath{\hbox{\scshape ax}\axdelta}}}
\def\axdense{{\ensuremath{\hbox{\scshape den}}}}
\def\proves{\mathrel{\vdash}}
\def\nproves{\mathrel{\nvdash}}
\newcommand\lOr{\bigvee}
\newcommand\lequiv{\leftrightarrow}
\newcommand\bd{\bigtriangleup} % Baaz Delta
\newcommand{\cred}[1]{{\color{red}#1}}
\newcommand{\cF}{\mathcal{F}}
\newcommand{\cC}{\mathcal{C}}
\newcommand{\suchthat}{\mathrel{:}}
\newcommand{\I}{v}
\newcommand\qe[1]{\exists #1}
\newcommand\qa[1]{\forall #1}
\newcommand{\vup}{{\uparrow}}
\newcommand{\vdn}{{\downarrow}}
\newcommand{\dequiv}{\Leftrightarrow}
\newcommand{\limp}{\supset}
\newcommand{\Impl}{\limp}
\newcommand{\bbR}{\mathbb{R}}
\newcommand{\bbN}{\mathbb{N}}
\newcommand{\bbC}{\mathbb{C}}
\newcommand{\bbQ}{\mathbb{Q}}
\newcommand{\entails}{\mathrel{\Vdash}}
\newcommand{\gdl}[1]{{\bf G}_{#1}}
\newcommand{\oper}{\mathnormal{\circ}}
\newcommand{\lgc}{\mathnormal{\bf L}}
\def\lequiv{\leftrightarrow}
\def\Feq{{\cF\!/\mathnormal{\equiv}}}
\newcommand{\sis}{\\[\medskipamount]}
\newcommand{\hh}{\ |\ }
\newcommand{\seq}{\Rightarrow}
\newcommand\movemakebox[3]{%
  \bgroup%
  \setbox0=\hbox{\hskip #2\vbox to 0pt{\vskip #3 {#1}}}%
  \dp0=0pt %
  \ht0=0pt %
  \wd0=0pt %
  %\hbox to \textwidth{\hfill\hbox{\box0}}%
  \hbox{\box0}
  \egroup}
\newcommand\img[4]{%
  \bgroup%
  \setbox0=\hbox{\hskip #3\vbox to 0pt{\vskip #4 \includegraphics[height=#2]{#1}}}%
  \dp0=0pt %
  \ht0=0pt %
  \wd0=0pt %
  %\hbox to \textwidth{\hfill\hbox{\box0}}%
  \hbox{\box0}
  \egroup}


\pgfdeclareimage[interpolate=true,width=8cm]{luka}{lukasiewicz}
\pgfdeclareimage[interpolate=true,width=8cm]{goedel}{goedel}
\pgfdeclareimage[interpolate=true,width=8cm]{product}{product}
\pgfdeclareimage[interpolate=true,width=3cm]{lukasmall}{lukasiewicz-tnorm}
\pgfdeclareimage[interpolate=true,width=3cm]{goedelsmall}{goedel}
\pgfdeclareimage[interpolate=true,width=3cm]{productsmall}{product}


\parskip=8pt plus 3pt minus 3pt

%%%%%%%%%%%%%%%%%%%%%%%%%%%%%%%%%%%%%%%
%
% here begins the stuff
%
%%%%%%%%%%%%%%%%%%%%%%%%%%%%%%%%%%%%%%%
\title{\textsc{Gödel Logics -- a short survey}}
\author{Norbert Preining\thanks{\url{norbert@preining.info}}}

\date{Logic For the Friendship of Nations\\ 
  UNESCO World Logic Day 2022\\
  Virtual, January 2022}
\institute{Japan} %Fujitsu Research}

\begin{document}


\frame{\titlepage}


\begin{frame}
  \frametitle{Today's program\hspace{4cm}}

  \begin{itemize}
  \item Development of many-valued logics\\
    $t$-norm based logics
  \item Gödel logics (propositional\\
    quantified propositional, first order)
  \item Gödel logics and \ldots
  % \item Quantified propositional Gödel logics
  % \item First Order Gödel Logics%
    \movemakebox{\begin{tikz} \draw (0,0) node [rotate=90] {\tiny Gödel with a somehow famous physicist}; \end{tikz}}{195pt}{-80pt}
    \img{goedeleistein.png}{130pt}{100pt}{-85pt}

    \begin{itemize}
    \item Topology
    \item Order Theory
    \item Computability
    \end{itemize}
  \item Other topics
    \begin{itemize}
    \item Kripke frames and beyond the reals
    \item Monadic fragment
    \item Proof theory
    \end{itemize}
  \item History
  \item Conclusion
  \end{itemize}
\end{frame}

\begin{frame}
  \frametitle{Development of many-valued logics}
  
  \begin{block}{The most important stops}
    \begin{itemize}
    \item \textsc{Platon}, \textsc{Aristoteles} (De Interpretatione IX), 
      \textsc{Ockham}: 
      \emph{future possibilities}, problem of determination vs. fatalism.
      \uncover<1-1>{\img{platon.jpg}{110pt}{-130pt}{10pt}}
      \uncover<1-1>{\img{aristotle.jpg}{110pt}{-50pt}{10pt}}
      \uncover<1-1>{\img{William_of_Ockham.png}{110pt}{40pt}{10pt}}\pause
    \item \textsc{\L ukasiewicz} 1920: 3-valued logic of
      \emph{non-determinism}
      \uncover<2-2>{\img{Lukasiewicz.jpeg}{110pt}{-180pt}{10pt}}\pause
    \item \textsc{Post} 1920: Many-valued logic dealing with
      functional completion
    \pause
    \item \textsc{Gödel} 1932: Finite valued logics for approximation
      of intuitionistic logic
      \uncover<4-4>{\img{goedel1.jpeg}{110pt}{90pt}{0pt}}\pause
    \item \textsc{Bo\v cvar} 1938: Logic of \emph{Paradoxa}
    \item \textsc{Kleene} 1952: Logic of the \emph{unknown}
    \item \textsc{Zadeh} 1965: Fuzzy sets and fuzzy logics
    \end{itemize}
  \end{block}
\end{frame}

%\subsection{Generalization}

\begin{frame}
  \frametitle{How do we continue?}

  \begin{block}{Arbitrary finite-valued logics}
    For all finite-valued logics with truth-value functions there is
    an automatic algorithm for generating a sequent calculus, proving
    completeness etc (MultLog, MultSeq: \relax{Baaz, Fermüller,
    Salzer, Zach et al.} 1996ff). 
  \end{block}

  \pause
  \begin{block}{Infinite valued logics}
    Does it make sense to take truth values from arbitrary partial
    orderings? 
  \end{block}

  \pause
  $\Rightarrow$ No, because \alert{every} logics with substitution property would
    be a many-valued logic!
    
    \medskip
    Take all sentences as truth values, and all sentences of the logic
    as designated truth values.
\end{frame}

\section{$t$-norm based logics}
\subsection{Design decisions}
\begin{frame}
  \frametitle{Design decisions}

  \begin{block}{Basic requirements}
    \begin{itemize}
    \item Extension of classical logic
    \item $[0,1]$ as super-set of the truth value set
    \item functional relation between the truth value of a formula and
      the one of its sub-formulas.
    \end{itemize}
  \end{block}

  \pause
  \begin{block}{Additional `natural' properties of the conjunction}
    \begin{itemize}
    \item associative ($(A\land B)\land C \Leftrightarrow A\land(B\land
      C)$) 
    \item commutative ($A\land B \Leftrightarrow B\land A$)
    \item order preserving\\ 
      If $A$ is less true than $b$, then $A\land C$ is less (or equal)
      true than $B\land C$.
    \item continuous
    \end{itemize}
  \end{block}
\end{frame}


\subsection{$t$-norms}
\begin{frame}
  \frametitle{Definition of (continuous) $t$-norms}

  \begin{definition}
    A $t$-norm is an associative, commutative, and monotone mapping
    from $[0,1]^2\to [0,1]$ with $1$ as neutral element.
    \begin{itemize}
    \item $(x \star y) \star z = x \star (y \star z)$
    \item $x \star y = y \star x$
    \item $x\le y \limp x\star z \le y\star z$
    \item $1\star x = x$
    \item $\star$ is continuous
    \end{itemize}
  \end{definition}

  \pause
  \begin{block}{Algebraic view}
    $\left<[0,1], \star, 1, \le\right>$ is a commutative and ordered monoid.
  \end{block}
\end{frame}

% %\subsection{\L ukasiewicz $t$-norm}

% \begin{frame}
%   \frametitle{\L ukasiewicz $t$-norm}
 
%  \begin{center}
%    \begin{pgfpicture}{0cm}{0cm}{8cm}{6cm}
%      \pgfputat{\pgfxy(0,0)}%
%      {\pgfbox[left,base]{\pgfuseimage{luka}}}
%      \pgfputat{\pgfxy(-1,5.5)}{\pgfbox[left,center]{$x \star_{\L} y = \max\{0,x+y-1\}$}}
%    \end{pgfpicture}
%  \end{center}
% \end{frame}


% %\subsection{Gödel $t$-norm}

% \begin{frame}
%   \frametitle{Gödel $t$-norm}
 
%  \begin{center}
%    \begin{pgfpicture}{0cm}{0cm}{8cm}{6cm}
%      \pgfputat{\pgfxy(0,0)}%
%      {\pgfbox[left,base]{\pgfuseimage{goedel}}}
%      \pgfputat{\pgfxy(-1,5.5)}{\pgfbox[left,center]{$x \star_{G} y = \min\{x,y\}$}}
%    \end{pgfpicture}
%  \end{center}
% \end{frame}

% %\subsection{Product $t$-norm}

% \begin{frame}
%   \frametitle{Product $t$-norm}
 
%  \begin{center}
%    \begin{pgfpicture}{0cm}{0cm}{8cm}{6cm}
%      \pgfputat{\pgfxy(0,0)}%
%      {\pgfbox[left,base]{\pgfuseimage{product}}}
%      \pgfputat{\pgfxy(-1,5.5)}{\pgfbox[left,center]{$x \star_{\Pi} y = xy$}}
%    \end{pgfpicture}
%  \end{center}
% \end{frame}

%\subsection{From $t$-norm to the logic}

\begin{frame}
  \frametitle{From $t$-norm to the logic}

  \pause
  \begin{block}{The residuum of a $t$-norm}
    Every $t$-norm has a residuum
    \begin{eqnarray*} 
      x \star z \le y &\Leftrightarrow& z \le (x\Rightarrow y)\\
      x \Rightarrow y &:=& \max \{ z \suchthat x\star z \le y \}
    \end{eqnarray*}
  \end{block}

  \pause
  \begin{block}{Truth functions for operators}
    \begin{itemize}
    \item strong conjunction $\&$: defined via the $t$-norm
    \item implication $\limp$: defined via the residuum
    \item Negation: $¬A := A \limp \bot$
    \item (weak) disjunction: $A\lor B := (A\limp B)\limp B$
    \item (weak) conjunction: $A\land B := ¬(¬A\lor¬B)$
    \item strong disjunction: $A\lOr B := ¬(A\limp¬B)$
    \end{itemize}
  \end{block}
\end{frame}


%\subsection{Properties of $t$-norms}
\begin{frame}
  \frametitle{Basic $t$-norms}

  \begin{center}
    \begin{tabular}{p{3.7cm}p{3.7cm}p{3.7cm}}
      Gödel & \only<2->{\L ukasiewicz} &  \only<3->{Product}\\
      \begin{pgfpicture}{0cm}{0cm}{3cm}{3cm}
        \pgfputat{\pgfxy(0,0)}{\pgfbox[left,base]{\pgfuseimage{goedelsmall}}}
      \end{pgfpicture}
      &
      \only<2->{%
        \begin{pgfpicture}{0cm}{0cm}{3cm}{3cm}
          \pgfputat{\pgfxy(0,0)}{\pgfbox[left,base]{\pgfuseimage{lukasmall}}}
        \end{pgfpicture}
      }
      &
      \only<3->{%
        \begin{pgfpicture}{0cm}{0cm}{3cm}{3cm}
          \pgfputat{\pgfxy(0,0)}{\pgfbox[left,base]{\pgfuseimage{productsmall}}}
        \end{pgfpicture}
      }
      \\
      \raggedright non-trivial idempotent elements
      \hphantom{hheaffd af kada dj ka} &
      \only<2->{%
        \raggedright no non-trivial idempotent elements, but zero
	divisors &
      }
      \only<3->{%
        \raggedright  no non-trivial idempotent elements, no zero divisors
      }
    \end{tabular}
  \end{center}
\end{frame}

\begin{frame}
  \frametitle{Representation of  $t$-norm}
  \begin{theorem}[Mostert and Shields, 1957]
    Every $t$-norm is the ordinal sum of \L ukasiewicz $t$-norm and
    Product $t$-norm.
  \end{theorem}

  \begin{center}
    \begin{pgfpicture}{0cm}{0cm}{5cm}{5cm}
      \pgfrect[stroke]{\pgfxy(0,0)}{\pgfxy(5,5)}
      \pgfrect[stroke]{\pgfxy(0.5,0.5)}{\pgfxy(1,1)}
      \pgfrect[stroke]{\pgfxy(1.5,1.5)}{\pgfxy(0.5,0.5)}
      \pgfrect[stroke]{\pgfxy(3,3)}{\pgfxy(1,1)}
      \pgfrect[stroke]{\pgfxy(4.5,4.5)}{\pgfxy(0.5,0.5)}
      \pgfsetdash{{0.05cm}{0.2cm}}{0.1cm}
      \pgfxyline(0,0)(0.5,0.5)
      \pgfxyline(2,2)(3,3)
      \pgfxyline(4,4)(4.5,4.5)
      \pgfputat{\pgfxy(1,1)}{\pgfbox[center,center]{\L}}
      \pgfputat{\pgfxy(1.75,1.75)}{\pgfbox[center,center]{$\Pi$}}
      \pgfputat{\pgfxy(3.5,3.5)}{\pgfbox[center,center]{$\Pi$}}
      \pgfputat{\pgfxy(4.75,4.75)}{\pgfbox[center,center]{\L}}
      \pgfputat{\pgfxy(1.5,3.5)}{\pgfbox[center,center]{G}}
      \pgfputat{\pgfxy(3.5,1.5)}{\pgfbox[center,center]{G}}
    \end{pgfpicture}
  \end{center}
\end{frame}


%\section{Questions and results}
\subsection{Basic Logic}
\begin{frame}
  \frametitle{Questions and results}
  \begin{itemize}
    \item Basic logic: the logic of all $t$-norms (\relax{Hajek} 1998)
    \item Axiomatizability: propositional logic: easy, first-order:
      only Gödel logics are axiomatizable (\relax{Scarpellini} 1962,
      \relax{Horn} 1969, \relax{Takeuti, Titani} 1984, \relax{Takano}
      1987) 
    \item calculi for propositional logic: sequent calculus for Gödel
      logic (\relax{Avron} 1991, \emph{hyper sequent calculus}), $\Pi$- und
      \L-Logik (\relax{Gabbay}, \relax{Metcalfe}, \relax{Olivetti}
      2003). 
    \item calculi for first order logic: only for Gödel logic
      (\relax{Baaz}, \relax{Zach} 2000) 
    \item Game interpretation: \L ukasiewicz Logic: interpretation via
      Ulam's games (\relax{Mundici}, 1991-93),
      Gödel, Product, \L ukasiewicz: interpretation via Lorenzen style
      games (\relax{Giles} 1970; \relax{Fermüller},
      \relax{Metcalfe}, \relax{Ciabattoni} 2003-04)
    \item other questions: automatic theorem proving, size of
      families, \dots
  \end{itemize}
\end{frame}


\cutin{Gödel Logics}

\def\aaaa#1#2#3#4#5#6#7{%
  \only<#6->{\draw (#1, #3) -- (#1, #2);
  \draw (#1, #3) node[anchor=south] {\alt<#6>{\alert{#4}}{#4}};
  \draw (#1, #2) node[anchor=north] {\alt<#6>{\alert{#5}}{#5}};}
  \only<#6>{\draw (0,-3) node[anchor=south west] {#7};}
}
\begin{frame}
  \frametitle{History}
  \begin{center}
  \begin{tikzpicture}
    \shade[left color=gray,right color=gray!30] (0,0) rectangle +(10,0.5);
    \draw (0,1.7) node[anchor=south west] {Timeline};
    \aaaa{0.42}{-0.4}{0.5}{1933}{Gödel\strut}1{finitely valued logics}
    \aaaa{4.08}{-0.4}{0.5}{1959}{Dummett\strut}2{infinitely valued propositional Gödel logics}
    \aaaa{5.57}{-0.4}{0.5}{1969}{Horn\strut}3{linearly ordered Heyting algebras}
    \aaaa{7.71}{-0.4}{0.5}{1984}{Takeuti-Titani\strut}4{intuitionistic fuzzy logic}
    \aaaa{8.71}{-0.9}{1}{1991}{Avron\strut}5{hypersequent calculus}
    \aaaa{9.4}{-1.4}{0.5}{1998}{Hájek\strut}6{$t$-norm based logics}
    \aaaa{8.57}{-1.9}{1.5}{since 90ies}{Viennese group\strut}7{proof theory,
      \#, Kripke, qp, fragments, \ldots}
  \end{tikzpicture}
  \end{center}
\end{frame}

\cutin{Propositional Logics}

\begin{frame}
  \frametitle{Propositional logic}

  Usual propositional language, $\lnot A$ is
  defined as $A\limp\bot$.

  \begin{block}{Evaluations} 
  Fix a truth value set 
  $\{0,1\}\subseteq V \subseteq [0,1]$\\
  $v$ maps propositional variables to elements of~$V$
  \begin{align*}
    \I(A\land B) &= \min\{\I(A),\I(B)\}\\
    \I(A\lor B)  &= \max\{\I(A),\I(B)\}\\
    \I(A\limp B) &= \begin{cases}\I(B) & \text{if $\I(A) > \I(B)$} \\
      1     & \text{if $\I(A) \le \I(B)$.}\end{cases}
  \end{align*}
  \end{block}
\end{frame}

\begin{frame}
  \frametitle{Negation}

  This yields the following definition of the semantics of~$\lnot$:
  \[\I(\lnot A) = \begin{cases} 0 & \text{if $\I(A)>0$}\\
    1 & \text{otherwise}\end{cases}\]
\end{frame}

\begin{frame}
  \frametitle{Takeuti's observation}
    Gödel implication
    \[
    \I(A\limp B) = \begin{cases}\I(B) & \text{if $\I(A) > \I(B)$} \\
      1     & \text{if $\I(A) \le \I(B)$.}\end{cases}
    \]
    is the only one satisfying: 
    \begin{itemize}
    \item $\I(A)\le\I(B) \dequiv \I(A\limp B)=1$

      \medskip
    \item $\Pi \cup \{ A \} \entails B \dequiv \Pi \entails A\limp B$

      \medskip
    \item $\Pi \entails B \Rightarrow 
      \min\{\I(A) \suchthat A\in\Pi \} \le \I(B)$\\ (and if 
      $\Pi=\emptyset\Rightarrow 1\le\I(B)$)
    \end{itemize}
\end{frame}

\begin{frame}
  \frametitle{Definition of the logic}

  \[ \gdl{V} = \{ A \suchthat \forall v \mbox{ into $V$}: v(A) = 1\} \]

  \pause
  \begin{block}{Examples}
    \begin{tabbing}
      $V_1 = \{0, 1/2, 1\}, V_2 = \{0, 1/3, 1\} \quad$ \=\kill
      $V = \{0,1\}$ \> $\to \gdl V = CPL$\\[4pt]
      $V_1 = \{0, 1/2, 1\}, V_2 = \{0, 1/3, 1\}$ \> 
        $\to \gdl{V_1} = \gdl{V_2}$\\[4pt]
      $V_\vup = \{1 - 1/n \suchthat n\ge1\} \cup \{1\}$ \>
        $\to \gdl{V_\vup} = \gdl\vup$\\[4pt]
      $V_\vdn = \{1/n \suchthat n\ge1\} \cup \{0\}$ \>
        $\to \gdl{V_\vdn} = \gdl\vdn$
    \end{tabbing}
  \end{block}
\end{frame}

\begin{frame}
  \frametitle{Propositional completeness}
  \begin{itemize}
  \item Lindenbaum algebra of the formulas
  \item show that the algebra $\Feq$ is a subalgebra of
    \[X = \prod_{i=1}^{n!} \cC(\bot,\pi_i(p_1,\ldots,p_n),\top)\]
    ($\cC(\ldots)$ being the chain consisting of the listed elements, and
    the $\pi_i$ all the permutations) by defining
    \[\phi(|\alpha|) = (|\alpha|_{\cC_1},\ldots,|\alpha|_{\cC_{n!}})\]
  \end{itemize}
\end{frame}

% \begin{frame}
%   \frametitle{Example L-algebra and chains}
%   \begin{tikzpicture}[scale=1]
%     \draw (-3,0) -- (0,3) -- (3,0) -- (0,-3) -- (-3,0);
%     \draw (-2,-1) -- (1,2);
%     \draw (-1,-2) -- (2,1);
%     \draw (1,-2) -- (-2,1);
%     \draw (2,-1) -- (-1,2);
%     \draw (0,-3) node[anchor=north] {$\bot,\bot$};
%     \draw (0,-3) node[anchor=south] {$\bot$};

%     \draw (-1,-2) node[anchor=north] {$p,\bot$};
%     \draw ( 1,-2) node[anchor=north] {$\bot,q$};

%     \draw (-2,-1) node[anchor=north] {$q,\bot$};
%     \draw ( 2,-1) node[anchor=north] {$\bot,p$};

%     \draw (0,-1) node[anchor=north] {$p,q$};
%     \draw (0,-1) node[anchor=south] {$p\land q$};
%     \draw (-3,0) node[anchor=north] {$\top,\bot$};
%     \draw ( 3,0) node[anchor=north] {$\bot,\top$};
    
%     \draw (-1,0) node[anchor=north] {$q,q$};
%     \draw (-1,0) node[anchor=south] {$q$};

%     \draw ( 1,0) node[anchor=north] {$p,p$};
%     \draw ( 1,0) node[anchor=south] {$p$};
    
%     \draw (-2,1) node[anchor=north] {$\top,q$};
%     \draw (-2,1) node[anchor=south east] {$p\limp q$};

%     \draw ( 2,1) node[anchor=north] {$p,\top$};
%     \draw ( 2,1) node[anchor=south west] {$q\limp p$};

%     \draw (-1,2) node[anchor=north] {$\top,p$};
%     \draw (-1,2) node[anchor=south east] {$(p\limp q)\lor q$};

%     \draw ( 1,2) node[anchor=north] {$q,\top$};
%     \draw ( 1,2) node[anchor=south west] {$(q\limp p)\lor p$};
%     \draw ( 0,3) node[anchor=north] {$\top,\top$};
%     \draw ( 0,3) node[anchor=south] {$\top$};

%     \draw ( 2,-3) node[anchor=west] {\vbox{$L$-algebra of\\
%          $\cC(\bot,p,q,\top) \times \cC(\bot,q,p,\top)$}};

%     \draw ( 4,0) node[anchor=west] {\vbox{%
%       \tiny 
%             upper label\\representatives%
% 	    \\[\medskipamount]%
%             lower label\\elements of direct product%
% 	    }};
%   \end{tikzpicture}
% \end{frame}

\begin{frame}
  \frametitle{Consequences}
  \begin{itemize}
  \item countably many propositional Gödel logics
   \[ 
     \gdl{2} \supset \gdl{3} \supset \ldots \supset \gdl n \supset \ldots
     \supset \gdl\vup = \gdl\vdn = \gdl V = \gdl\infty = \bigcap_{n\ge 2} \gdl n
   \]
   (where $V$ is any infinite truth value set)\sis
  \item if $f: V_1 \mapsto V_2$ with $f(0) = 0$ and $f(1) = 1$, order-preserving
    ($x<y \Rightarrow f(x) < f(y)$), then
    \[ \gdl{V_1} \supseteq \gdl{V_2} \]\sis
  \item check on satisfiability and validity
  \end{itemize}
\end{frame}

\cutin{Quantified Propositional Logics}

\begin{frame}
  \frametitle{\onslide<2->{\cred{Quantified} }Propositional Logic}

  Fix a truth value set 
  $\{0,1\}\subseteq V \subseteq [0,1]$%
  \onslide<2->{, \cred{$V$ closed}}\\
  $v$ maps propositional variables to elements of~$V$
  \begin{align*}
    \I(A\land B) &= \min\{\I(A),\I(B)\}\\
    \I(A\lor B)  &= \max\{\I(A),\I(B)\}\\
    \I(A\limp B) &= \begin{cases}\I(B) & \text{if $\I(A) > \I(B)$} \\
      1     & \text{if $\I(A) \le \I(B)$}\end{cases}\\
    \onslide<2->{
    \cred{\I(\qa p A(p))} &\cred{= \inf \{ \I(A(p)) \suchthat p\in P \}}\\
    \cred{\I(\qe p A(p))} &\cred{= \sup \{ \I(A(p)) \suchthat p\in P \}}
    }
  \end{align*}
\end{frame}


\begin{frame}
  \frametitle{Properties}

  \begin{center}
    \begin{tabular}[h]{lcccc}
      \toprule%
       & $V_\downarrow$ & $V_\uparrow$ & $V_\infty$\\
       \midrule
       Decidability & \textsc{s1s} & \textsc{s1s} & \textsc{qe}\\
       Axiomatisation & \textsc{hs+$\oper$} & \textsc{hs+$\oper$}
       & \textsc{hs, gs}\\
       QE & {with $\oper$} & {with $\oper$} & {yes}\\
       \bottomrule
    \end{tabular}
  \end{center}
  (Baaz, Veith, Zach, P. 2000--)
  \pause
  \begin{itemize}
  \item uncountably many different quantified propositional logics
    (coding the topological structure)\pause
  \item $\gdl{\uparrow}^{qp} = \bigcap_{n\in \bbN} \gdl{n}^{qp}$
  \item $\bigcap_{V\subseteq[0,1]}\gdl{V}^{qp}$ is \cred{not} a
    quantified propositional Gödel logic (in contrast
    to propositional and first-order Gödel logics)
  \end{itemize}
\end{frame}


\cutin{First Order Logics}


\begin{frame}
  \frametitle{First Order Gödel Logics}

  Fix a truth value set 
  $\{0,1\}\subseteq V \subseteq [0,1]$, $V$ closed

  Interpretation $\I$ consists of
  \begin{itemize}
  \item a nonempty set $U$, the universe of $\I$

    \medskip
  \item for each $k$-ary predicate symbol~$P$ a function $P^\I: U^k\to V$

    \medskip
  \item for each $k$-ary function symbol~$f$, a function $f^\I: U^k\to U$

    \medskip
  \item for each variable~$x$ an object $x^\I\in U$
  \end{itemize}
\end{frame}

\begin{frame}
  \frametitle{Semantic cont.}
  Extend the valuation to all formulas
  \begin{align*}
    \I(A\land B) &= \min\{\I(A),\I(B)\}\\[\medskipamount]
    \I(A\lor B)  &= \max\{\I(A),\I(B)\}\\[\medskipamount]
    \I(A\limp B) &= \begin{cases}\I(B) & \text{if $\I(A) > \I(B)$} \\[\medskipamount]
      1     & \text{if $\I(A) \le \I(B)$}\end{cases}\\[\medskipamount]
    \I(\qa x A(x)) &= \inf \{ \I(A(u)) \suchthat u\in U \}\\[\medskipamount]
    \I(\qe x A(x)) &= \sup \{ \I(A(u)) \suchthat u\in U \}
  \end{align*}
  
\end{frame}

\begin{frame}
  \frametitle{Horn-Takeuti-Titani-Takano -- axiomatizability}

  Axiomatizability of $\gdl{[0,1]}$:
  \begin{eqnarray*}
    \text{LIN:} & A\limp B \lor B\limp A\\
    \text{QS:}  & \qa x(A(x) \lor B) \limp (\qa xA(x) \lor B)\\[10pt]\pause
    \text{\textsf{GL}:} & \text{IL} +  \text{LIN} + \text{QS}
  \end{eqnarray*}

  \pause
  \begin{itemize}
  \item Horn (1969)\\
    \emph{logic with truth values in a linearly ordered Heyting
      algebra}

    \medskip
  \item Takeuti-Titani (1984), Takano (1987)\\
    \emph{intuitionistic fuzzy logic}
  \end{itemize}
\end{frame}

\begin{frame}
  \frametitle{Takano's proof}
  
  \begin{itemize}
  \item set of formulas $\cF$, equivalence relation $\equiv$ by
    provable equivalence
    
    \medskip
  \item show that $\cF/\mathnormal{\equiv}$ is a (linear) Gödel
    algebra

    \medskip
  \item embed $\cF/\mathnormal{\equiv}$ into $[0,1]$

    \medskip
  \item show that embedding preserves infima and suprema
    (order-theoretic infima versus topological infima)
  \end{itemize}
\end{frame}




% \newcommand{\nnn}[2]{
%   \draw[line width=1pt] (#1,-0.2) -- (#1,0.2);
%   \draw (#1,0.3) node[anchor=south] {#2};
% }
% \begin{frame}
%   \frametitle{Embedding procedure}
%   \[ \cF = \{ F_1 = [\bot], F_2 = [\top], F_3, \ldots \} \]
  

%   \begin{tikzpicture}
%     \draw[line width=1pt] (0,0) -- (6,0);
%     \nnn{0}{$F_1$}
%     \nnn{6}{$F_2$}
%     \draw (0,-0.3) node[anchor=north] {$0$};
%     \draw (6,-0.3) node[anchor=north] {$1$};

%     \onslide<2->{
%       \nnn{3}{$F_3$}
%       \draw (8,1) node {$F_1 < F_3 < F_2$};
%     }
%     \onslide<3->{
%       \nnn{4.5}{$F_4$}
%       \draw (8,0.5) node {$F_3 < F_4 < F_2$};
%     }
%     \onslide<4->{
%       \nnn{3.75}{$F_5$}
%       \draw (8,0) node {$F_3 < F_5 < F_4$};
%     }
%     \onslide<5->{
%       \nnn{1.5}{$F_6$}
%       \draw (8,-0.5) node {$F_1 < F_6 < F_3$};
%     }
%     \onslide<6->{
%       \nnn{5.25}{$F_7$}
%       \draw (8,-1) node {$F_4 < F_7 < F_2$};
%     }
%   \end{tikzpicture}
% \end{frame}

% \begin{frame}
%   \frametitle{Conditions on truth value set and embedding}
%   \begin{itemize}
%   \item contains a countable dense-in-itself subset that includes $0$
%     and $1$
    
%     \medskip
%   \item each element of this subset (but $0$ and $1$) is infimum and
%     supremum of elements in the subset
    
%     \medskip
%   \item the embedding preserves infima and suprema (order theoretic to
%     topologic) 
%   \end{itemize}
% \end{frame}

\begin{frame}
  \frametitle{Connections}

  Gödel Logics and \ldots
  \begin{itemize}
  \item topology\bis
  \item order theory\bis
  \item computation
  \end{itemize}
\end{frame}

\cutin{Gödel Logics and Topology}

\begin{frame}
  \frametitle{Possible truth value sets}

  \begin{block}{Perfect set}
    A set $P\subseteq \bbR$ is perfect if it is closed and all its
    points are limit points in $P$.
  \end{block}

  \begin{block}{Cantor-Bendixon}
    Any closed $V\subseteq \bbR$ can be uniquely written as $V = P\cup
    C$, with $P$ a perfect subset of $V$ and $C$ countable and open.
  \end{block}
% \end{frame}

% \begin{frame}
  \pause
  \begin{block}{Examples for perfect sets}
  \begin{itemize}
  \item $[0,1]$, any closed interval, any finite union of closed
    intervals

    \medskip
  \item Cantor Middle Third set $\bbC$: all numbers of $[0,1]$ that do
    not have a $1$ in the triadic notation (cut out all open middle
    intervals recursively)\\
    (perfect but nowhere dense)
  \end{itemize}
\end{block}
\end{frame}

% \pause
%   \begin{block}{Consequence}
%     The first order Gödel logics of $[0,1]$ and of $\bbC$ coincide.
%   \end{block}
% \end{frame}

% \begin{frame}
%   \frametitle{Full characterization of Axiomatizability}

%   \begin{block}{Recursively axiomatizable}
%     \begin{itemize}
%     \item finitely valued
%     \item $V$ contains a perfect subset~$P$ and $0\in P$ (standard)
%     \item $V$ contains a perfect subset~$P$ and $0$ is isolated
%     \end{itemize}
%   \end{block}

%   \pause
%   \begin{block}{Not recursively enumerable}
%     \begin{itemize}
%     \item countably infinite truth value set
%     \item every neighbourhood of~$0$ is countably infinite
%     \end{itemize}
%   \end{block}

%   (P. -- PhD; Baaz, P., Zach 2007)
% \end{frame}

% \cutin{Addition of $\bd$}

\begin{frame}
  \frametitle{The $\bd$ operator}
  \[ \mathop{\bd}(x) = \begin{cases} 1 & \text{if $x=1$}\\ 0 & \text{otherwise} \end{cases} \]

  \begin{itemize}
  \item introduced and axiomatised by Takeuti and Titani in their discussion
    of intuitionistic fuzzy logic\mis
  \item Baaz introduced and axiomatised in the context of Gödel logics\mis
  \item parallels the `recognizability' of $0$, i.e., makes $1$ recognizable.\mis
  \item axiomatization of Gödel logics with $\bd$ using Hilbert style
    calculus
  \end{itemize}
\end{frame}

\begin{frame}
  \frametitle{Axiomatisation of $\bd$}

  Baaz gave the following Hilbert style axiomatisation of the $\bd$ 
  operator:
  \begin{align*}
    \axdi &\quad \bd A\lor \lnot\bd A\\
    \axdii &\quad \bd (A\lor B) \limp (\bd A \lor \bd B)\\
    \axdiii&\quad \bd A \limp A\\
    \axdiv &\quad \bd A \limp \bd\bd A\\
    \axdv   &\quad \bd (A\limp B)\limp(\bd A\limp\bd B) \\
    \axdr & \quad A \proves \bd A
  \end{align*}
  
  Extension of the interpretation:
  \[
  \I(\bd A) = \begin{cases} 1 & \text{if\ } \I(A) = 1 \\
    0 & \text{if\ } \I(A) < 1
  \end{cases}
  \]
\end{frame}


\begin{frame}
  \frametitle{Full characterization of $\bd$-Axiomatizability}

  \begin{block}{Recursively axiomatizable}
    \begin{itemize}
    \item finitely valued
    \item $V$ contains a perfect subset~$P$ and for both~$0$ and~$1$
      it holds that they are either in the perfect kernel or isolated
      (4 cases)
    \end{itemize}
  \end{block}

  \pause
  \begin{block}{Not recursively enumerable}
    \begin{itemize}
    \item countably infinite truth value set
    \item either~$0$ or~$1$ is not isolated but not in the perfect kernel
    \end{itemize}
  \end{block}

  %(Baaz, P. 2016)
\end{frame}

%\cutin{Satisfiability}

\begin{frame}
  \frametitle{Axiomatizability}
  
\begin{center}
\begin{tabular}[ht]{l|ccc|c|c}
  \multicolumn{6}{c}{$\VALd_V$ \hspace{2em} $(\iSATd_V)^c$ \hspace{2em} $(\oSATd_V)^c$}\\
  \toprule
       & \multicolumn{3}{c|}{uncountable} & countable inf & finite \\
  with~$\bd$ & $1\in V^\infty$ & $1$ isolated & otherwise  &  &\\
  \midrule
  $0\in V^\infty$ & re & re & not re & / & / \\
  $0$ isolated & re & re & not re & not re & re\\
  otherwise & not re & not re & not re & not re & / \\
  \bottomrule

\end{tabular}

\end{center}

\begin{center}
\begin{tabular}[ht]{l|c|c|c}
  \multicolumn{4}{c}{$\VAL_V$ \hspace{2em} $(\iSAT_V)^c = (\oSAT_V)^c$}\\
  \toprule
  without~$\bd$ & uncountable & countable inf & finite \\
  \midrule
  $0\in V^\infty$ & re & / & / \\
  $0$ isolated & re & only $(\iSAT_V)^c$, $(\oSAT_V)^c$ re & re\\
  otherwise & not re  & not re & / \\
  \bottomrule
\end{tabular}
\end{center}

  (Baaz, P., Zach 2007; Baaz, P. 2016)

\end{frame}

\cutin{Gödel Logics and Order Theory}

\begin{frame}
  \frametitle{Dummett -- number of different logics}

  \begin{block}{Dummett (1959)}
    All propositional (Gödel) logics based on infinite truth value sets
    coincide. Thus, in total there are $\aleph_0$ different propositional logics.
  \end{block}

  \pause
  \begin{block}{Quantified propositional}
    $\aleph_1$ by coding empty and non-empty intervals
  \end{block}

  \pause
  \begin{block}{First order?}
    Lower bounds: always $\aleph_0$ (finitely valued, quantifier alterations,
    Cantor-Bendixon rank)
  \end{block}
\end{frame}

\begin{frame}
  \frametitle{Counting first order logics}

  \begin{block}{Comparing logic}
    If there is an injective, continuous and order preserving
    embedding from $V_1$ into $V_2$ that preserves~$0$ and~$1$, then
    $\gdl{V_1}\supseteq\gdl{V_2}$.
  \end{block}

  \pause
  \begin{block}{Fraïssé  Conjecture (1948), Laver (1971)}
    A $(Q,\le)$ with reflexive and transitive $\le$ is a
    \emph{quasi-ordering}.

    The set of scattered linear orderings ordered by embeddability is
    a \emph{well-quasi-ordering} (does not contain infinite
    anti-chains nor infinitely descending chains)
  \end{block}
\end{frame}

\begin{frame}
  \frametitle{Examples for quasi-orderings}

  \begin{block}{Example}
    The collection of all  linear orderings together with 
    embeddability form a quasi-ordering, but not a partial
    ordering.

    $\bm{\eta}$ and $\bm{\eta}+\mathbf{1}$ are different order types,
    but each embeddable into the other.
  \end{block}

  \pause
  \begin{block}{Example}
    The collection of all linear orderings contain infinite descending
    chains, e.g. the order types of dense suborderings of $\bbR$.
  \end{block}
\end{frame}

\begin{frame}
  \frametitle{Transfer to Gödel logics}

  \begin{block}{Generalized Fraïssé Conjecture}
    The class of countable closed subsets of the reals with respect to
    injective and continuous embeddability is a well-quasi-ordering.
  \end{block}

  \pause
  \begin{block}{GFC for Gödel logics}
    The class of countable Gödel logics, ordered by $\supseteq$, is a
    wqo. 
  \end{block}


  \pause
  \begin{block}{Final result}
    The number of first order Gödel logics is $\aleph_0$.
  \end{block}

  (Beckmann, Goldstern, P. 2008)

\end{frame}

% \begin{frame}
%   \frametitle{Counting in WQOs}

%   Let $(Q, {\le})$ be a wqo
%   with uncountable many $\equiv$-equivalence classes.
%   Then there exists a  1-1 monotone map $f:\omega_1\to Q$. 

%   Each countable Gödel logic is a subset of the fixed countable set
%   of all formulas, thus it cannot contain a copy of $\omega_1$.

%   \pause
%   The class of countable Gödel logics is countable.

%   \pause
%   \begin{block}{Final result}
%     The number of first order Gödel logics is $\aleph_0$.
%   \end{block}
% \end{frame}


\cutin{Gödel Logics and Computation}



\begin{frame}
  \frametitle{Motivation}
\begin{itemize}
\item
Arnon Avron: \emph{Hypersequents, Logical Consequence and Intermediate Logics
  for Concurrency} 
Ann.Math.Art.Int. 4 (1991) 225-248
\pause

\begin{itemize}
\item
\textit{The second, deeper objective of this paper is to contribute towards a
  better understanding of the notion of logical consequence in general, and
  especially its possible relations with parallel computations}
%\pause

\item
\textit{We believe that these logics [...] could serve as bases for parallel
  $\lambda$-calculi.} 
%\pause

\item
\textit{The name ``communication rule'' hints, of course, at a certain
  intuitive interpretation that we have of it as corresponding to the idea of
  exchanging information between two multiprocesses: [...]}
\end{itemize}
%\pause

% \item
% General aim: provide Curry-Howard style correspondences for parallel
% computation, starting from logical systems with good intuitive
%   algebraic / relational semantics. 
\end{itemize}

\end{frame}

\begin{frame}
  \frametitle{Setting the stage}
  \begin{center}
    \tikzset{every picture/.style={scale=1}}
    \begin{tikzpicture}
      \draw (0,-5) node {\ } ;
        \draw (0,0) node[draw,thick,ellipse] {IL} ; 
        \draw (8,0) node[draw,thick,ellipse] {$\lambda$} ; 

% ND 
        \draw (4,0) node[draw,thick,ellipse] {ND} ;
        \draw (2.9,0) node {$\Leftrightarrow$} ;
        \draw (2,0) node[draw,thick,ellipse] {LJ} ;
        \draw (1,0) node {$\Leftrightarrow$} ;
        \draw (6,0) node {$\Longleftrightarrow$} ;
        \draw (6,0) node[anchor=south] {Curry Howard} ;

      \onslide<2>{%
        \draw (4,-2) node {\emph{\alert{Every proof system hides a model of
            computation.}}} ;
     }

%GL
	\onslide<3->{%
      \draw (0,-3) node[draw,thick,ellipse] {GL} ; 
        \draw (2,-3) node[draw,thick,ellipse] {HLK} ;
        \draw (0.9,-3) node {$\Leftrightarrow$} ;
        \draw (8,-3) node[draw,thick,ellipse] {?} ;
        \draw (6,-3) node {$\Longleftrightarrow$?} ;
      \alert{%
        \draw (4,-3) node[draw,thick,ellipse] {HND} ;
        \draw (2.95,-3) node {$\Leftrightarrow$} ;
        % \draw (3.8,-4) node {today's topic} ;
      }}
      \onslide<4->{%
        \draw (3.8,-5) node [align=center]
        {General aim: provide Curry-Howard style\\
          correspondences for parallel computation,\\
          starting from logical systems with good intuitive\\
          algebraic / relational semantics. } ;
    }
    \end{tikzpicture}
  \end{center}
\end{frame}


% \begin{frame}
%   \frametitle{Previous work}
%   \begin{block}{Hirai, FLOPS 2012}
%     \textit{A Lambda Calculus for Gödel-Dummett Logics Capturing
%       Waitfreedom}
%     \begin{itemize}
%     \item change of both syntax and semantics
%     \item different calculus
%     \end{itemize}
%   \end{block}

%   \pause
%   \begin{block}{Baaz, Ciabattoni, Fermüller 2000}
%     \textit{A Natural Deduction System for Intuitionistic Fuzzy Logic}\\
% % why is textit not visible?
%     (will be discussed later)
%     % \begin{itemize}
%     % \item direct translation from HLK
%     % \item inductive definition that does not interplay with normalisation
%     % \item normalisation only via translation to HLK
%     % \end{itemize}
%   \end{block}
% \end{frame}


\begin{frame}
  \frametitle{Wishlist}
  Properties we want to have:

  \begin{block}{(semi) local}
    \begin{itemize}
    \item construction of deductions:\\
      apply ND inspired rules to extend a HND deductions
    \item modularity of deductions:\\
      reorder/restructure deductions
    \item analyticity (sub-formula property, \ldots)
    \end{itemize}
  \end{block}

  \pause
  \begin{block}{normalisation}
    \begin{itemize}
    \item procedural normalisation via conversion steps
%    \item by rearranging deductions -- computation steps
    \end{itemize}
  \end{block}
\end{frame}


\begin{frame}
  \frametitle{Our approach to Hyper Natural Deduction}
  \pause
  \begin{center}
    \tikzset{every picture/.style={scale=0.5}}
    \begin{tikzpicture}
      \draw (-2,0) node {%
        \AxiomC{\alt<2>{$\alert{\Gamma\seq A}$}{$\Gamma\seq A$}} 
        \AxiomC{\alt<3>{$\alert{\Delta\seq B}$}{$\Delta\seq B$}}
        \LeftLabel{(com)}
        \BinaryInfC{%
          $
          \alt<4>{\alert{\Gamma\seq B}}{\Gamma\seq B} 
          \HS 
          \alt<5>{\alert{\Delta\seq A}}{\Delta\seq A}
          $
        }
        \DisplayProof
      } ;
      \onslide<2>{%
        \draw (6.8,0) node {%
          \alert{%
            \AxiomC{$\Gamma$}
            \DeduceC{$A$}
            \noLine
            \ProofLabel{}
            \UnaryInfC{\strut}
            \DisplayProof
          }
        };
      }
      \onslide<3>{%
        \draw (6.8,0) node {%
          \AxiomC{$\Gamma$}
          \DeduceC{$A$}
          \noLine
          \ProofLabel{}
          \UnaryInfC{\strut}
          \DisplayProof
        };
      }
      \onslide<4>{%
        \draw (6.8,0) node {%
          \AxiomC{$\Gamma$}
          \DeduceC{$A$}
          \ProofLabel{\alert{\llap{com}}}
          \UnaryInfC{$\alert{\strut B}$}
          \DisplayProof
        };
      }
      \onslide<5->{%
        \draw (6.8,0) node {%
          \AxiomC{$\Gamma$}
          \DeduceC{$A$}
          \ProofLabel{\llap{com}}
          \UnaryInfC{$\strut B$}
          \DisplayProof
        };
      }
      \onslide<3>{%
        \draw (10,0) node {%
          \alert{%
            \AxiomC{$\Delta$}
            \DeduceC{$B$}
            \noLine
            \UnaryInfC{\strut}
            \DisplayProof
          }
        };
      }
      \onslide<4>{%
        \draw (10,0) node {%
          \AxiomC{$\Delta$}
          \DeduceC{$B$}
          \noLine
          \ProofLabel{}
          \UnaryInfC{\strut}
          \DisplayProof
        };
      }
      \onslide<5>{%
        \draw (10,0) node {%
          \AxiomC{$\Delta$}
          \DeduceC{$B$}
          \ProofLabel{\llap{\alert{$\overline{\text{com}}$}}}
          \UnaryInfC{\strut$\alert{A}$}
          \DisplayProof
        };
      }
      \onslide<6->{%
        \draw (10,0) node {%
          \AxiomC{$\Delta$}
          \DeduceC{$B$}
          \ProofLabel{\llap{{$\overline{\text{com}}$}}}
          \UnaryInfC{\strut $A$}
          \DisplayProof
        };
      }
    \end{tikzpicture}  
  \end{center}

  \pause\pause\pause\pause\pause
  \begin{itemize}
  \item consider sets of derivation trees\mis
  \item divide communication (and split) into two dual parts\mis
  \item search for minimal set of conditions  that provides sound
    and complete deduction system
  \end{itemize}
\end{frame}

\begin{frame}[t]
  \frametitle{Reasoning in \HND}
  Double extension in \emph{the spirit} of ND:
  \begin{itemize}
  \item from one tree to set of trees
  \item additional rules
  \end{itemize}
  
  \pause
  \begin{center}
    From
    \quad
    \bottomAlignProof
    \AxiomC{}
    \DeduceC{$A$}
    \DisplayProof
    \quad
    and
    \quad
    \bottomAlignProof
    \AxiomC{}
    \DeduceC{$B$}
    \DisplayProof
    \quad
    % $\leadsto$
    form
    \quad
    \bottomAlignProof
    \AxiomC{}
    % \ProofLabel{$\si_1$}
    \DeduceC{$A$}
    \ProofLabel{$\ComR {x}AB$}
    \UnaryInfC{$B$}
    \DisplayProof
    \ 
    \bottomAlignProof
    \AxiomC{}
    % \ProofLabel{$\si_2$}
    \DeduceC{$B$}
    \ProofLabel{$\ComR {\bar x}BA$}
    \UnaryInfC{$A$}
    \DisplayProof
  \end{center}

  \pause
  \begin{center}
    From
    \quad
    \bottomAlignProof
  \AxiomC{$\Ga,\De$}
  \ProofLabel{$\si$}
  \DeduceC{$A$}
  \DisplayProof 
  \quad
%  $\leadsto$
form
  \quad
    \bottomAlignProof
  \AxiomC{$[\Ga],\De$}
  \ProofLabel{$\si$}
  \DeduceC{$A$}
  \ProofLabel{$\SplitR x{}{\Gamma}{\Delta}$}
  \UnaryInfC{$A$}
  \DisplayProof
%\ 
    \bottomAlignProof
  \AxiomC{$\Ga,[\De]$}
  \ProofLabel{$\si$}
  \DeduceC{$A$}
  \ProofLabel{$\SplitR {\bar x}{}{\Delta}{\Gamma}$}
  \UnaryInfC{$A$}
  \DisplayProof
  \end{center}
\end{frame}

\begin{frame}
  \frametitle{Results for \HND}
  %\begin{block}{Properties of \HND}
    \begin{itemize}
    \item sound and complete for standard first order Gödel logic\bis
    \item procedural normalization\bis
    \item sub-formula property
    \end{itemize}
  %\end{block}

  (Beckmann, P. 2016)
% \end{frame}

  \pause

% \begin{frame}
  \begin{block}{Beauty of this system}
  \begin{itemize}
  \item Hyper rules -- derivations are completely in ND style\bis
  \item Hyper rules mimic HLK/BCF system\bis
  \item natural style of deduction\bis
  % \item but: procedural definition (like BCF system):
  %   \begin{itemize}
  %   \item difficult to check whether a given figure forms a proof
  %   \item difficult to reason on normalisation (needs reshuffling of
  %     proof trees)
  %   \end{itemize}
  \end{itemize}
\end{block}
% \pause
  % We need criteria to check whether a set of trees forms a proof!
\end{frame}



\cutin{Other topics}



%\cutin{Other semantics}

% \begin{frame}
%   \frametitle{Gödel, Kripke frames and Intuitionistic Logic}
  
%   \begin{block}{Gödel (1933)}
%     Wanted to show that Intuitionistic Logic does not have a finite
%     matrix, i.e., is not a finitely valued logic.
%   \end{block}

%   \begin{block}{Kripke (60ies)}
%     Semantic for Intuitionistic Logic based on trees.

%     Axiom $(A\limp B) \lor (B\limp A)$ of Gödel logics implies
%     linearity on Kripke frames.
%   \end{block}

% \end{frame}

% \begin{frame}
%   \frametitle{Relating Gödel logics and logic on Kripke frames}

%   \begin{block}{`Truth values in Kripke frames'}
%     Sets of worlds in which a formula is true, is upward closed.

%     The set of upwards closed sets in $K$, $\mathrm{Up}(K)$, is a
%     Gödel algebra.

%     A (order theoretic) upper limit point~$w$ generates two distinct
%     upward closed sets:
%     \begin{align*}
%       w^\vup &= \{ v \in K \suchthat R(w,v) \}\\
%       w^{\vup*} &= w^\vup\setminus \{w\}
%     \end{align*}
%   \end{block}
% \end{frame}

% \newcommand{\nd}[1]{\fill (0,#1) circle (2pt);}

% \begin{frame}
%   \frametitle{Example}
%   $K = \omega + a + \omega^{-1}$ ($a$ being an arbitrary element)
  
%   The $a$ is upper and lower limit of the accessibility relation.
  
%   \begin{align*}
%     \mathrm{Up}(K) = & \{ n^\vup \suchthat n\in \omega \}\ \cup\\
%     & \{ a^\vup \}\ \cup\\
%     & \{\omega^{-1}\}\ \cup\\
%     & \{n^\vup \suchthat n\in \omega^{-1}\}
%   \end{align*}
% \end{frame}

% \begin{frame}
%   \frametitle{Mapping Kripke worlds into the reals}
%   Embed $\mathrm{Up}(K)$ into the truth value set such that the order
%   and existing infima and suprema are preserved.
%   \[
%   \begin{pgfpicture}{-5cm}{-4cm}{5cm}{2cm}
%     % \pgfsetlinewidth{0.8pt}
%     % \pgfxyline(-5,-5)(5,-5)
%     \pgfxyline(0,1)(5,1)
%     \pgfputat{\pgfxy(0,1)}{\pgfbox[right,center]{$V_K\quad$}}
%     \pgfxyline(1,1.1)(1,0.9)
%     \pgfxyline(2,1.1)(2,0.9)
%     \pgfxyline(3,1.1)(3,0.9)
%     \pgfxyline(4,1.1)(4,0.9)
%     \pgfputat{\pgfxy(-3,-3)}{\pgfbox[right,center]{$\mathrm{Up}(K)$}}
%     \pgfputat{\pgfxy(0,-3.5)}{\pgfbox[left,center]{ $K$}}
%     % \pgfsetlinewidth{0.4pt}
%     \pgfheaplabeledcentered{-1cm}{1cm}{}
%     \pgfheaplabeledcentered{-1.93cm}{2cm}{}
%     \pgfheaplabeledcentered{-2.07cm}{3cm}{}
%     \pgfheaplabeledcentered{-3cm}{4cm}{}
%     \pgfcircle[fill]{\pgfpoint{0cm}{-1cm}}{2pt}
%     \pgfcircle[fill]{\pgfpoint{0cm}{-2cm}}{2pt}
%     \pgfcircle[stroke]{\pgfpoint{0cm}{-3cm}}{2pt}
%     \pgfsetdash{{1pt}{1pt}}{0pt}
%     \pgfxyline(0,-1.4)(0,-2)
%     \pgfxyline(0,-2.4)(0,-2.93)
%     \pgfxyline(0,-3.07)(0,-3.8)
%     \pgfxyline(1.4,0.9)(2,0.9)
%     \pgfxyline(3.4,0.9)(4.6,0.9)
%     \pgfputat{\pgfxy(4,0.3)}{\pgfbox[center,bottom]{$w^\vup$}}
%     \pgfputat{\pgfxy(3,0.3)}{\pgfbox[center,bottom]{$w_3^\vup$}}
%     \pgfputat{\pgfxy(2,0.3)}{\pgfbox[center,bottom]{$w_3^{*\vup}$}}
%     \pgfputat{\pgfxy(1,0.3)}{\pgfbox[center,bottom]{$w_3^\vup$}}
%     \pgfputat{\pgfxy(0,-2.1)}{\pgfbox[left,top]{$w_3$}}
%     \pgfputat{\pgfxy(0,-0.9)}{\pgfbox[center,bottom]{$w_1$}}
%   \end{pgfpicture}
%   \]
% \end{frame}


% \newcommand{\pn}[2]{%
%   \draw[line width=0.5pt] (#1,1.8) -- (#1,2.2);
%   \draw (#1,1.8) node[anchor=north] {#2};
% }
% \begin{frame}
%   %\frametitle{The case of $\bbQ$}
%   \frametitle{The logic $\lgc(\bbQ)$}
%   Let $K=\bbQ$. Every rational number is a supremum. Thus, every
%   rational number is torn apart into two points. 

%   \pause
%   \medskip
%   Take an enumeration of $\bbQ = \{q_1,q_2,\dots\}$ and consider the
%   following enumeration induced on 
%   $\bbQ'= \{q_1^\vup,q_1^{\vup*},q_2^{\vup},q_2^{\vup*},\dots\}$. 

%   \begin{tikzpicture}
%     \draw[line width=1pt,dotted] (2,0) -- (2,4);
%     %\draw (0,4) circle (.2mm);
%     %\draw (0,4) arc (210:330:3cm and 6cm);
%     \pn{6}{$q_1^{\vup}$}
%     \draw (2,1) node[anchor=north west] {$q_1$};
%     \draw[dotted] (0,3) to[out=280,in=180] (2,1) to[out=0,in=260] (4,3);
%     \draw[line width=1pt] (5,2) -- (10,2);
%     \onslide<3->{
%       \pn{6.5}{$q_1^{\vup*}$}
%     }
%     \onslide<4->{
%       \draw[dotted] (0.5,3.5) to[out=280,in=180] (2,2) to[out=0,in=260] (3.5,3.5);
%       \draw (2,2) node[anchor=north west] {$q_2$};
%       \pn{8}{$q_2^{\vup}$}
%       \pn{8.5}{$q_2^{\vup*}$}
%     }
%   \end{tikzpicture}
% \end{frame}

% \begin{frame}
%   \frametitle{The logic $\lgc(\bbQ)$ cont.}

%   An embedding of $\bbQ'$ into $[0,1]$ preserving the order, infima
%   and suprema will generate a set which is isomorph to the border
%   points of the Cantor middle third set. The closure of this set is
%   the Cantor middle third set.

%   \pause
%   \medskip
%   Thus, $\lgc(\bbQ) = \gdl {\bbC_{[0,1]}} = \gdl{[0,1]}$
% \end{frame}

\begin{frame}
  \frametitle{Gödel Logics and Kripke Frames}

  \begin{block}{Gödel logic to Kripke frame}
    For each Gödel logic there is a countable linear Kripke frame such
    that the respective logics coincide.
  \end{block}

  \begin{block}{Kripke frames to Gödel logic}
    For each countable linear Kripke frame there is a Gödel truth
    value set such that the respective logics coincide.
  \end{block}

  (Beckmann, P. 2007)
\end{frame}

\begin{frame}
  \frametitle{Going beyond $\bbR$}

  \begin{block}{Takano (1987)}
    Axiomatization of the logic of linear Kripke frames based on
    $\bbQ$ (which is that of $\gdl{[0,1]}$).

    Axiomatization of the logic of linear Kripke frames based on
    $\bbR$ needs an additional axiom.
    
  \end{block}
\end{frame}

\cutin{Monadic Fragment}

\begin{frame}
  \frametitle{Decidability of validity and satisfiability}
  \begin{center}
\begin{tabular}[ht]{l|l|c|c}
  \toprule
      \multicolumn{2}{c|}{} & validity & satisfiability \\
  \midrule
  finite~$V$ & full monadic & Yes  & Yes \\
  \midrule
  infinite $V$ & full monadic    & No  & No  \\
  with         & witnessed       & No  & No  \\
  $\bd$        & quantifer prefix & $\forall^*\exists^*$ & (open!) $\exists^*\forall^*$\\
  \midrule
               & full monadic    & No  & $0$ isolated: Yes \\
               &                 &     & $0$ not isolated: No \\
  infinite $V$ & prenex          & No  & Yes \\
  without      & $\exists$, $\lnot$-free & No & Yes \\
  $\bd$        & $\exists$       & No  & Yes \\
               & $\lnot$-free    & No  & Yes \\
               & witnessed       & No  & Yes \\
  \bottomrule
\end{tabular}
\end{center}
  (Baaz, Ciabattoni, P. 2011; Baaz, P. 2016)
\end{frame}

%\cutin{Expressivity of Monadic logics}
  
\begin{frame}
  \frametitle{Expressivity of Monadic logics}
  \begin{center}
    Take standard first order language.

    \bigskip
    Question: What can we express over complete linear orders?

    \pause
    \bigskip
    Same question with one (1) monadic predicate symbol?
  \end{center}
\end{frame}

\begin{frame}
  \frametitle{The results}
  \begin{theorem}
    If $0\prec\al\prec\be\prec\om{\omega}$ with $\be\succeq\omega$, then
    $A_{\alpha,\beta} \in L(\alpha)$, 
    but $A_{\alpha,\beta} \not\in L(\beta)$.
  \end{theorem}

  \pause
  \begin{theorem}
  If $0\prec\al\prec\be\prec\om{\omega}$, then
  $A^*_\al \in L(\al^*)$, 
  but $A^*_\al \notin L(\be^*)$.
  \end{theorem}

  (Beckmann, P 2014)
\end{frame}

\begin{frame}
  \frametitle{Basic idea}
  \begin{center}
    Separate $2$ from $3$-valued logic

    \medskip
    \begin{tikzpicture}
      \draw (0,3) -- (6cm,3) ;
      \draw (0,3.2) -- (0,2.8) ;
      \draw (6,3.2) -- (6,2.8) ;
      \draw (0,2) -- (6cm,2) ;
      \draw (0,2.2) -- (0,1.8) ;
      \draw (6,2.2) -- (6,1.8) ;
      \draw (3,2.2) -- (3,1.8) ;
      \draw (0,2.5) node {$0$} ;
      \draw (6,2.5) node {$1$} ;
      \draw (3,2.5) node {$0.5$} ;
      \onslide<2>{      \draw (0,1.5) node {$x_3$} ;
      \draw (6,1.5) node {$x_1$} ;
      \draw (3,1.5) node {$x_2$} ;}
    \end{tikzpicture}

    \medskip
    \pause
    $(x_1\limp x_2) \lor (x_2 \limp x_3)$

    
  \end{center}
\end{frame}

% \begin{frame}
%   \frametitle{Example}
%   \begin{center}
%     $\alpha = \omega^22+\omega3 + 1\qquad\beta = \omega^22+\omega4$\\
%     \begin{tikzpicture}
%     \tword{0pt}{0pt}{1cm}
%     \tword{0pt}{1cm}{2cm}
%     \oneord{-1pt}{2cm}{2.4cm}
%     \oneord{-1pt}{2.4cm}{2.8cm}
%     \oneord{-1pt}{2.8cm}{3.2cm}
%     \zerord{-1pt}{3.2cm} \zerord{-1pt}{3.4cm}
%     \draw (0pt,0mm) node[anchor=west] {\tiny $0$} ;
%     \draw (0pt,0.33cm) node[anchor=west] {\tiny $\omega$} ;
%     \draw (0pt,0.66cm) node[anchor=west] {\tiny $\omega2$} ;
%     \draw (0pt,1cm) node[anchor=west] {\tiny $\omega^2$} ;
%     \draw (0pt,1.33cm) node[anchor=west] {\tiny $\omega^2+\omega$} ;
%     \draw (0pt,1.66cm) node[anchor=west] {\tiny $\omega^2+\omega2$} ;
%     \draw (0pt,2cm) node[anchor=west] {\tiny $\omega^22$} ;
%     \draw (0pt,2.4cm) node[anchor=west] {\tiny $\omega^22+\omega$} ;
%     \draw (0pt,2.8cm) node[anchor=west] {\tiny $\omega^22+\omega2$} ;
%     \draw (0pt,3.2cm) node[anchor=west] {\tiny $\omega^22+\omega3$} ;
%     \draw (0pt,3.4cm) node[anchor=west] {\tiny $\omega^22+\omega3 + 1$} ;
%     \draw (0pt,-8pt) node {$\alpha$} ;
%     \draw[dotted] (-1pt,3.6cm) circle (1mm) ;
%     \draw (0pt,3.6cm) node[anchor=west] {\tiny $\alpha$} ;
%     \def\aa{70pt}
%     \draw (\aa,-10pt) node {$\beta$} ;
%     \tword{\aa}{0pt}{1cm}
%     \tword{\aa}{1cm}{2cm}
%     \oneord{\aa - 1pt}{2cm}{2.4cm}
%     \oneord{\aa - 1pt}{2.4cm}{2.8cm}
%     \oneord{\aa - 1pt}{2.8cm}{3.2cm}
%     \oneord{\aa - 1pt}{3.2cm}{3.6cm}
%     \zerord{\aa-1pt}{3.6cm}
%     \draw[dotted] (\aa-1pt,3.8cm) circle (1mm) ;
%     \draw (\aa,3.8cm) node[anchor=west] {\tiny $\beta$} ;
%     %
%     \pause
%     %
%     \def\aa{-180pt}
%     \def\bb{20mm}
%     \twohor{\aa+128pt}{\aa+88pt}{\bb}
%     \twohor{\aa+88pt}{\aa+48pt}{\bb}
%     \onehor{\aa+48pt}{\aa+34pt}{\bb}
%     \onehor{\aa+34pt}{\aa+20pt}{\bb}
%     \onehor{\aa+20pt}{\aa+6pt}{\bb}
%      \zerohor{\aa-10pt}{\bb} \zerohor{\aa+6pt}{\bb} \zerohor{\aa-26pt}{\bb}
%     \draw (\aa+128pt,\bb+1mm) node[rotate=90,anchor=west] {\tiny $\up 0$} ;
%     \draw (\aa+128pt,\bb-1mm) node[anchor=north] {\tiny $\kI$} ;
%     \draw (\aa+108pt, \bb+5mm) node[anchor=center] {\tiny $\subset$} ;
%     \draw (\aa+88pt, \bb+1mm) node[rotate=90,anchor=west] {\tiny $\up {\omega^2}$} ;
%     \draw (\aa+68pt, \bb+5mm) node[anchor=center] {\tiny $\subset$} ;
%     \draw (\aa+48pt, \bb+1mm) node[rotate=90,anchor=west] {\tiny $\up {({\omega^2}2)}$} ;
%     \draw (\aa+41pt, \bb+5mm) node[anchor=center] {\tiny $\subset$} ;
%     \draw (\aa+34pt, \bb+1mm) node[rotate=90,anchor=west] {\tiny $\up {(\omega^22+\omega)}$} ;
%     \draw (\aa+27pt, \bb+5mm) node[anchor=center] {\tiny $\subset$} ;
%     \draw (\aa+20pt, \bb+1mm) node[rotate=90,anchor=west] {\tiny $\up {(\omega^22+\omega2)}$} ;
%     \draw (\aa+13pt, \bb+5mm) node[anchor=center] {\tiny $\subset$} ;
%     \draw (\aa+6pt, \bb+1mm) node[rotate=90,anchor=west] {\tiny $\up {(\omega^22+\omega3)}$} ;
%     \draw (\aa-2pt, \bb+5mm) node[anchor=center] {\tiny $\subset$} ;
%     \draw (\aa-10pt, \bb+1mm) node[rotate=90,anchor=west] {\tiny $\up {(\omega^22+\omega3+1)}$} ;
%     \draw (\aa-18pt, \bb+5mm) node[anchor=center] {\tiny $\subset$} ;
%     \draw (\aa-26pt, \bb+1mm) node[rotate=90,anchor=west] {\tiny $\up {\alpha} = \emptyset$} ;
%     \draw (\aa-26pt, \bb-1mm) node[anchor=north] {\tiny $\kO$} ;
%     \draw (\aa+130pt,\bb) node[anchor=west] {\tiny $\Up(\alpha)$} ;
%     \def\bb{6mm}
%     \twohor{\aa+128pt}{\aa+88pt}{\bb}
%     \twohor{\aa+88pt}{\aa+48pt}{\bb}
%     \onehor{\aa+48pt}{\aa+34pt}{\bb}
%     \onehor{\aa+34pt}{\aa+20pt}{\bb}
%     \onehor{\aa+20pt}{\aa+6pt}{\bb}
%     \onehor{\aa+6pt}{\aa-10pt}{\bb}
%     \zerohor{\aa-10pt}{\bb} \zerohor{\aa+6pt}{\bb} \zerohor{\aa-26pt}{\bb}
%     \draw (\aa+128pt,\bb-1mm) node[rotate=90,anchor=east] {\tiny $\up 0$} ;
%     \draw (\aa+128pt,\bb+1mm) node[anchor=south] {\tiny $\kI$} ;
%     \draw (\aa+108pt, \bb-5mm) node[anchor=center] {\tiny $\subset$} ;
%     \draw (\aa+88pt, \bb-1mm) node[rotate=90,anchor=east] {\tiny $\up {\omega^2}$} ;
%     \draw (\aa+68pt, \bb-5mm) node[anchor=center] {\tiny $\subset$} ;
%     \draw (\aa+48pt, \bb-1mm) node[rotate=90,anchor=east] {\tiny $\up {({\omega^2}2)}$} ;
%     \draw (\aa+41pt, \bb-5mm) node[anchor=center] {\tiny $\subset$} ;
%     \draw (\aa+34pt, \bb-1mm) node[rotate=90,anchor=east] {\tiny $\up {(\omega^22+\omega)}$} ;
%     \draw (\aa+27pt, \bb-5mm) node[anchor=center] {\tiny $\subset$} ;
%     \draw (\aa+20pt, \bb-1mm) node[rotate=90,anchor=east] {\tiny $\up {(\omega^22+\omega2)}$} ;
%     \draw (\aa+13pt, \bb-5mm) node[anchor=center] {\tiny $\subset$} ;
%     \draw (\aa+6pt, \bb-1mm) node[rotate=90,anchor=east] {\tiny $\up {(\omega^22+\omega3)}$} ;
%     \draw (\aa-2pt, \bb-5mm) node[anchor=center] {\tiny $\subset$} ;
%     \draw (\aa-10pt, \bb-1mm) node[rotate=90,anchor=east] {\tiny $\up {(\omega^22+\omega4)}$} ;
%     \draw (\aa-18pt, \bb-5mm) node[anchor=center] {\tiny $\subset$} ;
%     \draw (\aa-26pt, \bb-1mm) node[rotate=90,anchor=east] {\tiny $\up {\beta} = \emptyset$} ;
%     \draw (\aa-26pt, \bb+1mm) node[anchor=south] {\tiny $\kO$} ;
%     \draw (\aa+130pt,\bb) node[anchor=west] {\tiny $\Up(\beta)$} ;
    
%     \def\bb{13mm}
%     \draw (\aa+128pt,\bb) node {\tiny $x^3_1$} ;
%     \draw (\aa+88pt,\bb) node {\tiny $x^2_1$} ;
%     \draw (\aa+48pt,\bb) node {\tiny $x^2_2$} ;
%     \draw (\aa+34pt,\bb) node {\tiny $x^1_1$} ;
%     \draw (\aa+20pt,\bb) node {\tiny $x^1_2$} ;
%     \draw (\aa+6pt,\bb) node {\tiny $x^1_3$} ;
%     \draw (\aa-10pt,\bb) node {\tiny $x^1_4$} ;
%   \end{tikzpicture}
% \end{center}
% \end{frame}

%\cutin{Proof Theory}

\begin{frame}
  \frametitle{Proof theory}
  \begin{block}{Hypersequent}
    $\Gamma$, $\Pi$ finite multisets of formulas
    \[
    \Gamma_1 \seq \Pi_1 \hh \dots \hh \Gamma_n \seq \Pi_n
    \]
  \end{block}\pause
  \begin{block}{Rules}
    internal structural and logical (like LK)

    external weakening and contraction

    \[ \infer[(com)]{G\hh G'\hh \Gamma, \Delta' \seq A\hh \Delta, 
\Gamma' \seq A' }{G \hh \Gamma, \Delta \seq A & 
G'\hh \Gamma', \Delta' \seq A'} \]

%    \[ \infer[(cut)]{G \hh G' \hh \Gamma, \Gamma' \seq C}
%{G \hh \Gamma' \seq A  & G' \hh A, \Gamma \seq C} \]
  \end{block}
\end{frame}

\begin{frame}
  \frametitle{Calculus HG}
  \begin{block}{Sound and completeness}
    \textsf{\bfseries HG} is sound and complete for Gödel logics
    (propositional and first order)

    (Avron 1992, Baaz, Zach 2000)
  \end{block}\pause
  \begin{block}{Linearity}
    \[
    \infer{\seq (A \Impl B) \vee (B \Impl A)\strut}{%
      \onslide<3->{\infer{\seq (A \Impl B) \vee (B \Impl A) \hh \seq (A \Impl B)
        \vee (B \Impl A)\strut}{%
        \onslide<4->{\infer{\seq A \Impl B \hh \seq B \Impl A\strut}{%
          \onslide<5->{\infer[\onslide<6->{(com)}]{A\seq B \hh B \seq A\strut}{%
            \onslide<6->{A \seq A &B \seq B\strut}}}}}}}}
    \]
  \end{block}
\end{frame}


% \def\aaaa#1#2#3#4#5#6#7{%
%   \only<#6->{\draw (#1, #3) -- (#1, #2);
%   \draw (#1, #3) node[anchor=south] {\alt<#6>{\alert{#4}}{#4}};
%   \draw (#1, #2) node[anchor=north] {\alt<#6>{\alert{#5}}{#5}};}
%   \only<#6>{\draw (0,-3) node[anchor=south west] {#7};}
% }
% \begin{frame}
%   \frametitle{History}
%   \begin{center}
%   \begin{tikzpicture}
%     \shade[left color=gray,right color=gray!30] (0,0) rectangle +(10,0.5);
%     \draw (0,1.7) node[anchor=south west] {Timeline};
%     \aaaa{0.42}{-0.4}{0.5}{1933}{Gödel\strut}1{finitely valued logics}
%     \aaaa{4.08}{-0.4}{0.5}{1959}{Dummett\strut}2{infinitely valued propositional Gödel logics}
%     \aaaa{5.57}{-0.4}{0.5}{1969}{Horn\strut}3{linearly ordered Heyting algebras}
%     \aaaa{7.71}{-0.4}{0.5}{1984}{Takeuti-Titani\strut}4{intuitionistic fuzzy logic}
%     \aaaa{8.71}{-0.9}{1}{1991}{Avron\strut}5{hypersequent calculus}
%     \aaaa{9.4}{-1.4}{0.5}{1998}{Hájek\strut}6{$t$-norm based logics}
%     \aaaa{8.57}{-1.9}{1.5}{since 90ies}{Viennese group\strut}7{proof theory,
%       \#, Kripke, qp, fragments, \ldots}
%   \end{tikzpicture}
%   \end{center}
% \end{frame}


% \begin{frame}
%   \frametitle{What we didn't discuss}
%   \begin{itemize}
%   \item Algebraic connections\\
%     Gödel algebras, first order theory with safe interpretations in algebras\sis
%   \item Proof theory\\
%     Hypersequent calculus, calculus of relations\sis
%   \item SAT and VAL in monadic (and sub-) class\\
%     varied landscape, partly decidable, partly undecidable\sis
%   \item Modalities\sis
%   \item Complexity\sis
%   \item Relation to game theory\sis
%   \end{itemize}
% \end{frame}


\begin{frame}
  \frametitle{Open problems}
  \begin{itemize}
  \item intensional versus extensional definition\sis
  \item Herbrand disjunctions\sis
  \item Calculi for other than the standard logic\sis
  \item equivalent of $L(\bbR)$, the logic of the Kripke frame of
    $\bbR$ within an extended `real' setting\sis
  \item equivalence of `one logic per truth-value set' for Gödel
    algebras\sis
  \item quantified propositional logics -- largely untapped\sis
  \item computational model
  \end{itemize}
\end{frame}

\begin{frame}
  \frametitle{Recapitulation}
  \pause
  \begin{block}{Standard meta-theory}
    \begin{itemize}
    \item soundness, completeness
    \item axiomatizability
    \item decidability of satisfiability an validity
    \item sub-classes, monadic and other fragments
    \item proof theory
    \item representation theorems
    \end{itemize}
  \end{block}

  \pause
  \begin{block}{Relation to different areas}
    \begin{itemize}
    \item order theory, topology, polish spaces
    \item Kripke frames
    \item (Heyting) algebras
    \item computation
    \item \ldots
    \end{itemize}
  \end{block}

  % \pause
  % \begin{center}
  %   \vspace{-10pt}
  %   Proper logic -- no fuzzyness
  % \end{center}
\end{frame}

\begin{frame}
  \frametitle{Conclusion}
  Although not \alert{traditional} logic, it provides a rich meta-theory and
  there are still many unexplored topics.

  Application-wise of relevance due to ease of modelling and
  well-behaved logic even on first-order level. (medical expert
  system, database modelling, \ldots)

  % Lots of math, in particular order theory  and topology of the reals

  \pause
  Fun!

  \pause
  \begin{center}
    \Large Thanks
  \end{center}
\end{frame}

\end{document}



%%% Local Variables:
%%% mode: latex
%%% TeX-master: t
%%% End:
