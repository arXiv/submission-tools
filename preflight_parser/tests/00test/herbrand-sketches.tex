%
% herbrand-sketches.tex
%


\section{Sketches made from Herbrand Disjunctions}
\label{sec:herbrandsketch}

In \cite{Prei97KGC} we gave a definition of a construction and proved
the equivalence of sketches and proofs by translating a given proof
in~\pelk into a proof by constructions. In this way we depend on the
special calculus~\pelk and speed-up results as obtained in
section~\ref{sec:orevkov} would not be generally valid. Therefore we
aimed at shifting the translation from \pelk to \hd and in this way we
get rid of any dependencies on a calculus.

\subsection{Transformation from Herbrand disjunction to Construction}
\label{sec:herbrandsketch:herbratransformation}

We want to note that we cannot start with the \hd of an arbitrary formula,
because we can only prove a formula with a sketch if it is a formula of
projective geometry. We therefore assume that a given formula~$A$ was proven
in a sound calculus for projective geometry.

Therefore the formula $\Ax\limp A$ is valid in first order logic and has
an \hd of the form
   $$(\Ax(t_1)\limp A(t_1))\lor\dots\lor(\Ax(t_n)\limp A(t_n))$$
So the formula we really want to proof with sketches is
   $$A(t_1)\lor\dots\lor A(t_n).$$

The transformation from a Herbrand disjunction to a construction is done in
the following steps:
\begin{enumerate}
\item Minimizing the term depth
\item Reformation of the geometric part into a DNF
\item Building up the construction from the literals and terms
\end{enumerate}

Part~1 is described in sec.~\ref{sec:herbrand:minimize}, part~2 is
trivial. 


\subsubsection*{Building up the construction}
\label{sec:herbrandsketch:transformation:construction}

The way to building up the construction is as follows:
We start with an empty node and add step by step for each literal in the \hd a
case-distinction action, i.e. two vertices labeled with~$L$ and~$\bar L$,
respectively. But before these actions can be done the terms occuring in the
literal have to be build up, too. So we will refer to the partial construction
of first building the terms, if necessary, and then the case-distinction
action as the \emph{construction of the literal~$L$}, denoted as~$\cC(L)$.

If we have already a part of the construction and want to add a new literal,
then we have to find the \emph{open} nodes, i.e.\ those which are not
contradictious and 
which do not prove a part of the \hd. In all these open nodes the construction
for the literal has to be added. Note that some of the actions of the added
construction can be redundant or contradictious, according to whether subterms
are already constructed and which further identities are shown. 
%This operation
%of concatenating two trees, we denote with~$\cT\oplus \cT'$.

After having added all the literals there should be only closed nodes, some
closed by contradictions, some by parts of the \hd, but it is not obvious why
all the nodes are closed. The following lemma gives us a hint:

\begin{lemma}\label{lemma:1}
If there is a construction for a formula~$A$ and in another construction there
is a node which derives~$\bar A$, then below this node there are only
contradictious leaves.
\end{lemma}

\beginproof Take this node and attach the whole construction
deriving~$A$. This of course leads to something which is not a construction,
but with some operations similar to those for converting a preconstruction to
a construction (see~\cite{Prei96MT}) it is possible to obtain a construction
in which all the nodes are obviously contradictious. \endproof

So now assume that there is a node in the construction from above which is
open, i.e. it is not contradictious and does not derive the \hd. Then for
every sub-conjunction there must be at least one literal whose
negation is in the  
node. We now need a construction of the \hd to close this node. This is not a
vicious circle, because since we assumed that the original formula was proven
in some reasonable calculus, there is a proof of this formula in~\pelk,
which can be translated to a construction.

But this would cause another dependency on the calculus~\pelk, which we
wanted to overcome to obtain general results on the complexity of
constructions. For this case we will need the following lemmas:

\begin{lemma}\label{lemma:2}
If below a node there are only contradictory nodes and no occurrence of a
critical constellation, then the node by itself is contradictory. 
\end{lemma}

\beginproof Let~$N$ be the upper node, $N'$ ($L, R$) be the successor(s) and
 $\alpha$, ($\alpha, \beta$) the actions leading from~$N$ to~$N'$ ($L, R$).
We have to scrutinize all the possible actions on whether they could generate
a contradiction in the next node:

If $\alpha = [PQ]$ or~$\alpha = (gh)$, then the new formulas are $P\cI [PQ]$
and~$Q\cI [PQ]$, whose negation not in the upper
node because the term isn't constructed yet. Therefore if in~$N'$
there is a contradiction, it must be also in~$N$. 

The argumentation for the other cases are similar. The only situation when this
argumentation can not work, is if there is a critical constellation where the
following substitutions generate the contradictions.
\endproof

Now we have to show how to eliminate rules between certain occurrences
of critical constellations.

\begin{lemma}\label{lemma:3}
If all the leafs under a node~$N$ are contradictory, then these
contradictions are automatically derived with the critical constellation
procedure, i.e.\ the procedure which only solves critical constellations until
none is left.
\end{lemma}

\beginproof 
We now assume that we have a node~$N$ with two successors~$L$ and~$R$,
the action is a critical constellation $(P,Q;g,h)$, i.e. from~$N$
to~$L$ there is a vertex labeled with~$P=Q$ and from~$N$ to~$R$ there
is a vertex labeled with~$g=h$. Furthermore assume that~$L$ and~$R$
are contradictory (otherwise use lemma~\ref{lemma:2}).

Assume that~$N$ is not contradictory (otherwise we can kill the
critical constellation below and proceed to the next one). Therefore
the critical constellation actions generated the contradictions.

We now have to analyze the possibilities how this critical
constellation has emerged out of the predecessor of~$N$. There are
four possibilities for actions leading to the node~$N$: The joining,
the meeting, a case distinction or a critical constellation. The first
two are analog.

First for the joining: It must be the joining of~$P$ and~$Q$,
otherwise the predecessor already has a critical constellation. So the
critical constellation is in fact~$(P,Q;[PQ],h)$. The contradiction
in~$R$, where the vertex labeled with~$[PQ]=h$ is leading to, must be
of the form~$(X\cI h, X\notcI h)$ or~$h\not= h$. In the latter case
there must be a formula~$[PQ]\not=h$ in~$N$ and therefore also in
the predecessor of~$N$, which is impossible since this term is
constructed on the vertex leading to~$N$.

If the contradiction is~$(X\cI h, X\notcI h)$, then it can come from
either~$(X\cI [PQ], X\notcI h)$ or from~$(X\cI h, X\notcI [PQ])$
in~$N$. In the latter case the formulas~$X\notcI [PQ]$ would be
present in the predecessor of~$N$, too, which is impossible. In the
former case~$X$ has to be either~$P$ or~$Q$. But if~$X=P$ then we
would have~$(P\cI h, P\notcI h)$ in~$N$, otherwise~$(Q\cI h, Q\notcI
h)$, and in both cases already the node~$N$ would be contradictory.

The meeting of two lines is handled analog.

Now for the part where the vertex leading to~$N$ comes from a case
distinction. Assume that below the node following the negative vertex
there are no other case distinctions (otherwise we start the whole
procedure below this). So we can eliminate all the actions below the
node and only critical constellations are left. But since the negative
literal cannot induce a critical constellation they must already be
present in the predecessor of~$N$ and therefore the case distinction
can be eliminated.

For the last part, if there is a critical constellation above, we have
eliminated all the intermediate actions, that is what we wanted.
\endproof

Now we see the importance to know that there is a construction. Because if we
start with the procedure from above and end up in an open node, we know that
this node is contradictory or the critical constellation procedure produces a
subtree with all leaves closed. This now is independent from a particular
calculus.

Combining the above lemmas and methods we obtain

\begin{theorem}\label{theorem:1}
For any formula proven in any reasonable\footnote{i.e.\ a calculus comparable
to~\pelk} and sound calculus for \pg, there is a construction which deduces
this formula.
\end{theorem}



One interesting consequence of the way this theorem is proven is that
sketches as they are used are not constructive in the usual sense,
meaning that there are non-constructive elements in it. One of the
most obvious of these elements is the case distinction, which of
course is necessary. Another one is the occurrence of critical
constellations, which are incarnations of case distinctions.

This fact, that sketches aren't constructive in the strict sense of
the word, is in fact interesting, because at the first glance we would
accept them and even name them as primary candidates for constructive
objects. 


\subsection{An example}
\label{sec:herbrandsketch:example}

We want to give an example for such a sketch: The fact we want to
prove is the fact that~$\ao$ does not lie on the line through two
diagonal-points. That is with
\begin{align*}
  g &:= [\ao\bo]\\
  h &:= [\co\deo]\\
  j &:= [\ao\co]\\
  k &:= [\bo\deo]\\
  E &:= (gh)\\
  F &:= (jk)\\
  l &:= [EF]
\end{align*}
we want to proof that~$\ao\notcI l$. For the sketch compare
fig.~\ref{fig:diagonal}. 

\begin{figure}[ht!]
  \[\epsfbox{diagonal.7}\]
  \caption{Sketch of $\ao\notcI l$}
  \label{fig:diagonal} 
\end{figure}

The corresponding construction tree is given
in~fig.~\ref{fig:construction}. This example demonstrates the
automatic deduction of a contradiction with the help of critical
constellations. From node~1 to node~2 the various terms are
constructed and renamed, this is a trivial part and no critical
constellation will arise. But in node~2 we start proving~$\ao\notcI l$
by contradiction. We split the construction adding once~$A\cI l$
(node~3r) and once~$A\notcI l$ (node~3l). In node~3r a critical
constellation arises, which can be solved in two ways, either leading
to a contradiction in node~4l or to a new critical constellation in
node~4r. The solution to the new critical constellation produces
contradictions in both nodes~5l and~5r, which closes all the nodes
below the addition of~$A\cI l$ in node~2. The only open node left over
is node~3l, which proves~$A\notcI l$.

\begin{figure}[ht!]
  \[\epsfbox{diagonal.8}\]
  \caption{Construction tree for $\ao\notcI l$}
  \label{fig:construction}
\end{figure}


\subsection{Estimations of the length of the constructions}
\label{sec:herbrandsketch:estimations}

With the estimation of the length of terms in the Herbrand disjunction
given in Sec.~\ref{sec:herbrand:minimize:length} we can give an
estimation of the length of a construction as follows:

In every step of the construction of the terms and the
case-distinctions the branches can be split (doubled), which yields a
multiplicand of~$2^{d_m\#t+\#l}$. Furthermore in every final
(prefinal) node there can be a chain of critical constellations, but
not more than~$\#t$, where~$\#t$ is the number of terms, $\#l$ the number
of literals and~$d_m$ the maximal term depth. We thus obtain the
following (very rough) bound on the length~$l_\cC$ of a
construction~$\cC$ made from a \HD:
  $$ l_\cC \le 2^{d_m\#t+\#l}\cdot \#t.$$

%%% Local Variables: 
%%% mode: latex
%%% TeX-master: "paper"
%%% End: 
