%Quality Control Editor: We have provided a PDF that shows the tracked changes in your file as in a Word document. This method makes it easier for you to match the edited file with your original file and make any necessary edits to your file in your LaTeX program. Please let us know if you require further assistance.

%% ****** Start of file apstemplate.tex ****** %
%%
%%
%%   This file is part of the APS files in the REVTeX 4 distribution.
%%   Version 4.1 of REVTeX, October 2009
%%
%%
%%   Copyright (c) 2001, 2009 The American Physical Society.
%%
%%   See the REVTeX 4 README file for restrictions and more information.
%%
%
% This is a template for producing manuscripts for use with REVTEX 4.0
% Copy this file to another name and then work on that file.
% That way, you always have this original template file to use.
%
% Group addresses by affiliation; use superscriptaddress for long
% author lists, or if there are many overlapping affiliations.
% For Phys. Rev. appearance, change preprint to twocolumn.
% Choose pra, prb, prc, prd, pre, prl, prstab, prstper, or rmp for journal
%  Add 'draft' option to mark overfull boxes with black boxes
%  Add 'showpacs' option to make PACS codes appear
%  Add 'showkeys' option to make keywords appear
%\documentclass[amsmath,amssymb,aps,prb,twocolumn,groupedaddress]{revtex4-2}
\documentclass[amsmath,amssymb,aps,pra,preprint,groupedaddress]{revtex4-2}
%\documentclass[amsmath,amssymb,aps,pra,twocolumn,groupedaddress]{revtex4-1}
%\documentclass[aps,prl,preprint,superscriptaddress]{revtex4-1}
%\documentclass[amsmath,amssymb,aps,prl,twocolumn,groupedaddress]{revtex4-1}

% You should use BibTeX and apsrev.bst for references
% Choosing a journal automatically selects the correct APS
% BibTeX style file (bst file), so only uncomment the line
% below if necessary.
\bibliographystyle{apsrev4-2}
\pdfoutput=1

\usepackage{graphicx}% Include figure files
\usepackage{dcolumn}% Align table columns on decimal point
\usepackage{bm}% bold math
%\usepackage{longtable}
%\usepackage{hyperref}% add hypertext capabilities
\usepackage{ulem}
%\usepackage{ums}
\usepackage{xcolor,soul}

\usepackage{longtable}
%\usepackage{geometry}
%\setcounter{table}{-1}


\newcommand*\Laplace{\mathop{}\!\mathbin\bigtriangleup}
\newcommand*\DAlambert{\mathop{}\!\mathbin\Box}


\begin{document}

% Use the \preprint command to place your local institutional report
% number in the upper righthand corner of the title page in preprint mode.
% Multiple \preprint commands are allowed.
% Use the 'preprintnumbers' class option to override journal defaults
% to display numbers if necessary
%\preprint{}

%Title of paper
\title{Langevin and Navier-Stokes Simulations of 3D Protoplasmic Streaming and a Nontrivial Effect of Boundary Fluid Circulation}

% repeat the \author .. \affiliation  etc. as needed
% \email, \thanks, \homepage, \altaffiliation all apply to the current
% author. Explanatory text should go in the []'s, actual e-mail
% address or url should go in the {}'s for \email and \homepage.
% Please use the appropriate macro foreach each type of information

% \affiliation command applies to all authors since the last
% \affiliation command. The \affiliation command should follow the
% other information
% \affiliation can be followed by \email, \homepage, \thanks as well.
%\email[]{koibuchi@mech.ibaraki-ct.ac.jp}
%\homepage[]{Your web page}
%\thanks{}
%\altaffiliation{}

\author{Shuta Noro \textit{$^{1}$}}
\author{Satoshi Hongo \textit{$^{1}$}}
\author{Shinichiro Nagahiro \textit{$^{1}$}}
\author{Hisatoshi Ikai \textit{$^{1}$}}
\author{Hiroshi Koibuchi \textit{$^{2}$}}\email[]{koibuchi@ibaraki-ct.ac.jp; koibuchih@gmail.com}
\author{Madoka Nakayama \textit{$^{3}$}}
\author{Tetsuya Uchimoto \textit{$^{4}$}}
\author{Jean-Paul Rieu \textit{$^{5}$}}
\affiliation{
$^{1}$\quad~National Institute of Technology (KOSEN), Sendai College, 48 Nodayama, Medeshima-Shiote, Natori-shi, Miyagi 981-1239, Japan \\
$^{2}$\quad~National Institute of Technology (KOSEN), Ibaraki College, 866 Nakane, Hitachinaka, Ibaraki 312-8508, Japan \\
$^{3}$\quad~Research Center of Mathematics for Social Creativity, Research Institute for Electronic Science, Hokkaido University, Sapporo, Japan \\
$^{4}$\quad~Institute of Fluid Science (IFS), Tohoku University, 2-1-1 Katahira, Aoba-ku Sendai 980-8577, Japan \\
$^{5}$  Univ Lyon, Universit$\acute{e}$ Claude Bernard Lyon 1, CNRS UMR-5306, Institut Lumi$\grave{e}$re Mati$\grave{e}$re, F-69622, Villeurbanne, France
}

%Collaboration name if desired (requires use of superscriptaddress
%option in \documentclass). \noaffiliation is required (may also be
%used with the \author command).
%\collaboration can be followed by \email, \homepage, \thanks as well.
%\collaboration{}
%\noaffiliation{}

%\date{\today}

\begin{abstract}
In this paper, we report numerical results obtained by Langevin Navier-Sokes (LNS) simulations of the velocity distribution of three-dimensional (3D) protoplasmic streaming. Experimentally observed and reported peaks of the velocity distribution in plant cells, such as those of {Nitella flexilis}, were recently reproduced by our group with LNS simulations. However, these simulations are limited to Couette flow that is a simplified and two-dimensional (2D) phenomenon of protoplasmic streaming. Therefore, to reproduce the peaks in natural 3D flows, the simulations should be extended to three dimensions. This paper describes LNS simulations on 3D cylinders discretized by regular cubes in which fluid particles are activated by random Brownian force with strength $D$. We find that for finite $D$, the distribution of velocity $h(V), V\!=\!|\vec{V}|$ has two different peaks at $V\to 0$ and a finite $V$ and that the quantity $h(V_z)$ for $|V_z|$ along the longitudinal direction is also nontrivially affected by Brownian motion. Moreover, we study the effects of the circular motion of the boundary fluid on streaming and find that circular motion enhances the fluid velocity inside the plant cells.
\end{abstract}
%
% insert suggested PACS numbers in braces on next line
%\pacs{64.60.-i \sep 68.60.-p \sep 87.16.D-}
% insert suggested keywords - APS authors don't need to do this
%\keywords{}

%\maketitle must follow title, authors, abstract, \pacs, and \keywords
\maketitle

% body of paper here - Use proper section commands
% References should be done using the \cite, \ref, and \label commands
%\section{}
% Put \label in argument of \section for cross-referencing
%\section{\label{}}
%\subsection{}
%\subsubsection{}

%----------------------------------------------------------
\section{Introduction\label{intro}}
%----------------------------------------------------------
The size of plant cells, such as those of {Nitella flexilis}, in water ranges up to 1 $({\rm mm})$ in diameter, and the flows in cells, which are called protoplasmic streaming, have recently attracted considerable interest from researchers in biological and agricultural fields \cite{VLubics-Goldstein-Protoplasma2009,ShimmenYokota-COCB2004,TominagaIto-COPB2015}. Figures \ref{fig-1}(a) and (b) show an optical image of plants in water and illustrate the streaming and its direction inside a cell.
%%%%%%%%%%%%%%%%%%%%%%%%%%%%%%%%%%%%%%%%%%%%%%%%%%%%%%%%%%%%%%%%%%%%
\begin{figure}[h!]
\begin{center}
\includegraphics[width=11.5cm]{fig-1-crop.pdf}

\caption{ (a) A plant in water (image obtained using  an optical camera), (b) an illustration of boundary flows that cause the streaming inside cells, and the flow directions of a cell section at the boundary region. \label{fig-1}
}
\end{center}
\end{figure}
%%%%%%%%%%%%%%%%%%%%%%%%%%%%%%%%%%%%%%%%%%%%%%%%%%%%%%%%%%%%%%%%%%%%

It is well-known that the flows in plant cells are driven by a so-called molecular motor in which a myosin molecule moves along actin filaments. Hence, the mechanism of flow activation is the same  as that of animal cells \cite{Squires-Quake-RMP2005,McintoshOstap-CSatAG2016,Astumian-Science2020,Julicher-etal-RMP1997}. Recently, Tominaga et al. reported that the size of a plant depends on the streaming speed, implying that the velocity of myosin molecules determines plant size \cite{TominagaIto-COPB2015}. Therefore, protoplasmic streaming has been extensively studied by experimental and theoretical techniques, including fluid dynamical simulations \cite{Goldstein-etal-PRL2008,Goldstein-etal-PNAS2008,Goldstein-etal-JFM2010,Raymond-Goldstein-IF2015,Kikuchi-Mochizuki-PlosOne2015,Niwayama-etal-PNAS2010}.

Kamiya and Kuroda first measured the flow velocity $V_z({\rm \mu m/s})$ along the longitudinal direction of a cell using an optical microscope ($V_z$ for $\theta\!=\!0$ in Fig. \ref{fig-2}(a)) \cite{Kamiya-Kuroda-1956,Kamiya-Kuroda-1958}. The corresponding physical quantities, such as the kinematic viscosity, have also been reported \cite{Kamiya-Kuroda-1973,Kamiya-1986,Tazawa-pp1968}.
The flow direction on the side of the cell boundary is not always parallel to the longitudinal axis; rather, it is twisted, forming the so-called indifferent zone (Fig. \ref{fig-2}(a)).
Subsequently, using magnetic resonance velocimetry on whole points of a cylinder section, Goldstein et al. measured the velocity and reported the positional dependence of several different lines on the section \cite{Goldstein-etal-JFM2010}. They also theoretically studied the flow field by combining the Stokes equation or the Navier-Stokes (NS) equation for low Reynolds numbers and an advection-diffusion equation for a new variable concentration. Their results agreed well with the experimental results \cite{Goldstein-etal-PRL2008,Goldstein-etal-PNAS2008}.



%%%%%%%%%%%%%%%%%%%%%%%%%%%%%%%%%%%%%%%%%%%%%%%%%%%%%%%%%%%%%%%%%%%%
\begin{figure}[t]
\begin{center}
\includegraphics[width=11.5cm]{fig-2-crop.pdf}
\caption{ (a) Flow velocity ${\vec V}$ inside a cell and (b) scattered light intensity vs. Doppler shift frequency obtained by the laser-light scattering technique, where the peak at 93 Hz corresponds to a velocity of 72 $({\rm \mu m/s})$ \cite{Mustacich-Ware-BJ1976}. The dashed lines in (a) denote the positions for the numerical measurement of the velocity $V$. The angle $\theta$ is assumed to be $\theta\!=\!0^\circ,35^\circ, 75^\circ$ \cite{Goldstein-etal-JFM2010}. The solid line in (a) denotes the indifferent zone, where two opposite boundary velocities contact each other. }
 \label{fig-2}
\end{center}
\end{figure}
%%%%%%%%%%%%%%%%%%%%%%%%%%%%%%%%%%%%%%%%%%%%%%%%%%%%%%%%%%%%%%%%%%%%
Approximately 25 years after the Kamiya and Kuroda's measurements, Mustacich and Ware observed streaming by using the laser-light scattering technique \cite{Mustacich-Ware-PRL1974,Mustacich-Ware-BJ1976, Mustacich-Ware-BJ1977,Sattelle-Buchan-JCS1976}. They reported that the spectra of scattered light show two different peaks at $V\!\to\! 0$ and $V\!\not=\! 0$, where $V$ denotes the fluid velocity (Fig. \ref{fig-2}(b)). These two different velocities reflect streaming. The first peak at $V\!\to\! 0$ is caused by the Brownian motion of the particles \cite{Sattelle-Buchan-JCS1976}, and the second peak at $V\!\not=\! 0$ is due to the velocity of the particles dragged by the molecular motor on the periphery of the cell (Fig. \ref{fig-2}(a)).

Recently, the peaks were numerically reproduced by Refs. \cite{Egorov-etal-POF2020,Noro-etal-2021} by simplifying three-dimensional (3D) streaming to two-dimensional (2D) Couette flow and by using the Langevin Navier-Stokes (LNS) equation that includes random Brownian motion \cite{Hossain-etal-POF2022}. However, the 3D nature of streaming, such as the circulation on the cell boundary, is modified to be parallel to the longitudinal direction in 2D simulations. Therefore, 3D simulations are preferable for a better understanding of streaming. In this paper, we model streaming by a 3D LNS simulation technique for variable velocities and pressures on cylindrical lattices composed of regular cubes. The primary objectives of this study are to find two different peaks in the distribution of velocities corresponding to the experimentally observed velocities and to determine whether the circulation of the particles at the cell boundary has a nontrivial effect on streaming. We find that random Brownian motion plays an essential role in enhancing mixing.




%----------------------------------------------------------
\section{Methods\label{methods}}
%----------------------------------------------------------
%----------------------------------------------------------
\subsection{Langevin Navier--Stokes equation and the discrete equation\label{LNS-equation}}
%----------------------------------------------------------
The LNS equation is given by a set of coupled equations for the velocity $\vec{V}\!=\!(V_x,V_y,V_z)({\rm m/s})$ and pressure $p({\rm Pa})$ such that
\begin{eqnarray}
\label{NS-eq-org}
\begin{split}
&\frac{\partial {\vec V}}{\partial t}=-\left ({\vec V}\cdot \nabla\right){\vec V}-{\rho}^{-1} {\it \nabla} p +\nu \Laplace {\vec V} + {\vec \eta},\\
&\nabla\cdot {\vec V}=0,
\end{split}
\end{eqnarray}
where $\rho({\rm kg/m^3})$ and $\nu({\rm m^2/s})$ denote the fluid density and kinematic viscosity, respectively \cite{Egorov-etal-POF2020,Noro-etal-2021}. The final term $\vec{\eta}({\rm m/s^2})$ on the right-hand side of the first equation corresponds to the random Brownian force per unit density. Such a Langevin equation is used as a numerical technique in particle physics for functions on a lattice \cite{KGWilson-PRD1985,Ukawa-Fukugita-PRL1985,Hofler-Schwarzer-PRE2000}, and hence, we consider that Brownian motions can be combined with the NS equation for fluids. Thus, LNS simulations partly include a particle simulation scheme, as in the lattice Boltzmann simulation technique \cite{Succi-LBM2001,Ladd-PRL1993,Bhadauria-etal-POF2021,Fu-etal-POF2022}.

The variables velocity $\vec{V}$ and pressure $p$ are used in Eq. (\ref{NS-eq-org}), which is different from the LNS equation for the flow function $\psi$ and vorticity $\omega$ in Ref.~\cite{Egorov-etal-POF2020}, where the condition $\nabla\cdot {\vec V}\!=\!0$ is exactly satisfied for all $t$. By contrast, this divergence-less condition is not always satisfied in the time evolution of Eq. (\ref{NS-eq-org}) even if it is satisfied in the initial configuration. The original MAC method is a simple technique to resolve this problem \cite{McKee-etal-CandF2007}; however, $\nabla\cdot {\vec V}\!=\!0$ is not always satisfied even in the convergent solution satisfying ${\partial {\vec V}}/{\partial t}=0$, and hence, the simplified MAC (SMAC) method, which is a well-known revised MAC method is used in this paper. Here, we briefly introduce the SMAC technique.


To solve Eq. (\ref{NS-eq-org}), we impose the condition
\begin{eqnarray}
\label{SS-condition}
\frac{\partial {\vec V}}{\partial t}=0,
\end{eqnarray}
which is the steady-state condition. To obtain ${\vec V}$ satisfying this condition, we numerically solve the following discrete equation with the time step ${\it \Delta} t$
\begin{eqnarray}
\label{NS-eq-time-step}
{\vec V}(t+{\it \Delta}t)= \vec{V}(t) +{\it \Delta} t \left[ \left (-{\vec V}\cdot \nabla\right){\vec V}(t)-{\rho}^{-1} {\it \nabla} p(t+{\it \Delta} t) +\nu \Laplace {\vec V}(t)\right] +\sqrt{2D{\it \Delta} t} \,\vec{g},
\end{eqnarray}
where we use the same symbol $t$ for the discrete time in this equation as the real time $t$ in the original LNS equation in Eq. (\ref{NS-eq-org}). The relationship between the Gaussian random numbers $\vec{g}$ and Brownian force $\vec{\eta}$ is given by $\sqrt{2D{\it \Delta} t} \,\vec{g}\!=\!\vec{\eta}{\it \Delta} t$ \cite{Egorov-etal-POF2020}.
We should note that the discrete time $t$ in Eq. (\ref{NS-eq-time-step}) is introduced to obtain the steady-state solution satisfying Eq. (\ref{SS-condition}) and is different from the real time $t$ in Eq. (\ref{NS-eq-org}).
Thus, the Brownian random force $\vec{g}$ is incremented only when the convergent solution $\vec{V}$ of Eq. (\ref{NS-eq-time-step}) is obtained. The condition $\nabla\cdot {\vec V}\!=\!0$ is imposed only on the convergent configurations, and it is not always satisfied on $\vec{V}(t)$ in the time evolution process of Eq. (\ref{NS-eq-time-step}).
From Eq. (\ref{NS-eq-time-step}), we understand that $\nabla\cdot {\vec V}(t+{\it \Delta} t)\!=\!0$ is not always satisfied even if $\nabla\cdot {\vec V}(t)\!=\!0$ is satisfied because the terms independent of ${\vec V}(t)$ on the right-hand side are not always divergence-less. Moreover, the time evolution of $p(t)$ is not specified. Therefore, we introduce a temporal velocity $\vec{V}^*(t)$ and rewrite Eq. (\ref{NS-eq-time-step}) as follows:
\begin{eqnarray}
\label{NS-eq-time-step-temporal-1}
&&{\vec V}^*(t)= \vec{V}(t) +{\it \Delta} t \left[ \left (-{\vec V}\cdot \nabla\right){\vec V}(t)-{\rho}^{-1} {\it \nabla} p(t) +\nu \Laplace {\vec V}(t)\right] +\sqrt{2D{\it \Delta} t} \,\vec{g}(t),\\
\label{NS-eq-time-step-temporal-2}
&&{\vec V}(t+{\it \Delta} t)={\vec V}^*(t)-{\it \Delta} t {\rho}^{-1} {\it \nabla} \left[ p(t+{\it \Delta} t)-p(t)\right].
\end{eqnarray}


By applying the divergence operator $\nabla\cdot$ to Eq. (\ref{NS-eq-time-step-temporal-2}),
we obtain
\begin{eqnarray}
\label{SMac-eq-0}
\nabla\cdot{\vec V}(t+{\it \Delta} t)=\nabla\cdot{\vec V}^*(t)-{\it \Delta} t {\rho}^{-1} \Laplace \left[ p(t+{\it \Delta} t)-p(t)\right].
\end{eqnarray}
Then, assuming the condition $\nabla\cdot{\vec V}(t+{\it \Delta} t)\!=\!0$,
we obtain Poisson's equation for $\phi(t) \!=\! p(t+{\it \Delta} t)\!-\!p(t)$ such that
\begin{eqnarray}
\label{SMac-Poiss}
\Laplace \phi(t)=\frac{\rho}{{\it \Delta} t}\nabla\cdot{\vec V}^*(t),\quad \phi(t) = p(t+{\it \Delta} t)-p(t).
\end{eqnarray}
Thus, combining Eq. (\ref{NS-eq-time-step-temporal-1}) for the time evolution of ${\vec V}^*(t)$ with Poisson's equation in Eq. (\ref{SMac-Poiss}) for $\phi(t) \!=\! p(t+{\it \Delta} t)\!-\!p(t)$, we implicitly obtain the time evolution ${\vec V}(t+{\it \Delta} t)$ with the condition $\nabla\cdot{\vec V}(t+{\it \Delta} t)\!=\!0$. The time evolution of $p$ from $p(t)$ to $p(t\!+\!{\it \Delta} t)$ can also be obtained by adding the solution $\phi(t)$ to $p(t)$ such that $p(t)\!+\!\phi(t)$.

The simulation procedure can be summarized as follows:
\begin{enumerate}
\item[(i)] Calculate $V^*(t)$ by Eq. (\ref{NS-eq-time-step-temporal-1}) using the current $V(t)$, $p(t)$ and $\vec{g}$
\item[(ii)] Solve Poisson's equation for $\phi(t)$ in Eq. (\ref{SMac-Poiss})
\item[(iii)] Calculate $V(t+{\it \Delta} t)$ and $p(t+{\it \Delta} t)$ by Eq. (\ref{NS-eq-time-step-temporal-2}) and $p(t)\!+\!\phi(t)$, respectively
\item[(iv)] Repeat procedures (i)--(iii) until the convergence criteria given below are satisfied.
\end{enumerate}
This technique for updating ${\vec V} (t)$ is slightly different from that of the original MAC method, where ${\vec V} (t)$ is explicitly updated to ${\vec V}(t+{\it \Delta} t)$, and hence, $\nabla\cdot{\vec V}\!=\!0$ is not always satisfied and may be slightly violated even in the convergent solution. This violation becomes larger for larger Brownian force strength $D$ in the original MAC method; however, it is negligibly small in the convergent solutions in the SMAC method. Detailed information on $\nabla\cdot{\vec V}\!=\!0$ is given below.




We assume the following convergent criteria for $\vec{V}$ and $p$ such that
\begin{eqnarray}
\label{convergent-time-step}
\begin{split}
&{\rm Max}\left[\left||\nabla\cdot\vec{V}_{ijk}(t+{\it \Delta} t)|-|\nabla\cdot\vec{V}_{ijk}(t)|\right|\right]<1\times 10^{-8},\\
&{\rm Max}\left[|\vec{V}_{ijk}(t+{\it \Delta} t)-\vec{V}_{ijk}(t)|\right]<1\times 10^{-8},\\
&{\rm Max}\left[|p_{ijk}(t+{\it \Delta} t)-p_{ijk}(t)|\right]<1\times 10^{-8},
\end{split}
\end{eqnarray}
and the criterion for the iterations of the Poisson equation is
\begin{eqnarray}
\label{convergent-Poisson}
{\rm Max}\left[|\phi_{ijk}(n+1)-\phi_{ijk}(n)|\right]<1\times 10^{-10},
\end{eqnarray}
$n$ denotes the iteration step for solving the Poisson equation in Eq. (\ref{SMac-Poiss}).
We should note that the first condition in Eq. (\ref{convergent-time-step}) is satisfied in the early iterations, and therefore, this convergence condition is actually unnecessary. One additional point to note is that only the convergent solution satisfies $\nabla\cdot{\vec V}\!=\!0$. In this sense, the obtained numerical solution of the LNS equation in Eq. (\ref{NS-eq-org}) is a steady-state solution characterized by Eq. (\ref{SS-condition}) as mentioned above.


The most time-consuming part of this process is to solve the Poisson equation for $\phi$, which is simulated by the Open-Mp parallelization technique coded in Fortran. The mean value of the physical quantity $\langle Q\rangle$ is obtained by
\begin{eqnarray}
\label{mean-of-Q}
\langle Q\rangle=(1/ n_{\rm s})\sum_{i=1}^{n_{\rm s}}Q_i,
\end{eqnarray}
where $Q_i$ denotes the $i$-th convergent configuration corresponding to the $i$-th Gaussian random force $\eta_i$. The symbol $n_{\rm s}$ is the total number of convergent configurations and is $n_{\rm s}=1000$ for all $D$ except $D\!=\!0$. For $D\!=\!0$, $n_{\rm s}$ should be $n_{\rm s}\!=\!1$ because no Gaussian random number is assumed in this case. For simplicity, the brackets $\langle \cdot \rangle$ are not used for the mean values henceforth.





%%%%%%%%%%%%%%%%%%%%%%%%%%%%%%%%%%%%%%%%%%%%%%%%%%%%%%%%%%%%%%%%%%%%
\begin{figure}[ht]
\begin{center}
\includegraphics[width=10.5cm]{fig-3-crop.pdf}
\caption{ Illustration of divergence $({{\it \Delta} x})^2\nabla\cdot \vec{V}$ at lattice point 0, where the dimension is assumed to be $D\!=\!2$ for simplicity, and the lattice spacing ${\it \Delta} x$ is assumed in the 3D simulations on lattice A and lattice B in this paper.
 \label{fig-3} }
\end{center}
\end{figure}
%%%%%%%%%%%%%%%%%%%%%%%%%%%%%%%%%%%%%%%%%%%%%%%%%%%%%%%%%%%%%%%%%%%%


To check the accuracy of this technique, we calculate the lattice average ${\rm Div}_{\rm ab}$ of $|\nabla\cdot \vec{V}|$ such that ${\rm Div}_{\rm ab}\!=\!({\it \Delta} x)^3\sum_{ijk}|\nabla\cdot \vec{V}_{ijk}|/\sum_{ijk}1 ({\rm m^3/s})$, where $\sum_{ijk}1$ is the total number of the internal lattice points. This  ${\rm Div}_{\rm ab}$
is considered to be the lattice average of the fluid volume flowing into or out of a cubic lattice enclosing a lattice point per second according to the Gauss theorem (see Fig. \ref{fig-3} for the 2-dimensional case).
For $D\!=\!0$, we numerically obtain $\sum_{ijk}|\nabla\cdot \vec{V}_{ijk}|\!=\!0 ({\rm 1/\beta s})$ for every time step $t$ (see the Appendix for the simulation unit $\beta$), and hence, ${\rm Div}_{\rm ab}\!=\!0$ for all lattice points, implying that $\nabla\cdot \vec{V}\!=\!0$ on lattice B. We also have $\sum_{ijk}|\nabla\cdot \vec{V}_{ijk}|\!\simeq\! 7.9\times10^{-9}({\rm1/\beta s})\!=\! 1.9\times10^{-7}({\rm 1/s})$ on lattice A, which implies that the total divergence is given by $({\it \Delta} x)^3\sum_{ijk}|\nabla\cdot \vec{V}_{ijk}|\!\simeq\!4.7\times 10^{-5}({\rm \mu m^3/s})\!=\! 4.7\times 10^{-11}({\rm \mu g/s})$, where $1({\rm \mu m^3})$ is replaced by $10^{-6}({\rm \mu g})$ because the density $\rho$ is considered to be the same as that of water $\rho\!=\!10^3({\rm kg/m^3})\!=\!10^{-3}({\rm g/mm^3})\!=\!10^{-12}({\rm g/\mu m^3})$. This value of the total divergence $4.7\times 10^{-11}({\rm \mu g/s})$, implying ${\rm Div}_{\rm ab}\simeq 4\times 10^{-17}({\rm \mu g/s})$, is sufficiently small for the scales of protoplasmic streaming.

In the case of $D\!\not=\!0$, the total divergence fluctuates from one convergent configuration to another configuration depending on the Brownian random forces. However, its mean value is almost independent of $D$, and the maximum value is approximately given by $|\sum_{ijk}\nabla\cdot \vec{V}_{ijk}|\!=\! 1\times 10^{-7}({\rm 1/\beta s})$, which is independent of Lattices A and B, and this value is comparable with the above-mentioned value for $D\!=\!0$ on lattice A. Thus, we find that the SMAC method is successful for the divergence-less condition for simulating the 3D LNS equation in Eq. (\ref{NS-eq-org}) under the condition of Eq. (\ref{SS-condition}).

It is important to mention the unit change in the physical parameters in the simulations. This unit change is a dimensionless approach to fluid mechanics in the sense that the boundary velocity $V_e({\rm m/s})$ is fixed to $V_0\!=\!1$ in the simulation units, as discussed in detail below.


%----------------------------------------------------------
\subsection{Lattices for simulations and the boundary condition\label{lattices}}
%----------------------------------------------------------
%%%%%%%%%%%%%%%%%%%%%%%%%%%%%%%%%%%%%%%%%%%%%%%%%%%%%%%%%%%%%%%%%%%%
\begin{figure}[ht]
\begin{center}
\includegraphics[width=11.5cm]{fig-4-crop.pdf}
\caption{ 3D cylindrical computational domains for (a) lattice A and (b) lattice B for streaming. The arrows in (a) and (b) indicate the directions of the boundary velocity that activates the streaming inside the cylinder, and the small cones represent the velocity directions. The length $L$ of the cylinder is determined such that the indifferent zone rotates once around the boundary $\Gamma_3$; $L\!=\!2\pi(R+1)\tan\phi$ (the unit of the lattice spacing is $a(=\!1)$), and consequently, the velocities and pressures on $\Gamma_1$ and $\Gamma_2$ are connected by a periodic boundary condition on lattice A. $L$ is fixed to $L\!=\!2R$ on lattice B.
To clearly visualize the cones, the size or diameter of the cylinders in (a) and (b) is four times smaller than that used for the simulations. The velocity $\vec{V}$ and pressure $p$ are fixed to $|\vec{V}|\!=\!1$ and $p\!=\!0$ in the simulation units on $\Gamma_3$ as the boundary condition for both lattices A and B.
 \label{fig-4} }
\end{center}
\end{figure}
%%%%%%%%%%%%%%%%%%%%%%%%%%%%%%%%%%%%%%%%%%%%%%%%%%%%%%%%%%%%%%%%%%%%
%%%%%%%%%%%%%%%%%%%%%%%%%%%%%%%%%%%%%%%%%%%%%%%%%%%%%%%%%%%%%%%%%%%%
\begin{figure}[h!]
\begin{center}
\includegraphics[width=9.5cm]{fig-5-crop.pdf}
\caption{Section of a cylindrical lattice of size $R\!=\!8$. The total number of vertices is $N_S\!=\!241$, which includes $N_B\!=\!48$ boundary vertices. The radius $r$ of the boundary vertices is given by $R\!-\!1<r<R\!+\!1$, on which the velocity $\vec{V}$ and pressure $p$ are fixed, and the $r$ value of the internal vertices is given by $r<R\!-\!1$.
 \label{fig-5} }
\end{center}
\end{figure}
%%%%%%%%%%%%%%%%%%%%%%%%%%%%%%%%%%%%%%%%%%%%%%%%%%%%%%%%%%%%%%%%%%%%
We show the details of the lattice construction for a cylindrical domain for streaming in plant cells. The actual cell surface is soft and is expected to bend and fluctuate. However, it is relatively rigid compared with the surface of animal cells, and therefore, we assume that the cylinder surface is rigid for simplicity. Thus, for the computational domain, we assume a 3D cylinder of radius $R$ and length $L$ (Fig. \ref{fig-4}). The unit length is fixed to the lattice spacing $a$, which is assumed to be $a\!=\!1$ in this and the next subsections, and therefore, $Ra$ and $La$ are written as $R$ and $L$ for simplicity. The lattice spacing $a$ is restored as ${\it \Delta} x$ with the physical or simulation unit in the following section.
The regions indicated by the symbols $\Gamma_i, (i=1,3)$ in the figure denote the boundary surfaces.


The fluid is activated by the molecular motors on the surface $\Gamma_3$, and the fluid particles are dragged on the boundary, as indicated by two large arrows in Fig. \ref{fig-4}. The contact line in which two different velocities coexist is called the indifferent zone, and divides $\Gamma_3$ into two domains. The angle of the zone is fixed to $\pi/3$ (or $60^\circ$), and the volume of the computational domain depends on this angle. Therefore, the angle of $60^\circ$ is assumed to be smaller than the actual angle in plant cells to save computational time. The length $L$ of the cylinder is fixed such that the indifferent zone rotates once around $\Gamma_3$. Therefore, the boundaries $\Gamma_1$ and $\Gamma_2$ are connected by a periodic boundary condition such that the velocities $\vec V$ and pressures $p$ on $\Gamma_1$ and $\Gamma_2$ are nearest neighbors to each other.
The boundary velocity $\vec{V} (|\vec{V}|\!=\!1)$ on $\Gamma_3$ on lattice A is fixed to be a unit tangential vector, and the orientation in one domain is opposite to that in the other, as shown in Fig. \ref{fig-4}(a). The tangential vectors are characterized by $|V_z|\!=\!\cos \phi (\phi\!=\!60^\circ)$, where $V_z$ is the $z$-component of $\vec{V}$. On lattice B, the boundary velocity is fixed to $|V_z|\!=\!1$.
Another boundary condition is $p\!=\!0$ at all points on $\Gamma_3$. From the computational viewpoint, $p$ should be fixed somewhere in the computational domain or boundary, and since no difference is expected in the points of $\Gamma_3$, we impose this condition on $p$.


To explain the lattice structure, we show a surface section of the cylinder (Fig. \ref{fig-5}). The building block is a regular cube with a side length $a(=\!1)$ for the lattice spacing. Therefore, the boundary shape is not a circle. Let $r$ be the distance of a vertex from the center of the section. The vertices at the region $R\!-\!1<r<R\!+\!1$ form the boundary, whereas $r<R\!-\!1$ are the internal points, where $R$ is the radius of the horizontal and vertical lines passing through the center of the section. In Fig. \ref{fig-5}, $R$ is given by $R\!=\!8a$, which is 4 times smaller than that of the lattice in Fig. \ref{fig-6} used for the simulations.


In Table \ref{table-1}, we summarize the geometries of lattices A and B. On the surface of lattice A, the angle $\phi$ of the indifferent zone to the vertical direction is $\phi\!=\!60^\circ$, while it is assumed to be $\phi\!=\!90^\circ$ on lattice B to examine the effects of the rotation of velocity on the boundary $\Gamma_3$ (Figs. \ref{fig-4}(a) and (b)).


%++++++++++++++++++++++++++++++++++
\begin{table}[htb]
\caption{Two different lattice geometries for the simulations.  The ratio $L/R$ is approximately $L/R\!\simeq\!11.2$ in lattice A, while it is exactly $L/R\!=\!2$ in lattice B, where $L$ is the cylinder length and $R$ is the radius with the unit of lattice spacing $a(=\!1)$ (Figs. \ref{fig-4}(a) and (b)). }
\label{table-1}
\begin{center}
 \begin{tabular}{ccccccccccc   }
 \hline
\vspace{-3mm}
lattice & $\phi$ &  $R$  && $L$ &&  internal & &  boundary  \\
        &   &     &&   &&  vertices & &  vertices \\
 \hline
A & $60^\circ$ & 32 && 358 && 1153800  && 66240  \\
B & $90^\circ$ & 40 && 80  && 406053    && 18468  \\ 
 \hline
\end{tabular}
\end{center}
\end{table}
%++++++++++++++++++++++++++++++++++

%----------------------------------------------------------
%\subsection
\subsection{
Normalized histograms of velocities $V$ and $|V_z|$ \label{calculation}}


%----------------------------------------------------------
%%%%%%%%%%%%%%%%%%%%%%%%%%%%%%%%%%%%%%%%%%%%%%%%%%%%%%%%%%%%%%%%%%%%
\begin{figure}[ht]
\begin{center}
\includegraphics[width=8.5cm]{fig-6-crop.pdf}
\caption{ Lattice points on the lines of three different angles $\theta$ for the calculation of $\vec{V}$. The line at $\theta\!=\!0^\circ$ is the vertical line along the $x$-axis, while those at $\theta\!=\!35^\circ$ and $\theta\!=\!75^\circ$ are composed of two symmetric lines due to the reflection symmetry $\theta\to-\theta$. The lattice section corresponds to lattice A for which the radius is given by $R\!=\!32$ with the unit of lattice spacing $a(=\!1)$.
 \label{fig-6} }
\end{center}
\end{figure}
%%%%%%%%%%%%%%%%%%%%%%%%%%%%%%%%%%%%%%%%%%%%%%%%%%%%%%%%%%%%%%%%%%%%
Velocity $\vec{V}$ is numerically measured at the lines shown in Figs. \ref{fig-2}(a) and \ref{fig-6} with the angle $\theta\!=\!0^\circ,35^\circ,75^\circ$ on the surface section at $z\!=\!L/2$ in the middle point of the cylinder. These angles are the same as those assumed in Ref. \cite{Goldstein-etal-JFM2010}.
Only a single section is used for the numerical measurements because the boundary velocities are circulating, and there is no equivalent section along the longitudinal direction of the cylinder, even though the sections at $z\!=\!0$ and $z\!=\!L$ are almost equivalent due to the periodicity of the circulation (Fig. \ref{fig-4}(a)). In the case of lattice B, the whole surface sections are equivalent; however, we also use the section at $z\!=\!L/2$ to numerically measure $\vec{V}$, as in the case of lattice A.





%%%%%%%%%%%%%%%%%%%%%%%%%%%%%%%%%%%%%%%%%%%%%%%%%%%%%%%%%%%%%%%%%%%%
\begin{figure}[h!]
\begin{center}
\includegraphics[width=9.5cm]{fig-7-crop.pdf}
\caption{Histogram $h(V)$ of velocity $V$ that is numerically calculated along the lines with angle $\theta$ on the surface section at $z\!=\!L/2$ shown in Fig. \ref{fig-6} and ranges in $0\leq\!V\!\leq\!V^{\rm max}$. Small ${\it \Delta}V$ is fixed to ${\it \Delta}V\!=\!r V_0/N$ by using the boundary velocity $V_0(=\!1 {\rm\;  in \; simulation \; unit)}$ and the numbers $r$ and $N$, where $N\!=\!16$ for all $V$, and $r$ is fixed $r\!=\!4$ for the $V$ values obtained under $D\!=\!1000$ and $r\!=\!2$ for all other $V$ values. The histogram $h(V_i)$ is obtained by counting the total number of lattice points along the line with angle $\theta$ (Fig. \ref{fig-6}) at which the velocity $V$ satisfies the condition $ V_i\!\leq\!V\!<\!V_{i+1}$. For data plotting, both $h(V)$ and $V^{\rm max}$ are normalized to 1.
 \label{fig-7} }
\end{center}
\end{figure}
%%%%%%%%%%%%%%%%%%%%%%%%%%%%%%%%%%%%%%%%%%%%%%%%%%%%%%%%%%%%%%%%%%%%
We should note that the experimentally observed intensity of laser-light scattering plotted in Fig. \ref{fig-2}(b) is considered to correspond to a distribution of velocity $h(V)$ for nearly the entire section because the spot of the laser light is not small compared with the sectional area (approximately $0.5 [{\rm mm}]$ diameter). Therefore, the intensity in Fig. \ref{fig-2}(b) does not always correspond to $h(V(\theta))$ of a specific $\theta$. However, recent experiments have revealed that more detailed information on the velocity distribution can be obtained \cite{Goldstein-etal-JFM2010}, and we show the details of the calculation technique of $h(V(\theta))$ (Fig. \ref{fig-7}). As described in Section \ref{calculation}, $V(\theta)$ is obtained at the lines with $\theta$ (Fig. \ref{fig-6}) on the section at $z\!=\!L/2$, where $V(\theta)$ denotes $V(\theta)\!=\!|\vec{V}(\theta)|$. The histogram $h(V_z(\theta))$ for $|V_z(\theta)|$ is obtained by the same technique, and hence, we only show the case for $h(V(\theta))$. Many samples of $V(\theta)$ are obtained for $D\!\not=\!0$ due to the many convergent configurations of the LNS simulations. By contrast, only a single convergent configuration can be used to calculate $h(V(\theta))$ in the case of $D\!=\!0$. Therefore,  the curves for $h(V)$ and $h(V_z)$ presented in Fig. \ref{fig-8} are not always smooth.

We use the symbol $V$ for $V(\theta)$ for simplicity.
Let $V^{\rm max}$ be the maximum velocity of the samples of all three values of $\theta$.
A small interval of velocity ${\it \Delta}V$ is defined by ${\it \Delta}V\!=\!rV_0/N$, where $V_0\!=\!1$ is the boundary velocity, and the multiplication factor $r$ is fixed to $r\!=\!2$ for $0\!\leq\!D\!<\!300$, $r\!=\!3$ for $D\!=\!300$ and $r\!=\!4$ for $D\!=\!1000$ for lattice A, and $r\!=\!2$ for $0\!\leq\!D\!<\!1000$ and $r\!=\!3$ for $D\!=\!1000$ for lattice B. The number $N$ is fixed to $N\!=\!16$ and $N\!=\!20$ for lattices A and B, respectively. Using this ${\it \Delta}V$, the histogram $h(V_i)$ for $V_i\!=\!i {\it \Delta}V$ is calculated by counting the number of $V$ satisfying $V_i\!\leq\!V\!<\!V_{i+1}$, where $V_{i+1}\!=\!V_i\!+\!{\it \Delta}V$.





Here, we show the details of the normalization of the distributions or histograms $h(V_z)$ of $|V_z|$ and $h(V)$ of $V$ measured on these lines. The maximum speed $|V_z|^{\rm max}$ is obtained from the velocities for $\theta\!=\!0^\circ,35^\circ,75^\circ$ and normalized to $|V_z|^{\rm max}\!=\!1$. The maximum velocity $V^{\rm max}$ is also obtained from those for $\theta\!=\!0^\circ,35^\circ,75^\circ$ and normalized to $V^{\rm max}\!=\!1$. These $|V_z|^{\rm max}$ and $V^{\rm max}$ values correspond to the boundary velocity $\vec{V}_0$ if $D\!=\!0$; however, this correspondence is not always true for $D\!>\!0$ due to the random Brownian motion.

The distributions $h(V_z)$ and $h(V)$ are also normalized such that $0\!\leq\!h(V_z)\!\leq\!1$ and $0\!\leq\!h(V)\!\leq\!1$ by obtaining $h(V_z)^{\rm max}$ and $h(V)^{\rm max}$, respectively. These $h(V_z)^{\rm max}$ and $h(V)^{\rm max}$ values are calculated for each $\theta\!=\!0^\circ,35^\circ,75^\circ$ in contrast to $|V_z|^{\rm max}$ and $V^{\rm max}$.

%-------------------------------
\subsection{Input parameters}
%-------------------------------

The physical parameters characterizing protoplasmic streaming are the density $\rho_e ({\rm kg/m^3})$, kinematic viscosity $\nu_e ({\rm m^2/s})$, boundary velocity $V_e ({\rm m/s})$, and diameter of the cell $d_e ({\rm m})$, which are given in Table \ref{table-2}.
%++++++++++++++++++++++++++++++++++
\begin{table}[htb]
\caption{Physical parameters $\nu_{e}, V_{e}, d_{e}$ corresponding to the protoplasmic streaming in plant cells, expressed in physical units.   }
\label{table-2}
\begin{center}
 \begin{tabular}{ccccccc}
 \hline
 $\rho_e ({\rm kg/m^3})$ && $\nu_{e}~({\rm  m^2}/{\rm  s})$ && $V_{e}~({\rm \mu m}/{\rm s})$ && $d_{e}~({\rm \mu m})$  \\
 \hline
 $1\times 10^{3}$       &&  $1\times 10^{-4}$       && $50$    &&   $500$     \\
\hline
\end{tabular}
\end{center}
\end{table}
%++++++++++++++++++++++++++++++++++
These values are given in Refs. \cite{Kamiya-Kuroda-1956,Kamiya-Kuroda-1958,Kamiya-Kuroda-1973,Kamiya-1986,Tazawa-pp1968} and are the same as those assumed in the 2D LNS simulations in Refs. \cite{Egorov-etal-POF2020,Noro-etal-2021}.


Using the factors $\alpha, \beta$ and $\lambda$ for the unit change (see Appendix \ref{append-A}), we obtain the parameters in Table \ref{table-3} in the simulation units. These are used in the simulations in this paper.
%++++++++++++++++++++++++++++++++++
\begin{table}[htb]
\caption{Parameters assumed in the simulations;  these values are given in the simulation units. The lattice spacing ${\it \Delta}x_0(=\!d_e/n_X)$ is given by ${\it \Delta}x_0\!=\!7.8125 ({\rm \mu m})$ (Lattice A: $n_X\!=\!64$) and ${\it \Delta}x_0\!=\!6.25 ({\rm \mu m})$ (Lattice B: $n_X\!=\!80$) in the physical unit, where $d_e\!=\!500 ({\rm \mu m})$. \label{table-3} }
\begin{center}
 \begin{tabular}{cccccccccccc   }
 \hline
 Lattice && $\rho_0~[\frac{\rm \lambda kg}{\rm (\alpha m)^3}]$ && $\nu_0~[\frac{\rm (\alpha m)^2}{\rm \beta s}]$ &&\ $V_0~[\frac{\rm \alpha m}{\rm \beta s}]$ &&  ${\it \Delta}x_0~[{\rm \alpha m}]$    &&   ${\it \Delta}t_0~[{\rm \beta s}]$  \\
 \hline
 A && $1\times 10^{-3}$       && $1\times 10^6$       &&  1   &&   3.90625  && $5\times10^{-7}$  \\
B && $1\times 10^{-3}$       && $1\times 10^6$       &&  1   &&   3.125   && $5\times10^{-7}$  \\
 \hline
\end{tabular}
\end{center}
\end{table}
%++++++++++++++++++++++++++++++++++
The strength $D({\rm m^2/s^3})$ of the Brownian motion is varied in the simulations because this is necessary to observe the dependence of physical quantities on $D$ \cite{Egorov-etal-POF2020,Noro-etal-2021}.
Here, we comment on the implication of the $D$-dependence of the velocity distributions for the physics of fluids in plant cells. As discussed in Refs. \cite{Egorov-etal-POF2020,Noro-etal-2021}, a variation of $D$ is equivalent to that of physical parameters such as the kinematic viscosity $\nu_e$, boundary velocity $V_e$ and diameter $d_e$. Many combinations of these parameter changes can yield a change in $D$. However, it should be emphasized that the complicated parameter dependence of the velocity distributions $h(V_z)$ and $h(V)$ is summarized by a simple $D$-dependence.


The results depend on the physical parameters $\nu_{e}, V_{e}, d_{e}$. This problem is discussed in Ref. \cite{Noro-etal-2021} for the 2D LNS equation of velocity and pressure, and this discussion is also applicable to the 3D LNS equations in this paper. Therefore, we do not discuss this problem in detail; however,  Ref. \cite{Noro-etal-2021} indicates that if the simulation results obtained by the parameters in Table \ref{table-3} are consistent with the experimental data of a normalized velocity distribution, then any experimental data can be simulated by changing only $\nu_0$ and $D$. The question is whether this assumption is correct. To consider this problem, we focus on the interesting property in the experimentally observed velocity distribution; namely, two different peaks are observed, and both are expected to be influenced by the Brownian motion of fluid particles.


%----------------------------------------------------------
\section{Numerical results \label{numerical-results}}
%----------------------------------------------------------
%----------------------------------------------------------
\subsection{Velocity distribution for $D\!=\!0$ \label{D_zero}}
%----------------------------------------------------------
First, we show the results for $D\!=\!0$ on lattice A. The distributions or histograms $h(V_z)$ of $|V_z|$ and $h(V)$ of $V$ are plotted in Figs. \ref{fig-8}(a) and (b), where $|V_z|$ and $V$ are normalized as described in the preceding subsection. The plotted data are calculated from a single convergent configuration of $\vec{V}$ and at a small number of data points as described above. Therefore, small fluctuations are observed in the data. We find that the shape of $h(V_z)$ vs. $|V_z|$ is nearly the same as that of $h(V)$ vs. $V$. A peak can be observed in both $h(V_z)$ and $h(V)$ at $V\to 0$ for $\theta\!=\!75^\circ$; however, this peak does not correspond to the Brownian motion of the fluid particles because no Gaussian random force is assumed. For the same reason, no peak is observed in either $h(V_z)$ or $h(V)$ at finite velocities for any $\theta$.
The mean value of all samples of $\theta\!=\!0^\circ,35^\circ75^\circ$ is also plotted by the (\textcolor{green}{${\bullet}$}) symbols denoted by ``all''. These data are smooth because the total number of samples is relatively large.
%%%%%%%%%%%%%%%%%%%%%%%%%%%%%%%%%%%%%%%%%%%%%%%%%%%%%%%%%%%%%%%%%%%%
\begin{figure}[ht]
\begin{center}
\includegraphics[width=12.5cm]{fig-8-crop.pdf}
\caption{ Normalized velocity distribution (a) $h(V_z)$ vs. $|V_z|$, (b) $h(V)$ vs. $V$, and the position dependence (c) $V_z(r)$ vs. $r$ obtained on lattice A, and (d), (e), (f) those obtained on lattice B for $D\!=\!0$. The data denoted by the (\textcolor{green}{${\bullet}$}) symbols with the letters ``all'' are the mean values of all samples obtained at $\theta\!=\!0^\circ,35^\circ75^\circ$.
No clear difference is observed between the values obtained on lattices A and B.
 \label{fig-8} }
\end{center}
\end{figure}
%%%%%%%%%%%%%%%%%%%%%%%%%%%%%%%%%%%%%%%%%%%%%%%%%%%%%%%%%%%%%%%%%%%%
Fig. \ref{fig-8}(c) shows the dependence of $V_z$ on the distance $r$ from the center of the surface section.

The results obtained on lattice B are shown in Figs. \ref{fig-8}(d)--(f). Both $h(V_z)$ and $h(V)$ for $\theta\!=\!0^\circ, 35^\circ$ are almost flat or independent of $|V_z|$ and $V$, respectively, implying that the flow fields on these lines of the cylindrical section resemble that of Couette flow without Brownian motion. We find no significant difference between the results for lattices A and B for $D\!=\!0$, indicatingthat the flow field inside the cylinder is almost independent of the fluid circulation on the boundary, at least for $D\!=\!0$ zero Brownian motion.

The distributions $h(V_z)$ and $h(V)$ in Figs. \ref{fig-8}(d) and (e) are almost the same because the boundary velocity is along the $z$ direction on lattice B. By contrast, they are slightly different from each other in Figs. \ref{fig-8}(a) and (b) on lattice A, as expected from the rotating boundary velocity around the cylinder. The position dependences of $V_z$ in Fig. \ref{fig-8}(c) are close to those in Fig. \ref{fig-8}(f) and are almost the same as those reported in Ref. \cite{Goldstein-etal-JFM2010}.


%----------------------------------------------------------
\subsection{Velocity distribution for $D\!\not=\!0$ \label{D_nonzero}}
%----------------------------------------------------------
%%%%%%%%%%%%%%%%%%%%%%%%%%%%%%%%%%%%%%%%%%%%%%%%%%%%%%%%%%%%%%%%%%%%
\begin{figure}[ht]
\begin{center}
\includegraphics[width=12.5cm]{fig-9-crop.pdf}
\caption{ (a) $h(V_z)$ vs. $|V_z|$, (b) $h(V)$ vs. $V$, and (c) $V_z(r)$ vs. $r$ obtained on lattice A, and (d), (e), and (f) those obtained on lattice B. $D$ is fixed to $D\!=\!1$. The data denoted by the (\textcolor{green}{${\bullet}$}) symbosl with the letters ``all'' are obtained by using all samples of $\theta\!=\!0^\circ,35^\circ75^\circ$.
 \label{fig-9} }
\end{center}
\end{figure}
%%%%%%%%%%%%%%%%%%%%%%%%%%%%%%%%%%%%%%%%%%%%%%%%%%%%%%%%%%%%%%%%%%%%
The results corresponding to $D\!=\!1$, which are nonzero, finite and small, are plotted in Figs. \ref{fig-9}(a)--(c) as obtained on lattice A and in Figs. \ref{fig-9}(d)--(f) for lattice B. We find that $h(V_z)$ in Fig. \ref{fig-9}(a) is close to $h(V_z)$ at $D\!=\!0$ plotted in Fig. \ref{fig-8}(a); however, a nontrivial difference can be observed for $h(V)$ plotted in Fig. \ref{fig-9}(b) obtained on lattice A and $h(V)$ in Fig. \ref{fig-9}(e) on lattice B. Indeed, $h(V)$ drops to $h(V)\!\to\! 0$ at $V\!\simeq\! 0.9$ in Fig. \ref{fig-9}(b); in other words, fluid particles of velocity $V^{\rm max}\!>\!V_0^{\rm max}$ exist on lattice A, implying that the Brownian motion changes the flow field so that it deviates from Couette flow even though $D$ is sufficiently small. By contrast, $h(V_z)$ and $h(V)$ plotted in Figs. \ref{fig-9}(d) and (e) remain unchanged from those in Figs. \ref{fig-8}(d) and (e) obtained at $D\!=\!0$ on lattice B, implying that the nontrivial deviation from the Couette flow appears only on lattice A.
The position dependence of $V_z$ is almost independent of lattices A and B, as shown in Figs. \ref{fig-9}(c) and (f).

%%%%%%%%%%%%%%%%%%%%%%%%%%%%%%%%%%%%%%%%%%%%%%%%%%%%%%%%%%%%%%%%%%%%
\begin{figure}[ht]
\begin{center}
\includegraphics[width=12.5cm]{fig-A0-crop.pdf}
\caption{(a) $h(V_z)$ vs. $|V_z|$, (b) $h(V)$ vs. $V$, and (c) $V_z(r)$ vs. $r$ obtained on lattice A, and (d), (e), and (f) those obtained on lattice B. $D$ is fixed to $D\!=\!20$. The data denoted by the (\textcolor{green}{${\bullet}$}) symbol with the letters ``all'' are obtained by using all samples of $\theta\!=\!0^\circ,35^\circ75^\circ$.
 \label{fig-10} }
\end{center}
\end{figure}
%%%%%%%%%%%%%%%%%%%%%%%%%%%%%%%%%%%%%%%%%%%%%%%%%%%%%%%%%%%%%%%%%%%%
Now, we plot the results corresponding to $D\!=\!10$ obtained on lattices A and B in Figs. \ref{fig-10}(a)--(c) and Figs. \ref{fig-10}(d)--(f), respectively. For a relatively large $D(=\!10)$, we also find a nontrivial difference in $h(V)$ plotted in Figs. \ref{fig-10}(b) and (e), where two vertical dashed lines are drawn at the two peaks. The peaks at smaller $V$ represent the Brownian motion of fluid particles \cite{Sattelle-Buchan-JCS1976}, and those at larger $V$ correspond to the boundary velocity. The positions $V$ of the peaks on lattice A are the same as those on lattice B. However, all curves $h(V)$ in Fig. \ref{fig-10}(b) have a peak at the smaller velocity position, while $h(V)$ for $\theta\!=\!75$ in Fig. \ref{fig-10}(e) has no peak in the smaller velocity region, indicating that Brownian motion is not always reflected in $h(V)$ on lattice B. The fact that Brownian motion is reflected in small velocity fluid particles for all $\theta$ on lattice An indicates that the rotating boundary velocity effectively increases the strength of the Brownian motion $D$. Recalling the formula $D\sim T/(\mu \tau_e^2)$ in Ref. \cite{Egorov-etal-POF2020}, where $T$, $\mu$ and $\tau_e$ are the temperature, viscosity and macroscopic relaxation time \cite{Coffey-Kalmykov-CP1993,Feldmanetal-Wiley2006}, respectively, we conclude that the rotation of the velocity effectively lowers the fluid viscosity.

We also note that fluid particles appear for which $V^{\rm max}$ and $|V_z|^{\rm max}$ are larger than the boundary velocities $V_0$ and $V_{0z}$ corresponding to the peak in the large velocity region. This appearance of large velocity particles is a nontrivial effect of random Brownian motion. The position dependence of $V_z$ is still almost the same for lattices A and B even for $D\!=\!10$, as plotted in Figs. \ref{fig-10}(c) and (f), and it is also independent of $D$ because no difference can be found between the  $V_z$ in Figs. \ref{fig-10}(c) and (f) and those in Figs. \ref{fig-8}(c) and (f) and \ref{fig-9}(c) and (f).

%%%%%%%%%%%%%%%%%%%%%%%%%%%%%%%%%%%%%%%%%%%%%%%%%%%%%%%%%%%%%%%%%%%%
\begin{figure}[ht]
\begin{center}
\includegraphics[width=12.5cm]{fig-A1-crop.pdf}
\caption {$h(V)$ vs. $V$ for (a) $D\!=\!20$, (b) $D\!=\!100$, (c) $D\!=\!1000$ obtained on lattice A, and   $h(V)$ vs. $V$ for (d) $D\!=\!20$, (e) $D\!=\!100$, (f) $D\!=\!1000$ obtained on lattice B.
 \label{fig-11} }
\end{center}
\end{figure}
%%%%%%%%%%%%%%%%%%%%%%%%%%%%%%%%%%%%%%%%%%%%%%%%%%%%%%%%%%%%%%%%%%%%
To see the dependence on $D$, we further increase $D$ to $D\!=\!20$, $D\!=\!100$ and $D\!=\!1000$ and plot $h(V)$ in Figs. \ref{fig-11}(a)--(f). The effect of Brownian motion is now clearly visible even for $\theta\!=\!75^\circ$ on lattice B (Fig. \ref{fig-11}(d)); a peak appears for the small velocity region if $D$ is increased to $D\!=\!20$. If $D$ is increased to $D\!=\!100$, we find a nontrivial difference between Figs. \ref{fig-11}(b) and (e) in that the peak positions corresponding to the boundary velocity $\vec{V}_0$ in the large velocity region are different from each other. The fact that the position of the peak is located at a lower velocity on lattice A than on lattice B means that $V^{\rm max}$ is larger on lattice A than on lattice B, again implying that the circular velocity on the cylindrical surface effectively increases $D$ on lattice A. For a further increment of $D$, the second peak corresponding to the boundary velocity is expected to move left because $V^{\rm max}$ becomes increasingly large. Thus, for sufficiently large $D$, the two different peaks will merge, as shown in Figs. \ref{fig-11}(c) and (f), where $D\!=\!1000$. In such a case of a sufficiently large $D$, Brownian motion is dominant for the activation force of the fluid particles, and the boundary velocity is effectively neglected. As a consequence, the velocity distribution $h(V)$ shows ideal gas behavior \cite{Egorov-etal-POF2020}.

Finally, in this subsection, we show the maximum velocities $V^{\rm max}$ and $V_z^{\rm max}$ obtained on lattices A and B in Fig. \ref{fig-12}(a) and (b), where $V_z^{\rm max}$ denotes the absolute maximum of the $z$ component of $\vec{V}$. These data are calculated on the lines for $\theta\!=\!0^\circ,35^\circ,75^\circ$ on the cylinder section at $z\!=\!L/2$. We find from Fig. \ref{fig-12}(a) that $V^{\rm max}$ of lattice A becomes larger than that of lattice B at $D\!\geq\!20$ even though these values are almost the same for sufficiently small $D$.
%%%%%%%%%%%%%%%%%%%%%%%%%%%%%%%%%%%%%%%%%%%%%%%%%%%%%%%%%%%%%%%%%%%%
\begin{figure}[ht]
\begin{center}
\includegraphics[width=10.5cm]{fig-A2-crop.pdf}
\caption{Maximum velocities (a) $V^{\rm max}$ vs. $D$ and (b) $V_z^{\rm max}$ vs. $D$ obtained on lattices A and B. The difference in $V^{\rm max}$ starts to appear at $D\!\simeq\!20$, and the magnitude relation $V_z^{\rm max}(B)>V_z^{\rm max}(A)>$ is reversed to $V_z^{\rm max}(B)<V_z^{\rm max}(A)$ at $D\!\simeq\!100$. The numerical data are expressed in the simulation units, and $V^{\rm max}\!=\!1$ and $V_z^{\rm max}\!=\!1$ correspond to $50 ({\rm \mu m/s})$.
 \label{fig-12} }
\end{center}
\end{figure}
%%%%%%%%%%%%%%%%%%%%%%%%%%%%%%%%%%%%%%%%%%%%%%%%%%%%%%%%%%%%%%%%%%%%
In Fig. \ref{fig-12}(b), we find that $V_z^{\rm max}(B)>V_z^{\rm max}(A)$ for a sufficiently small $D$ region because the boundary velocity rotates on lattice A while it is along the $z$ direction on lattice B. This magnitude relation is reversed at approximately $D\!=\!100$. Thus, we confirm that the velocity of the fluid particles is increased by random Brownian motion, indicating that the mixing of biological materials is enhanced by Brownian motion with the help of nonparallel velocity circulation.
Quantitatively, an increment of $D$ from $D\!=\!10$ to $D\!=\!50$, for example, increases $V^{\rm max}$ by approximately 25 $\%$ on lattice A. Because $D$ is the strength of Brownian motion, this increment of $V^{\rm max}$ is considered to be an effect of the local Brownian motion of the fluid particles. Moreover, $D\propto T$, where $T$ is the temperature\cite{Egorov-etal-POF2020}, and hence, this increment can be caused by an increment in the temperature.



%----------------------------------------------------------
\subsection{Snapshots of velocity and pressure \label{velocity-pressure}}
%----------------------------------------------------------
%%%%%%%%%%%%%%%%%%%%%%%%%%%%%%%%%%%%%%%%%%%%%%%%%%%%%%%%%%%%%%%%%%%%
\begin{figure}[ht]
\begin{center}
\includegraphics[width=16.0cm]{fig-A3-crop.pdf}
\caption{Snapshots of velocity and pressure obtained on lattice A at the section in the middle of cylinder $z\!=\!L/2(=\!159)$. Velocities corresponding to (a) $D\!=\!0$, (b) $D\!=\!20$, (c) $D\!=\!100$, (d) $D\!=\!1000$, and velocity with pressure corresponding to (e) $D\!=\!0$, (f) $D\!=\!20$, (g) $D\!=\!100$, (h) $D\!=\!1000$. Small red cones represent the velocity; the same velocity is shown in both the upper and lower rows, and the pressure $p$ is shown only in the lower row. Only velocities at every other vertex are shown.
 \label{fig-13} }
\end{center}
\end{figure}
%%%%%%%%%%%%%%%%%%%%%%%%%%%%%%%%%%%%%%%%%%%%%%%%%%%%%%%%%%%%%%%%%%%%
We show snapshots of the velocity and pressure obtained on the section of the cylinder at $z\!=\!L/2$ of lattice A for several different $D$ values. The direction of the boundary velocity at $z\!=\!L/2$ is the opposite to that on the boundaries $\Gamma_1$ at $z\!=\!0$ and $\Gamma_2$ at $z\!=\!L$ (Fig. \ref{fig-4}(a)). The velocities denoted by the cones in Figs. \ref{fig-13}(a)--(d) are of convergent configurations corresponding to (a) $D\!=\!0$, (b) $D\!=\!20$, (c) $D\!=\!100$ and (d) $D\!=\!1000$. For clear visualization, only the velocities at every other vertex are shown. The same velocities with pressures are shown in Figs. \ref{fig-13}(e)--(h), where the pressures $p$ are normalized to $0\!\leq\!p\!\leq\!1$ and represented by the color gradation. In this normalization, the boundary pressure $p\!=\!0$ changes to $p\!\simeq\!0.5$. Because the $z$ component of the velocity is negative $V_z\!<\!0$ in the lower part of the section in Figs. \ref{fig-13}(e)--(h), the velocities are hidden behind the sectional surfaces for the pressure visualization.

An examination of Figs. \ref{fig-13}(a) and (e) shows that the fluid flows regularly according to the boundary velocity and that the pressure $p$ remains almost unchanged from the boundary pressure $p(\simeq\!0.5)$ at $D\!=\!0$. The velocity and pressure are confirmed to be disturbed at nonzero $D$, and the disturbance becomes stronger as $D$ increases.

We note that the pressure at higher $D$ values does not always vary smoothly but rather is randomly distributed on the surface section. This condition of the pressure configuration is relatively close to that of Couette flow in parallel plates obtained by 2D LNS simulations \cite{Noro-etal-2021}.

To visualize the difference in the velocity configuration $\vec{V}$ between lattices A and B, we show snapshots of velocity $\vec{V}$ and pressure $p$ obtained on lattice B in Figs. \ref{fig-14}(a)--(h). The velocities in Figs. \ref{fig-14}(a)--(d) are shown at every third vertex. Because $p\simeq\!0.5$ at every point on the section for $D\!=\!0$ in the convergent configuration, we show snapshots of $\vec{V}$ and $p$ for $D\!=\!1$ instead of those for $D\!=\!0$ in Figs. \ref{fig-14}(a) and (e).
%%%%%%%%%%%%%%%%%%%%%%%%%%%%%%%%%%%%%%%%%%%%%%%%%%%%%%%%%%%%%%%%%%%%
\begin{figure}[ht]
\begin{center}
\includegraphics[width=16.0cm]{fig-A4-crop.pdf}
\caption{Snapshots of velocity and pressure obtained on lattice B at the section of the middle of cylinder $z\!=\!L/2(=\!40)$. Velocity corresponding to (a) $D\!=\!1$, (b) $D\!=\!20$, (c) $D\!=\!100$, (d) $D\!=\!1000$, and velocity with pressure corresponding to (e) $D\!=\!1$, (f) $D\!=\!20$, (g) $D\!=\!100$, (h) $D\!=\!1000$. Small red cones represent the velocity; the same velocity is shown in both the upper and lower rows, and the pressure $p$ is shown only in the lower row. Only velocities at every third vertex are shown.
 \label{fig-14} }
\end{center}
\end{figure}
%%%%%%%%%%%%%%%%%%%%%%%%%%%%%%%%%%%%%%%%%%%%%%%%%%%%%%%%%%%%%%%%%%%%
The velocity configurations on lattice B are clearly different from those on lattice A for all $D$, as expected from the difference in the boundary condition for $\vec{V}$.


%----------------------------------------------------------
\section{Summary and conclusion\label{summary}}
%----------------------------------------------------------
In this paper, we numerically study the velocity distribution of protoplasmic streaming in plant cells by using the LNS simulations on 3D cylinders discretized by regular cubic lattices. The goal of our study is to determine whether experimentally observed and reported peaks in the velocity distribution can be reproduced by LNS simulations. Additionally, we are interested in whether the circular motion of fluid over the surfaces of cells has nontrivial effects on the flows inside the cells.

The velocity distribution $h(V_z)$ of $|V_z|$ along the longitudinal direction of the cylinder and $h(V)$ of the magnitude of velocity $V(=\!|\vec{V}|)$ are calculated on two different lattices A and B. These $|V_z|$, $V$ and their distributions are obtained for the angle $\theta$ from the vertical direction of $\theta\!=\!0^\circ, 35^\circ, 75^\circ$. The distributions $h(V_z)$ and $h(V)$ as the mean values of all $\theta$ are also calculated.
The boundary velocity rotates on the cylindrical surface on lattice A, while the boundary velocity of lattice B is parallel to the longitudinal direction. The strength of random Brownian motion $D$ is varied in the range $0\!\leq\!D\!\leq\!1000$ as an input parameter.


We find two different peaks in $h(V)$ at two different velocities $V_1$ and $V_2$ ($V_1<V_2$) when $D$ is increased to $10\!\leq\!D\!\leq 100$ or greater, where $V_2$ corresponds to the boundary velocity. The Brownian motion of fluid particles is reflected in the emergence of the first peak in $h(V)$ at $V_1$ and the second peak of $h(V)$ at $V_2$ such that the curve $h(V)$ has a tail at $V>V_2$. This appearance of fluid particles at a velocity $V>V_2$ higher than the boundary velocity $V_2$ is also a nontrivial effect of Brownian motion, and is expected to play a nontrivial role in enhancing mixing.

For the effect of the nonparallel circulation of the boundary velocity that is known to be activated by the so-called molecular motor, we find that circular motion increases the maximum velocity speed. This velocity increment implies that the circular motion enhances the mixing of biological materials. Importantly, the enhancement of mixing appears only in the presence of the Brownian motion of the fluid particles.



%----------------------------------------------------------
\section{Data Availability Statement}
%----------------------------------------------------------
The data supporting the findings of this study are available from the corresponding author upon reasonable request.

%----------------------------------------------------------
\acknowledgements
%----------------------------------------------------------
H.K. acknowledges Fumitake Kato for discussions.
This work was supported in part by a Collaborative Research Project J20Ly09 of the Institute of Fluid Science (IFS), Tohoku University, and in part by a Collaborative Research Project of the National Institute of Technology (KOSEN), Sendai College. Numerical simulations were performed on the Supercomputer system "AFI-NITY" at the Advanced Fluid Information Research Center, Institute of Fluid Science, Tohoku University.

\appendix

%-------------------------------




%-------------------------------
\section{Physical units and simulation units\label{append-A}}
%-------------------------------
In the simulations, the physical units $({\rm m, s, kg})$ are changed to simulation units $({\rm \alpha m, \beta s, \lambda kg})$ using positive numbers $\alpha, \beta$ and $\lambda$. Using these numbers for $V_e, \nu_e,\rho_e$, we have the relations $V_e [{\rm m/s}]\!=\!V_e \beta/\alpha [{\rm \alpha m/(\beta s)}]$, $\nu_e [{\rm m^2/s}]\!=\!\nu_e \beta/\alpha^2 [{\rm (\alpha m)^2/(\beta s)}]$, and $\rho_e [{\rm kg/m^3}]\!=\!\rho_e \alpha^3/\lambda [{\rm \lambda kg/(\alpha m)^3}]$ in physical units. The right-hand sides of these relations can be written as $V_0 [{\rm \alpha m/(\beta s)}]$, $\nu_0 [{\rm (\alpha m)^2/(\beta s)}]$, and $\rho_0 [{\rm \lambda kg/(\alpha m)^3}]$ in simulation units. Therefore,
\begin{eqnarray}
\label{numbers-for-units}
\alpha=\frac{\nu_e}{\nu_0}\frac{V_0}{V_e}, \quad \beta=\frac{\nu_e}{\nu_0}\left(\frac{V_0}{V_e}\right)^2, \quad \lambda=\frac{\rho_e}{\rho_0}\left(\frac{\nu_e}{\nu_0}\frac{V_0}{V_e}\right)^3
\end{eqnarray}

In addition to these numbers, we need positive numbers $\gamma, \delta$ for the lattice and time discretization such that $n_X\to \gamma n_X$ and $n_T\to \delta n_T$ for the physical quantities to be independent of $n_X$ and $n_T$. In this expression, $n_X$ and $n_T$ are connected to the lattice spacing ${\it \Delta}x ({\rm m})$ and the discrete time step ${\it \Delta}t ({\rm s})$ such that ${\it \Delta}x ({\rm m})\!=\!d_e/n_X$ and ${\it \Delta}t ({\rm s})\!=\!\tau_e/n_T$, respectively, where $\tau_e$ is the relaxation time \cite{Coffey-Kalmykov-CP1993,Feldmanetal-Wiley2006}. In this paper, we do not provide details on this problem for $n_X$ and $n_T$, and $\gamma, \delta$ are fixed to $\gamma\!=\!1, \delta\!=\!1$. We assume the values given in Table \ref{table-B1} for the unit change.
%++++++++++++++++++++++++++++++++++
\begin{table}[htb]
\caption{Values for the change of physical units and simulation units.  \label{table-B1}}
\begin{center}
 \begin{tabular}{cccccccccccc   }
 \hline
$\alpha$ && $\beta$ && $\lambda$ && $\gamma$ && $\delta$  \\  \hline
 $2 \times10^{-6}$  && $4\times 10^{-2}$ && $8\times10^{-12}$ && $1$ && 1  \\ \hline
\end{tabular}
\end{center}
\end{table}
%++++++++++++++++++++++++++++++++++
Using the assumed numbers $\alpha$, $\beta$, and $\gamma$ in Table \ref{table-B1}, we have
$V_0\!=\!V_e \beta/\alpha\!=\!(50\!\times\!10^{-6})(4 \!\times\!10^{-2})/(2\!\times\!10^{-6})\!=\!1 [{\rm \alpha m/(\beta s)}]$, $\nu_0\!=\!\nu_e \beta/\alpha^2\!=\!(1\!\times\!10^{-4})(4 \!\times\!10^{-2})/(2\!\times\!10^{-6})^2 \!=\!1\!\times\!10^{-6}[{\rm (\alpha m)^2/(\beta s)}]$, and $\rho_0\!=\!\rho_e \alpha^3/\lambda\!=\!(1 \!\times\!10^{3})(2\!\times\!10^{-6})^3/(8\!\times\!10^{-12})\!=\! 1 \!\times\!10^{-3}[{\rm \lambda kg/(\alpha m)^3}]$.

The lattice spacing ${\it \Delta}x_0$ in the simulation unit is given by ${\it \Delta}x_0\!=\!\alpha^{-1}\frac{d_e}{n_X}\!=\!(2\!\times\!10^{-6})^{-1}(500\!\times\!10^{-6})/(2R)$, where the diameter $2R$ of the cylinder is assumed to be $n_X$, which is the total number of discretizations introduced in the case of a regular square lattice of size $L\!\times\! L$ with $L\!=\!n_X{\it \Delta}x_0$. For $n_X\!=\!2R\!=\!64$ on lattice A ($n_X\!=\!2R\!=\!80$ on lattice B), we have ${\it \Delta}x_0\!=\!3.90625$ (${\it \Delta}x_0\!=\!3.125$).

The discrete time step ${\it \Delta}t_0$ can also be expressed by ${\it \Delta}t_0\!=\!\beta^{-1}\frac{\tau_e}{n_T}$ using macroscopic relaxation time $\tau_e$ and the total number of time discretizations $n_T$. However, $\tau_e$ is not always given, and therefore, we simply assume ${\it \Delta}t_0\!=\!5\!\times\!10^{-7}$ for the simulation unit.




\section*{References}
\begin{thebibliography}{0}


\bibitem{VLubics-Goldstein-Protoplasma2009}
J. Verchot-Lubicz, and R.E.Goldstein, 
{\it Cytoplasmic streaming enables the distribution of molecules and vesicles in large plant cells}, 
Protoplasma, {\bf 240}, 99-107.(2009); DOI 10.1007/s00709-009-0088-x

\bibitem{ShimmenYokota-COCB2004}
T. Shimmen and T. Yokota, {\it Cytoplasmic streaming in plants}, Current Opinion in Cell Biology, {\bf 16 (1)}, 68--72 (2004); 
https://doi.org/10.1016/j.ceb.2003.11.009


\bibitem{TominagaIto-COPB2015}
M. Tominaga and K. Ito, {\it The molecular mechanism and physiological role of cytoplasmic streaming}, Current Opinion in Plant Biology, {\bf 27}, 104--110 (2015); https://doi.org/10.1016/j.pbi.2015.06.017

% molecular motor
\bibitem{Squires-Quake-RMP2005}
T.M. Squires and S.R. Quake,
{\it Microfluidics: Fluid physics at the nanoliter scale.},
Rev. Mod. Phys.{\bf 77}, 977--1026  (2005);
https://doi.org/10.1103/RevModPhys.77.977

\bibitem{McintoshOstap-CSatAG2016}
B.B. McIntosh and E.M. Ostap, 
{\it Myosin-I molecular motors at a glance}, 
Cell Science at a Glance, {\bf 129}, 2689--2695 (2016); 
https://doi:10.1242/jcs.186403

\bibitem{Astumian-Science2020}
R.D. Astumian,
{\it Thermodynamics and Kinetics of a Brownian Motor},
Science, {\bf 276}, 917--922 (2020); 
http://science.sciencemag.org/content/276/5314/917

\bibitem{Julicher-etal-RMP1997}
F. J${\ddot{\rm u}}$licher, A. Ajdari and J. Prost,
{\it Modeling molecular motors},
Rev. Mod. Phys. {\bf 69(4)}, 1269--1282 (1997);
https://link.aps.org/doi/10.1103/RevModPhys.69.1269






% ------- streaming theoretical
\bibitem{Goldstein-etal-PRL2008}
J-W. Meent, I. Tuvalk and R.E. Goldstein, {\it Nature's Microfluidic Transporter: Rotational Cytoplasmic Streaming at High P$\acute {\rm e}$clet Numbers}, Phys. Rev. Lett. {\bf 101}, 178102(1-4) (2008); DOI: 10.1103/PhysRevLett.101.178102

\bibitem{Goldstein-etal-PNAS2008}
R.E. Goldstein, I. Tuvalk and J-W. van de Meent, {\it Microfluidics of cytoplasmic streaming and its implications for intracellular transport},
PNAS, {\bf 105}, 3663-3667 (2008); 
https://www.pnas.org cgi doi 10.1073 pnas.0707223105

% ------- streaming experimentally
\bibitem{Goldstein-etal-JFM2010}
 J-W. van De Meent,  A.J. Sederman, L.F. Gladden and  R.E. Goldstein, {\it Measurement of cytoplasmic streaming in single plant cells by magnetic resonance velocimetry}, J. Fluid Mech. {\bf 642}, pp.5-14 (2010); doi:10.1017/S0022112009992187

\bibitem{Raymond-Goldstein-IF2015}
R.E. Goldstein and J-W. van de Meent, {\it Physical perspective on cytoplasmic streaming}, 
 Interface Focus. {\bf 5}: 20150030; 
 https://doi.org/10.1098/rsfs.2015.0030 

% ------- streaming experimentally
\bibitem{Kikuchi-Mochizuki-PlosOne2015}
K. Kikuchi and O. Mochizuki, {\it Diffusive Promotion by Velocity Gradient of Cytoplasmic Streaming (CPS) in Nitella Internodal Cells},
 Plos One, 0144938(1-12)  (2015); 
https://DOI:10.1371/journal.pone.0144938  

% ------- streaming numerically
\bibitem{Niwayama-etal-PNAS2010}
R. Niwayama, K. Shinohara and A. Kimura, {\it Hydrodynamic property of the cytoplasm is sufficient to mediate cytoplasmic streaming in the Caenorhabiditis elegans embryo}, 
PNAS, vol. {\bf 108}, pp.11900-11905  (2011);
https://doi.org/10.1073/pnas.1101853108


% ------- protoplasmic streaming
\bibitem{Kamiya-Kuroda-1956}
N. Kamiya and K. Kuroda, {\it Velocity Distribution of the Protoplasmic Streaming in Nitella Cells}, 
 Bot, Mag. Tokyo, {\bf 69}, 544-554 (1956);
  https://doi.org/10.15281/jplantres1887.69.544,

\bibitem{Kamiya-Kuroda-1958}
N. Kamiya and K. Kuroda, {\it Measurement of the Motive Force of the Protoplasmic Rotation in Nitella}, 
Protoplasma, {\bf 50}, 144-147 (1958). 

\bibitem{Kamiya-Kuroda-1973}
N. Kamiya and K. Kuroda, {\it Dynamics of Cytoplasmic Streaming in a Plant Cell}, 
Biorheology, {\bf 10}, 179-187 (1973). 

\bibitem{Kamiya-1986}
N. Kamiya, {\it Cytoplasmic streaming in giant algal cells: A historical survey of experimental approaches},
Bot, Mag. Tokyo, {\bf 99}, 441-496 (1986); 
https://doi.org/10.1007/BF02488723  

\bibitem{Tazawa-pp1968}
M. Tazawa, {\it Motive force of the cytoplasmic streaming in Nitella}, Protoplasma, {\bf 65}, 207-222 (1968).


 
% ------- physics experimental velocity distribution
\bibitem{Mustacich-Ware-PRL1974}
R.V. Mustacich and B.R. Ware, {\it Observation of Protoplasmic Streaming by Laser-Light Scattering}, 
 Phys. Rev. Lett. {\bf 33}, 617-620 (1974). 

\bibitem{Mustacich-Ware-BJ1976}
R.V. Mustacich and B.R. Ware, {\it A Study of Protoplasmic Streaming in Nitella by Laser Doppler spectroscopy}, 
 Boiophys. J. {\bf 16}, 373-388 (1976). 
 
\bibitem{Mustacich-Ware-BJ1977}
R.V. Mustacich and B.R. Ware, {\it Velocity Distributions of the Streaming Protoplasm in Nittella Flexilis}, 
 Boiophys. J. {\bf 17}, 229-241 (1977). 

\bibitem{Sattelle-Buchan-JCS1976}
D.B. Sattelle and P.B. Buchan, {\it Cytoplasmic Streaming in Chara Corallina studied by Laser Light Scattering}, 
 J. Cell. Sci. {\bf 22}, 633-643 (1976). 




% ------- Langevin NS for the streaming
\bibitem{Egorov-etal-POF2020}
V. Egorov O. Maksimova, I. Andreeva, H.Koibuchi, S. Hongo, S. Nagahiro, H. Ikai, M. Nakayama, S. Noro, T. Uchimoto and J-P. Rieu, {\it Stochastic fluid dynamics simulations of the velocity distribution in protoplasmic streaming},
Phys. Fluids, {\bf 32}, 121902(1-15) (2020);  
https://doi.org/10.1063/5.0019225

\bibitem{Noro-etal-2021}
S. Noro,  H. Koibuchi, S. Hongo, S. Nagahiro, H. Ikai, M. Nakayama,  T. Uchimoto and J-P. Rieu, 
{\it Langevin Navier-Stokes simulation of  protoplasmic streaming by 2D MAC method},
ArXiv 2112.10901.



% ------- Langevin sim.

\bibitem{Hossain-etal-POF2022}
M. T. Hossain, I. D. Gates and G. Natale, {\it Dynamics of Brownian Janus rods at a liquid-liquid interface}, Phys. Fluids 34, 012117 (2022); https://doi.org/10.1063/5.0076148.



\bibitem{KGWilson-PRD1985}
G.G. Batrouni, G.R. Katz, A.S. Kronfeld, G.P. Lepage, B.Svetitsky and K.G. Wilson, {\it Langevin simulations of lattice field theories}, Phys. Rev. D {\bf 32} 2736-2747 (1985).

\bibitem{Ukawa-Fukugita-PRL1985}
A. Ukawa and M. Fukugita, {\it Langevin Simulation Including Dynamical Quark Loops}, Phys. Rev. Lett. {\bf 55}, 1854-1857 (1985).

\bibitem{Hofler-Schwarzer-PRE2000}
K. H${\rm\ddot o}$fler and S. Schwarzer, {\it Navier-Stokes simulation with constraint forces: Finite-difference method for particle-laden flows and complex geometries}, Phys. Rev. E. {\bf 61}, 7146-7160 (2000).


%------------------------------ Lattice Boltzmann
\bibitem{Succi-LBM2001}
S. Succi, {\it The Lattice Boltzmann Equation: For Fluid Dynamics and Beyond (Numerical Mathematics and Scientific Computation)}, (Clarendon Press, Oxford, 2001).

\bibitem{Ladd-PRL1993}
A.J.C. Ladd, {\it Short-Time Motion of Colloidal Particles: Numerical Simulation via a Fluctuating
Lattice-Boltzmann Equation}, Phys. Rev. Lett. {\bf 70}, 1339-1342 (1993).


\bibitem{Bhadauria-etal-POF2021}
A. Bhadauria, B. Dorschner, and I. Karlin,  {\it Lattice Boltzmann method for fluid-structure interaction in compressible flow}, Phys. Fluids 33, 106111 (2021); https://doi.org/10.1063/5.0062117.

\bibitem{Fu-etal-POF2022}
X. Fu, J. Sun, and Y. Ba,  {\it Numerical study of droplet thermocapillary migration behavior on wettability-confined tracks using a three-dimensional color-gradient lattice Boltzmann model}, Phys. Fluids 34, 012119 (2022); https://doi.org/10.1063/5.0078345.





% ------- MAC
\bibitem{McKee-etal-CandF2007}
 McKee S,  Tom$\acute{\rm e}$ MF, Ferreira VG,  Cuminato JA, Castelo A, Sousa FS, Mangiavacchi N.
\newblock{{T}he {MAC} method}.
\newblock Computers $\&$ Fluids 2008;37: 907?930. 
doi:10.1016/j.compfluid.2007.10.006






% ------- macroscopic relaxation time
\bibitem{Coffey-Kalmykov-CP1993}
 W.T. Coffey and Yu.P. Kalmykov, {\it On the calculation of the macroscopic relaxation time from the Langevin equation for a dipole in a cavity in a dielectric medium}, Chemical Physics {\bf 169}, pp.165-172 (1993); https://doi.org/10.1016/0301-0104(93)80074-J

\bibitem{Feldmanetal-Wiley2006}
Y. Feldman, A. Puenko and Y. Ryabov, {\it Dielectric Relaxation Phenomena in Complex Materials}, in {\it Fractals, Diffusion, and Relaxation in Disordered Complex Syatems}, Eds.  W.T.Coffey and Y.P.Kalmykov,  Advanced Chemical Physics Vol.133, Wiley-Interscience, New Jersey, 2006.

\end{thebibliography}
\end{document}

%
% ****** End of file apstemplate.tex ******


