\documentclass{article}%
\usepackage{amsfonts}
\usepackage{amsmath}
\usepackage{hyperref}
\usepackage{amssymb}
\usepackage{graphicx}%
\setcounter{MaxMatrixCols}{30}
%TCIDATA{OutputFilter=latex2.dll}
%TCIDATA{Version=5.50.0.2960}
%TCIDATA{CSTFile=40 LaTeX article.cst}
%TCIDATA{Created=Tuesday, December 15, 2020 07:44:37}
%TCIDATA{LastRevised=Sunday, June 05, 2022 23:59:27}
%TCIDATA{<META NAME="GraphicsSave" CONTENT="32">}
%TCIDATA{<META NAME="SaveForMode" CONTENT="1">}
%TCIDATA{BibliographyScheme=Manual}
%TCIDATA{<META NAME="DocumentShell" CONTENT="Standard LaTeX\Blank - Standard LaTeX Article">}
%TCIDATA{ComputeDefs=
%$\rho$
%$r$
%$W\left(  z,n\right)  =\int_{0}^{z}\sqrt{1+4n^{2}\frac{s^{4n-2}}{\left(
%s^{2n}+1\right)  ^{4}}}ds$
%$R\left(  z,n\right)  =\frac{z^{2n}}{1+z^{2n}}$
%$W_{2}\left(  r,\rho\right)  =\int_{0}^{r}\sqrt{1+\rho^{2}u^{2}e^{-2u}}du$
%$m\left(  x\right)  =\frac{ax+b}{cx+d}$
%$\chi\left(  w\right)  $
%$\psi\left(  r\right)  $
%$u\left(  l,x\right)  $
%$u_{6}\left(  l,x\right)  =\frac{\left(  2l+3\right)  \left(  2l+5\right)
%}{4x^{2}+4}-\frac{5}{4\left(  x^{2}+1\right)  ^{2}}$
%$u_{3}\left(  l,x\right)  =l\left(  l+1\right)  \frac{1}{\left(
%1+x^{2}\right)  }+\frac{1}{\left(  1+x^{2}\right)  ^{2}}$
%$u_{10}\left(  l,x\right)  =\left(  2l+7\right)  \frac{2l+9}{4\left(
%w^{2}+1\right)  }-\frac{45}{4\left(  w^{2}+1\right)  ^{2}}$
%$u_{26}\left(  l,x\right)  =\allowbreak\left(  2l+23\right)  \frac
%{2l+25}{4\left(  w^{2}+1\right)  }-\frac{525}{4\left(  w^{2}+1\right)  ^{2}}$
%$s\left(  N,x\right)  =\frac{1}{4}\left(  N-1\right)  \frac{Nx^{2}-3x^{2}%
%+2}{\left(  x^{2}+1\right)  ^{2}}/N^{2}$
%$P\left(  x,n\right)  =\left(  1+x^{n}\right)  ^{1/n}$
%$I\left(  n\right)  =\frac{1}{n}\int_{0}^{1}\left(  1+y\right)  ^{-\frac{2}%
%{n}}y^{\frac{3}{n}-1}\,dy$
%$\delta\left(  n\right)  =\frac{1}{n}\operatorname{Beta}\left(  \frac{3}%
%{n},\frac{2n-3}{n}\right)  -\frac{1}{3}~\Gamma\left(  2-\frac{3}{n}\right)
%\Gamma\left(  1+\frac{3}{n}\right)  $
%$\omega\left(  w\right)  =\left(  1/w^{2}\right)  ^{p/2}$
%}
\pdfoutput=1
\textwidth 6.5 in
\textheight 9 in
\voffset -0.75 in
\hoffset -0.875 in
\newtheorem{theorem}{Theorem}
\newtheorem{acknowledgement}[theorem]{Acknowledgement}
\newtheorem{algorithm}[theorem]{Algorithm}
\newtheorem{axiom}[theorem]{Axiom}
\newtheorem{case}[theorem]{Case}
\newtheorem{claim}[theorem]{Claim}
\newtheorem{conclusion}[theorem]{Conclusion}
\newtheorem{condition}[theorem]{Condition}
\newtheorem{conjecture}[theorem]{Conjecture}
\newtheorem{corollary}[theorem]{Corollary}
\newtheorem{criterion}[theorem]{Criterion}
\newtheorem{definition}[theorem]{Definition}
\newtheorem{example}[theorem]{Example}
\newtheorem{exercise}[theorem]{Exercise}
\newtheorem{lemma}[theorem]{Lemma}
\newtheorem{notation}[theorem]{Notation}
\newtheorem{problem}[theorem]{Problem}
\newtheorem{proposition}[theorem]{Proposition}
\newtheorem{remark}[theorem]{Remark}
\newtheorem{solution}[theorem]{Solution}
\newtheorem{summary}[theorem]{Summary}
\newenvironment{proof}[1][Proof]{\noindent\textbf{#1.} }{\ \rule{0.5em}{0.5em}}
\begin{document}

\title{Potentials versus Geometry, Revisited\thanks{Not quite the same as
\textit{\href{https://en.wikipedia.org/wiki/Ford_v_Ferrari}{Ford v Ferrari}}.
Or is it?}}
\author{T. Curtright$^{\S }$ and S. Subedi$^{\infty}\medskip$\\Department of Physics, University of Miami, Coral Gables, Florida 33124\\$^{\S }${\footnotesize curtright@miami.edu\ \ \ \ \ }$^{\infty}$%
{\footnotesize sushil.subedi04@gmail.com}}
\date{}
\maketitle

\begin{abstract}
We revisit an old subject to discuss relationships between the dynamics for
particles subjected to potentials and the dynamics for particles moving freely
on background geometries, in the context of non-relativistic quantum
mechanics. \ In particular, we illustrate how selected geometries can be used
to regularize singular potentials. \ We also compute scattering amplitudes for
quanta incident on a static non-relativistic wormhole.

\end{abstract}
\tableofcontents

\section{Introduction}

Given a solution to Einstein's theory of gravity acting as a background
spacetime geometry, it is well-known that particle motion on this fixed
geometry can be described by an effective potential \cite{MTW}. \ But given a
potential, it is perhaps less widely recognized that an equivalent particle
dynamics can be described by an effective geometry. \ This equivalence is
discussed here in the context of non-relativistic quantum mechanics. \ That is
to say, only the non-relativistic form of Schr\"{o}dinger's equation is used
in our discussion, and only spatial dimensions are allowed non-trivial
geometry, while time is taken to be common to all frames and universal to all observers.

Geometrical models can be related to potential models in ways that often make
the physics easier to extract and understand from the perspective of one or
the other side of the relationship. \ For a generic potential the
corresponding geometry can be singular, especially if the potential is
singular (e.g. inverse powers of $r$). \ On the other hand, non-singular
geometries can give regularized forms of singular potentials that specify the
essential physical features of those potentials without mathematical
ambiguities or pathologies.

\section{Non-relativistic Theory}

Consider a non-relativistic particle of mass $\mu$\ with energy $E=\hbar
^{2}\varepsilon/\left(  2\mu\right)  $, subject to a potential $V\left(
\overrightarrow{r}\right)  =\hbar^{2}U\left(  \overrightarrow{r}\right)
/\left(  2\mu\right)  $, and whose wave function is a solution of the
Schr\"{o}dinger equation in flat Euclidean space.
\begin{equation}
-\nabla^{2}\Psi+U\Psi=\varepsilon\Psi\label{PotentialSE}%
\end{equation}
Equivalence between this potential problem and a suitably chosen non-flat
geometry can be obtained by relating the solutions of (\ref{PotentialSE}) with
those for \emph{free} particle motion on a specified manifold, written as
\begin{equation}
-\frac{1}{\sqrt{g}}\partial_{\mu}\left(  \sqrt{g}g^{\mu\nu}\partial_{\nu}%
\Psi\right)  =\varepsilon\Psi\label{GeometricSE}%
\end{equation}
In particular, given a rotationally invariant potential, $U\left(  r\right)
$, for a particle with specified angular momentum there is an equivalent
radial Schr\"{o}dinger equation that describes a particle moving freely on a
curved space whose geometry is nontrivial, and vice versa. \ The main result
can be expressed in general as an integro-differential modification of
\href{https://en.wikipedia.org/wiki/Riccati_equation}{the Riccati
equation}\ \cite{Ince}.

In the case of two spatial dimensions (2D) with Euclidean signature, for a
particle with angular momentum $m$ so that the particle's wave function
factorizes as $\Psi\left(  r,\theta\right)  =\Psi_{m}\left(  r\right)
\exp\left(  im\theta\right)  $,\ the main result is
\begin{equation}
\frac{dW}{dr}+W^{2}+\frac{m^{2}}{R_{0}^{2}}\exp\left(  -4\int_{r_{0}}%
^{r}W\left(  \mathfrak{r}\right)  d\mathfrak{r}\right)  =U\left(  r\right)
+\frac{m^{2}-1/4}{r^{2}} \label{2DGeomFromPotl}%
\end{equation}
Upon solving (\ref{2DGeomFromPotl}), the function $W$ encodes the geometry in
terms of the invariant distance on the curved 2D space as%
\begin{equation}
\left(  ds\right)  ^{2}=\left(  dr\right)  ^{2}+R^{2}\left(  r\right)  \left(
d\theta\right)  ^{2}\text{ \ \ with \ \ }R\left(  r\right)  =R_{0}\exp\left(
2\int_{r_{0}}^{r}W\left(  \mathfrak{r}\right)  d\mathfrak{r}\right)  \text{
\ \ i.e. \ \ }W\left(  r\right)  =\frac{d}{dr}\ln\sqrt{R\left(  r\right)  }
\label{2DMetric}%
\end{equation}
and the invariant Laplacian on the 2-manifold is%
\begin{equation}
\frac{1}{\sqrt{g}}\partial_{\mu}\left(  \sqrt{g}g^{\mu\nu}\partial_{\nu
}\right)  =\frac{1}{R\left(  r\right)  }\partial_{r}\left(  R\left(  r\right)
\partial_{r}\right)  +\frac{1}{R\left(  r\right)  ^{2}}\partial_{\theta}%
^{2}=\partial_{r}^{2}+2W\left(  r\right)  \partial_{r}+\frac{1}{R\left(
r\right)  ^{2}}\partial_{\theta}^{2} \label{2DInvLaplacian}%
\end{equation}
Alternatively, given the geometry of the surface with a specified $R\left(
r\right)  $, the corresponding effective potential in flat space for angular
momentum $m$ is given by%
\begin{equation}
U\left(  r\right)  =\frac{1}{2}\frac{R^{\prime\prime}\left(  r\right)
}{R\left(  r\right)  }-\frac{1}{4}\left(  \frac{R^{\prime}\left(  r\right)
}{R\left(  r\right)  }\right)  ^{2}+\left(  \frac{m}{R\left(  r\right)
}\right)  ^{2}-\frac{m^{2}-\frac{1}{4}}{r^{2}} \label{2DPotlFromGeom}%
\end{equation}
As should be expected, for $R\left(  r\right)  \neq r$ this effective radial
potential will depend on the angular momentum.

To establish the relation between the potential problem in 2D flat space and
the 2D curved space system without a potential, it is only necessary to
eliminate the first derivative term in (\ref{2DInvLaplacian}) by writing
$\Psi\left(  r,\theta\right)  =\exp\left(  -\int_{r_{0}}^{r}W\left(
\mathfrak{r}\right)  d\mathfrak{r}\right)  \psi\left(  r,\theta\right)  $ to
obtain%
\begin{equation}
\left(  \partial_{r}^{2}+2W\left(  r\right)  \partial_{r}+\frac{1}{R\left(
r\right)  ^{2}}\partial_{\theta}^{2}\right)  \Psi\left(  r,\theta\right)
=\exp\left(  -\int_{r_{0}}^{r}W\left(  \mathfrak{r}\right)  d\mathfrak{r}%
\right)  \left(  \partial_{r}^{2}+\frac{1}{R\left(  r\right)  ^{2}}%
\partial_{\theta}^{2}-W^{\prime}-W^{2}\right)  \psi\left(  r,\theta\right)
\label{GeomRadial}%
\end{equation}
and compare this to the flat space system with a potential $U$, as obtained by
writing $\Psi\left(  r,\theta\right)  =\psi\left(  r,\theta\right)  /\sqrt{r}%
$, namely,
\begin{equation}
\left(  \partial_{r}^{2}+\frac{1}{r}\partial_{r}+\frac{1}{r^{2}}%
\partial_{\theta}^{2}-U\left(  r\right)  \right)  \Psi\left(  r,\theta\right)
=\frac{1}{\sqrt{r}}\left(  \partial_{r}^{2}+\frac{1}{r^{2}}\partial_{\theta
}^{2}-U\left(  r\right)  +\frac{1}{4r^{2}}\right)  \psi\left(  r,\theta
\right)  \label{PotlRadial}%
\end{equation}
Separating variables as $\psi\left(  r,\theta\right)  =\psi_{m}\left(
r\right)  \exp\left(  im\theta\right)  $ for both (\ref{GeomRadial}) and
(\ref{PotlRadial}) leads to the same second-order radial equation provided%
\begin{equation}
-\frac{m^{2}}{R\left(  r\right)  ^{2}}-W^{\prime}-W^{2}=-\frac{\left(
m^{2}-1/4\right)  }{r^{2}}-U\left(  r\right)
\end{equation}
and hence (\ref{2DGeomFromPotl}) with $R\left(  r\right)  =R_{0}\exp\left(
2\int_{r_{0}}^{r}W\left(  \mathfrak{r}\right)  d\mathfrak{r}\right)  $, or
alternatively (\ref{2DPotlFromGeom}).

In the case of $N$ spatial dimensions (ND) with Euclidean signature
\cite{Green}, on a manifold with isotropic metric of the same form as
(\ref{2DMetric}), a particle with angular momentum $\ell$ has a wave function
that factorizes as $\Psi=\Psi_{\ell}\left(  r\right)  Y_{lm_{1}m_{2}\cdots
m_{N-2}}\left(  \Omega\right)  $,\ with all the angular dependence in the
\href{https://en.wikipedia.org/wiki/Spherical_harmonics}{hyperspherical
harmonics} $Y_{lm_{1}m_{2}\cdots m_{N-2}}\left(  \Omega\right)  $. \ The
invariant Laplacian acting on $\Psi$ is
\begin{equation}
\frac{1}{\sqrt{g}}\partial_{\mu}\left(  \sqrt{g}g^{\mu\nu}\partial_{\nu
}\right)  \Psi=\frac{1}{R\left(  r\right)  ^{N-1}}~\partial_{r}\left(
R\left(  r\right)  ^{N-1}\partial_{r}\right)  \Psi-\frac{1}{R\left(  r\right)
^{2}}~L^{2}\Psi
\end{equation}
where all the $N-1$ angular derivatives are contained in $L_{jk}%
\equiv-i\left(  x_{j}\partial_{k}-x_{k}\partial_{j}\right)  $ with
$L^{2}\equiv\sum_{1\leq j<k\leq N}L_{jk}L_{jk}$. \ The $Y_{lm_{1}m_{2}\cdots
m_{N-2}}$ are eigenfunctions of $L^{2}$ and form a complete set on the
hypersphere, $S_{N-1}$, analogous to the familiar
\href{https://en.wikipedia.org/wiki/Spherical_harmonics}{spherical harmonics}
$Y_{lm}$ on $S_{2}$. \ Acting on $Y_{lm_{1}m_{2}\cdots m_{N-2}}$ the $L^{2}%
\ $eigenvalues are given by \cite{Sommerfeld}
\begin{equation}
L^{2}Y_{lm_{1}m_{2}\cdots m_{N-2}}=l\left(  l+N-2\right)  Y_{lm_{1}m_{2}\cdots
m_{N-2}}%
\end{equation}
for $l=0,1,2,\cdots$, generalizing the well-known $N=3$ case. \ The eigenvalue
equation (\ref{GeometricSE}) then reduces to the radial equation%
\begin{equation}
\frac{-1}{R\left(  r\right)  ^{N-1}}~\partial_{r}\left(  R\left(  r\right)
^{N-1}\partial_{r}\Psi_{l}\left(  r\right)  \right)  +\frac{l\left(
l+N-2\right)  }{R\left(  r\right)  ^{2}}\Psi_{l}\left(  r\right)
=\varepsilon\Psi_{l}\left(  r\right)  \text{ \ \ for \ \ }0\leq r\leq\infty
\end{equation}
Comparison to a potential system in ND flat space, (\ref{PotentialSE}), again
leads to the main result in the form of a modified Riccati equation.
\begin{equation}
\frac{dW}{dr}+W^{2}+\frac{\ell\left(  \ell+N-2\right)  }{R_{0}^{2}}%
~\exp\left(  \frac{4}{1-N}\int_{r_{0}}^{r}W\left(  \mathfrak{r}\right)
d\mathfrak{r}\right)  =U\left(  r\right)  +\frac{\left(  \ell+\frac{1}%
{2}\left(  N-1\right)  \right)  \left(  \ell+\frac{1}{2}\left(  N-3\right)
\right)  }{r^{2}} \label{NDGeomFromPotl}%
\end{equation}
Upon solving for the function $W$, the effective geometry is encoded in the
invariant distance on the curved ND space as%
\begin{equation}
\left(  ds\right)  ^{2}=\left(  dr\right)  ^{2}+R^{2}\left(  r\right)  \left(
d\Omega\right)  ^{2}\text{ \ \ with \ \ }R\left(  r\right)  =R_{0}\exp\left(
\frac{2}{N-1}\int_{r_{0}}^{r}W\left(  \mathfrak{r}\right)  d\mathfrak{r}%
\right)  \text{ \ \ i.e. \ \ }W\left(  r\right)  =\frac{N-1}{2}\frac{d}{dr}\ln
R\left(  r\right)  \label{NDMetric}%
\end{equation}
Alternatively, given the geometry of the surface with a specified $R\left(
r\right)  $, the corresponding potential in flat space for angular momentum
$\ell$ is given by%
\begin{equation}
U\left(  r\right)  =\frac{\left(  N-1\right)  }{2}\frac{d^{2}}{dr^{2}}\ln
R\left(  r\right)  +\left(  \frac{N-1}{2}\right)  ^{2}\left(  \frac{d}{dr}\ln
R\left(  r\right)  \right)  ^{2}+\frac{\ell\left(  \ell+N-2\right)  }%
{R^{2}\left(  r\right)  }-\frac{\left(  \ell+\frac{1}{2}\left(  N-1\right)
\right)  \left(  \ell+\frac{1}{2}\left(  N-3\right)  \right)  }{r^{2}}
\label{NDPotlFromGeom}%
\end{equation}
As in the 2D case, for $R\left(  r\right)  \neq r$ this effective radial
potential will depend on the angular momentum.

\section{Examples}

For $\ell=0$ \textquotedblleft s-wave\textquotedblright\ solutions,
(\ref{NDGeomFromPotl}) is an \emph{unmodified} Riccati equation, a first-order
differential equation whose solutions are well-studied \cite{Ince}. \ For a
$1/r$ potential in ND, \emph{plus a constant}, the $\ell=0$ case alone
provides an illustration of potential $\Longrightarrow$\ effective geometry.
\ Let
\begin{equation}
U\left(  r\right)  =\frac{\kappa}{r}+\frac{\kappa^{2}}{\left(  N-1\right)
^{2}}%
\end{equation}
so that (\ref{NDGeomFromPotl})\ for $\ell=0$ becomes%
\begin{equation}
\frac{dW}{dr}+W^{2}=\frac{\kappa}{r}+\frac{\kappa^{2}}{\left(  N-1\right)
^{2}}+\frac{\left(  N-1\right)  \left(  N-3\right)  }{4r^{2}}%
\end{equation}
A particularly simple, rational solution is immediately seen to be%
\begin{equation}
W\left(  r\right)  =\frac{N-1}{2r}+\frac{\kappa}{N-1} \label{particular 1/r}%
\end{equation}
with general solutions obtained by quadrature \cite{Ince}. \ 

The 3D case is exceptional in that there is no $1/r^{2}$ term on the RHS of
the Riccati equation. \ This physically interesting case readily admits
another solution for a pure Coulomb potential, sans constant,
namely,\footnote{NB \ This is the $1/r$ potential expressed as it would be in
\href{https://en.wikipedia.org/wiki/Supersymmetric_quantum_mechanics}{supersymmetric
quantum mechanics}.}%
\begin{gather}
\frac{dW\left(  r\right)  }{dr}+W^{2}\left(  r\right)  =\frac{\kappa}{r}\text{
\ \ for \ \ }N=3\text{ \ \ with}\nonumber\\
W\left(  r\right)  =\sqrt{\frac{\kappa}{r}}\frac{I_{0}\left(  2\sqrt{\kappa
r}\right)  }{I_{1}\left(  2\sqrt{\kappa r}\right)  }=\frac{1}{r}+\frac{\kappa
}{2}-\frac{\kappa^{2}r}{12}+O\left(  r^{2}\right)
\end{gather}
which is well-approximated by (\ref{particular 1/r}) for small $r$. \ But as
before, this is a solution only for $\ell=0$.

The effective geometry corresponding to the solution (\ref{particular 1/r}) is
given by%
\begin{equation}
R\left(  r\right)  =Kre^{2\kappa r/\left(  N-1\right)  ^{2}}\ ,\ \ \ K=\frac
{R_{0}}{r_{0}}~e^{-2\kappa r_{0}/\left(  N-1\right)  ^{2}}%
\end{equation}
The invariant distance for this particular ND manifold is then%
\begin{equation}
\left(  ds\right)  ^{2}=\left(  dr\right)  ^{2}+K^{2}r^{2}e^{4\kappa r/\left(
N-1\right)  ^{2}}\left(  d\Omega\right)  ^{2}%
\end{equation}
In terms of more conventional isotropic coordinates, such that $\left(
ds\right)  ^{2}=\left(  \frac{dr\left(  \rho\right)  }{d\rho}\right)
^{2}\left(  d\rho\right)  ^{2}+\rho^{2}\left(  d\Omega\right)  ^{2}$, let%
\begin{equation}
\rho^{2}=K^{2}r^{2}e^{4\kappa r/\left(  N-1\right)  ^{2}}%
\end{equation}
and solve for $r\left(  \rho\right)  $ in terms of the
\href{https://en.wikipedia.org/wiki/Lambert_W_function}{Lambert W function}.
\ For a repulsive potential $\kappa\geq0$, and thus%
\begin{equation}
r\left(  \rho\right)  =\frac{\left(  N-1\right)  ^{2}}{2\kappa}%
~\operatorname{LambertW}\left(  \frac{2\kappa\rho}{K\left(  N-1\right)  ^{2}%
}\right)  \ ,\ \ \ \frac{dr\left(  \rho\right)  }{d\rho}=\frac{\left(
N-1\right)  ^{2}}{2\kappa\rho}\left(  \frac{\operatorname{LambertW}\left(
\frac{2\kappa\rho}{K\left(  N-1\right)  ^{2}}\right)  }%
{1+\operatorname{LambertW}\left(  \frac{2\kappa\rho}{K\left(  N-1\right)
^{2}}\right)  }\right)
\end{equation}
The resulting geometry is described by%
\begin{equation}
\left(  ds\right)  ^{2}=\left(  \frac{\left(  N-1\right)  ^{2}}{2\kappa\rho
}\frac{\operatorname{LambertW}\left(  \frac{2\kappa\rho}{K\left(  N-1\right)
^{2}}\right)  }{1+\operatorname{LambertW}\left(  \frac{2\kappa\rho}{K\left(
N-1\right)  ^{2}}\right)  }\right)  ^{2}\left(  d\rho\right)  ^{2}+\rho
^{2}\left(  d\Omega\right)  ^{2}%
\end{equation}
A canonical embedding of this $N$-manifold into an $N+1$ Lorentz space is
given by%
\begin{equation}
\left(  ds\right)  ^{2}=\left(  d\rho\right)  ^{2}+\rho^{2}\left(
d\Omega\right)  ^{2}-c^{2}\left(  dt\left(  \rho\right)  \right)  ^{2}%
\end{equation}
with%
\begin{equation}
c\frac{dt\left(  \rho\right)  }{d\rho}=\frac{\sqrt{\left(
1+\operatorname{LambertW}\left(  \frac{2\kappa\rho}{K\left(  N-1\right)  ^{2}%
}\right)  \right)  ^{2}-\left(  \frac{\left(  N-1\right)  ^{2}}{2\kappa\rho
}\operatorname{LambertW}\left(  \frac{2\kappa\rho}{K\left(  N-1\right)  ^{2}%
}\right)  \right)  ^{2}}}{1+\operatorname{LambertW}\left(  \frac{2\kappa\rho
}{K\left(  N-1\right)  ^{2}}\right)  } \label{TimeEmbedding}%
\end{equation}
For $\kappa\geq0$ this is a real embedding all the way down to $\rho=0$ if
$K\geq1$. \ For $K<1$ the embedding is real only for $\rho_{\min}\leq\rho
\leq\infty$ where $\rho_{\min}$ is given by the positive real root of the
radicand in (\ref{TimeEmbedding}).

Graphical representations of the embedded surface and generalizations to
situations where $\ell\neq0$ are left as an exercise for the reader, as is the
$\kappa<0$ situation. \ But it should already be evident from the $\ell=0$
case that a repulsive $1/r$ model is more easily understood as a potential
problem than it is from the geometrical side of the relationship. \ Rather
than pursue the quantum mechanics on the resulting ND manifold for this
example, we consider next a more troublesome singular potential which can be
regularized by mapping onto a well-known geometry.

\section{Wormholes}

Consider a smooth, spatial \textquotedblleft bridge\textquotedblright%
\ manifold \cite{ER} (i.e. a \textquotedblleft wormhole\textquotedblright%
\ \cite{Thorne}) given by%
\begin{equation}
\left(  ds\right)  ^{2}=\left(  dw\right)  ^{2}+R^{2}\left(  w\right)  \left(
d\Omega\right)  ^{2}\ ,\ \ \ -\infty\leq w\leq+\infty\label{WormholeMetric}%
\end{equation}
where $R\left(  w\right)  >0$ has an absolute minimum at $w=0$, and behaves
asymptotically as $R^{2}\left(  w\right)  \underset{w\rightarrow\pm
\infty}{\sim}w^{2}+O\left(  1/w^{2}\right)  $. \ For example, the $w$-form of
a static Ellis metric \cite{Ellis} in $N$ spatial dimensions is simply
\begin{equation}
R^{2}\left(  w\right)  =R_{0}^{2}+w^{2}\ ,\ \ \ -\infty\leq w\leq
+\infty\label{Ellis}%
\end{equation}
The metric (\ref{WormholeMetric}) leads to the invariant Laplacian
\begin{equation}
\frac{1}{\sqrt{g}}\partial_{\mu}\left(  \sqrt{g}g^{\mu\nu}\partial_{\nu}%
\Psi\right)  =\frac{1}{R\left(  w\right)  ^{N-1}}~\partial_{w}\left(  R\left(
w\right)  ^{N-1}\partial_{w}\right)  -\frac{1}{R\left(  w\right)  ^{2}}~L^{2}%
\end{equation}
where again all the $N-1$ angular derivatives are contained in $L^{2}$. \ 

The non-relativistic Schr\"{o}dinger energy eigenvalue problem for a particle
moving freely on this manifold is again solved by separating variables.
\begin{equation}
\frac{1}{\sqrt{g}}\partial_{\mu}\left(  \sqrt{g}g^{\mu\nu}\partial_{\nu}%
\Psi\right)  +\varepsilon\Psi=0\ ,\ \ \ \Psi=\Psi_{\ell}\left(  w\right)
Y_{\ell m_{1}m_{2}\cdots m_{N-2}}\left(  \Omega\right)
\end{equation}
The effective radial equation is now%
\begin{equation}
\frac{1}{R\left(  w\right)  ^{N-1}}~\partial_{w}\left(  R\left(  w\right)
^{N-1}\partial_{w}\Psi_{\ell}\left(  w\right)  \right)  +\left(
\varepsilon-\frac{\ell\left(  \ell+N-2\right)  }{R\left(  w\right)  ^{2}%
}\right)  \Psi_{\ell}\left(  w\right)  =0 \label{RadialWormhole}%
\end{equation}
Rather than compare this to rotationally invariant potential scattering in ND
flat space, as above, instead compare this eigenvalue equation to
one-dimensional potential scattering on the line $-\infty\leq w\leq+\infty$,
as given by%
\begin{equation}
\partial_{w}^{2}\Psi_{\ell}\left(  w\right)  +\left(  \varepsilon-U\left(
w\right)  \right)  \Psi_{\ell}\left(  w\right)  =0 \label{OnALine}%
\end{equation}
To establish the sought-for relation, again just eliminate the first
derivative term in (\ref{RadialWormhole}) by writing $\Psi_{\ell}\left(
w\right)  =\left(  R\left(  w\right)  \right)  ^{\left(  1-N\right)  /2}%
\psi_{\ell}\left(  w\right)  $ to obtain the energy eigenvalue equation%
\begin{align}
&  0=\left(  \partial_{w}^{2}+\left(  N-1\right)  \frac{R^{\prime}}{R}%
\partial_{w}+\left(  \varepsilon-\frac{\ell\left(  \ell+N-2\right)  }{R\left(
w\right)  ^{2}}\right)  \right)  \Psi_{\ell}\left(  w\right) \nonumber\\
&  =\left(  R\left(  w\right)  \right)  ^{\left(  1-N\right)  /2}\left(
\partial_{w}^{2}+\left(  \varepsilon-\frac{1}{2}\left(  N-1\right)
\frac{R^{\prime\prime}}{R}-\frac{1}{4}\left(  N-1\right)  \left(  N-3\right)
\left(  \frac{R^{\prime}}{R}\right)  ^{2}-\frac{\ell\left(  \ell+N-2\right)
}{R^{2}}\right)  \right)  \psi_{\ell}\left(  w\right)  \label{2ndOrder}%
\end{align}
where primes indicate derivatives with respect to $w$. \ The second-order
equation (\ref{2ndOrder}) is the same as that for potential scattering on the
line (\ref{OnALine}) if
\begin{equation}
U\left(  w\right)  =\frac{1}{2}\left(  N-1\right)  \frac{R^{\prime\prime}}%
{R}+\frac{1}{4}\left(  N-1\right)  \left(  N-3\right)  \left(  \frac
{R^{\prime}}{R}\right)  ^{2}+\frac{\ell\left(  \ell+N-2\right)  }{R^{2}}
\label{LinePotl}%
\end{equation}


\subsection{Static Ellis Metric}

For the Ellis metric (\ref{Ellis}), $R^{\prime}=w/R$, $R^{\prime\prime}%
=R_{0}^{2}/R^{3}$, so%
\begin{equation}
U\left(  w\right)  =\left(  N+2\ell-3\right)  \left(  N+2\ell-1\right)
\frac{1}{4R\left(  w\right)  ^{2}}-\left(  N-1\right)  \left(  N-5\right)
\frac{R_{0}^{2}}{4R\left(  w\right)  ^{4}} \label{NDEllisPotl}%
\end{equation}
In particular, for 3D,%
\begin{equation}
U\left(  w\right)  =\frac{\ell\left(  \ell+1\right)  }{R_{0}^{2}+w^{2}}%
+\frac{R_{0}^{2}}{\left(  R_{0}^{2}+w^{2}\right)  ^{2}}\text{ \ \ for \ \ }N=3
\label{3DEllisPotl}%
\end{equation}
This $U\left(  w\right)  $ may be thought of as a regularized form of
$1/r^{4}$, a singular potential that has been well-studied in 3D
\cite{Singular1950,Singular1971}.

The angular momentum \emph{in}dependent $1/R^{4}$ term in the potential
(\ref{NDEllisPotl}) is repulsive for $N\leq4$, absent for $N=5$, and
attractive for $N\geq6$. \ (It so happens to be a regularized form of an
attractive electrostatic potential for a point particle for $N=6$
\cite{Green,C et al.}.) \ In any case, non-relativistic scattering for the
regular potential $U\left(  w\right)  $ has both elastic and inelastic
components for all $N\geq2 $, where elastic scattering is interpreted as both
incident and emergent probability flux on the \textquotedblleft
upper\textquotedblright\ branch of the wormhole manifold, and inelastic
scattering is to be understood as emergent flux on the \textquotedblleft
lower\textquotedblright\ branch of the manifold, without any incident flux on
that lower branch. \ For incident flux on the wormhole's upper branch, with
$R_{0}>0$, there is always some probability flow through the bridge joining
the two branches so that particles are effectively absorbed by the wormhole,
as viewed by an observer located on the upper branch. \ For any $N$, the time
independent partial wave scattering amplitudes can be determined exactly for
$U\left(  w\right)  $ in terms of spheroidal wave functions \cite{DLMF}. \ An
Argand diagram of the phase shifts clearly reveals the inelastic behavior.
\ This is discussed in considerable detail in \cite{CS}.

The singular $1/r^{4}$ potential that arises from $U\left(  w\right)  $ in the
limit as $R_{0}\rightarrow0$ is somewhat problematic, as explained in
\cite{Singular1971}\ and references cited therein. \ At issue is whether the
Hamiltonian for such a potential admits a
\href{https://en.wikipedia.org/wiki/Self-adjoint_operator#:~:text=Self-adjoint%20operators%20are%20used%20in%20functional%20analysis%20and,represented%20by%20self-adjoint%20operators%20on%20a%20Hilbert%20space.}{self-adjoint}
extension for $0\leq r\leq\infty$, with only real eigenvalues and unitary time
evolution. \ For a repulsive $1/r^{4}$ this can be arranged with an
appropriate choice of boundary conditions at $r=0$. \ An attractive $1/r^{4}%
$\ is another story, however, with considerable ambiguity in the boundary
conditions at the origin. \ As stated in \cite{Singular1971}:\bigskip

\begin{quote}
\textquotedblleft Thus the basic feature of an attractive singular potential
is seen to lie in the fact that physical processes are not uniquely
determined. \ This gives rise to the possibility of imposing unusual or
unconventional boundary conditions in physical problems as a means of
representing particular physical processes. \ An example of a process of this
type is provided by particle absorption or capture.\textquotedblright\bigskip

\textquotedblleft Since the deficiency indices are infinite, nonself-adjoint
extensions are possible which would correspond to boundary conditions which
describe inelastic scattering.\textquotedblright\bigskip

\textquotedblleft Thus, in the singular case, the long-range part of the force
between particles does not alone suffice to determine their behavior; some
cutoff mechanism apparently must be provided.\textquotedblright\bigskip
\end{quote}

The regularization defined by the wormhole geometry avoids all these
ambiguities. \ The Hamiltonian for a non-relativistic system governed by
$U\left(  w\right)  $ is manifestly self-adjoint for $R_{0}>0$ when defined on
a \href{https://en.wikipedia.org/wiki/Rigged_Hilbert_space}{rigged Hilbert
space} with the usual boundary conditions as $w\rightarrow\pm\infty$. \ But
there is a price to be paid for this mathematical convenience, namely,
inelastic scattering, even when the potential is repulsive. \ This feature for
$U\left(  w\right)  $ is discussed further in \cite{CS}. \ However, examples
of such inelastic behavior can also be found in a class of wormhole
deformations, especially in the limit where the wormhole throat reduces to an
intrinsically flat hypercylinder. \ 

\subsection{A class of deformed wormholes}

An class\ of static wormholes may be defined in $N$ spatial dimensions by
\cite{CA et al.,AC}%
\begin{equation}
R\left(  w\right)  =\left(  R_{0}^{n}+w^{n}\right)  ^{1/n}=R_{0}P\left(
x,n\right)
\end{equation}
for even integer $n\geq2$, where we have defined the wormhole
\textquotedblleft profile\textquotedblright\ using an $L^{n}$ norm on the
plane, $P\left(  x,n\right)  =\left(  1+x^{n}\right)  ^{1/n}$ with $x=w/R_{0}%
$. \ This leads to%
\begin{equation}
R_{0}^{2}U\left(  w\right)  =\frac{\ell\left(  \ell+N-2\right)  }{P\left(
x,n\right)  ^{2}}+\frac{N-1}{4}\frac{x^{n-2}}{P\left(  x,n\right)  ^{2n}%
}\left(  \left(  2\left(  n-1\right)  +\left(  N-3\right)  x^{n}\right)
\right)
\end{equation}
In particular, in 3D,
\begin{equation}
\left.  R_{0}^{2}U\left(  w\right)  \right\vert _{N=3}=\frac{\ell\left(
\ell+1\right)  }{P\left(  x,n\right)  ^{2}}+\frac{\left(  n-1\right)  x^{n-2}%
}{P\left(  x,n\right)  ^{2n}}%
\end{equation}
The 3D $\ell=0$ and $\ell=1$ effective potentials for various $n$ are shown in
the following graphs.%

%TCIMACRO{\FRAME{itbpFU}{3.039in}{2.0167in}{0in}{\Qcb{$\left.  R_{0}%
%^{2}U\left(  w\right)  \right\vert _{N=3,\ \ell=0}$ versus $x=w/R_{0}$ for
%$n=2,4,8,16$ in red, green, orange, and blue, respectively.}}{}{pvgrfig1.jpg}%
%{\special{ language "Scientific Word";  type "GRAPHIC";
%maintain-aspect-ratio TRUE;  display "USEDEF";  valid_file "F";
%width 3.039in;  height 2.0167in;  depth 0in;  original-width 3.9271in;
%original-height 2.5936in;  cropleft "0";  croptop "1";  cropright "1";
%cropbottom "0";  filename 'PVGRFig1.jpg';file-properties "XNPEU";}} }%
%BeginExpansion
{\parbox[b]{3.039in}{\begin{center}
\includegraphics[
natheight=2.593600in,
natwidth=3.927100in,
height=2.0167in,
width=3.039in
]%
{C:/Users/Thomas Curtright/OneDrive - University of Miami/Desktop/arXiv attempts/graphics/PVGRFig1__1.pdf}%
\\
$\left.  R_{0}^{2}U\left(  w\right)  \right\vert _{N=3,\ \ell=0}$ versus
$x=w/R_{0}$ for $n=2,4,8,16$ in red, green, orange, and blue, respectively.
\end{center}}}
%EndExpansion
%TCIMACRO{\FRAME{itbpFU}{3.039in}{2.0228in}{0in}{\Qcb{$\left.  R_{0}%
%^{2}U\left(  w\right)  \right\vert _{N=3,\ \ell=1}$ versus $x=w/R_{0}$ for
%$n=2,4,8,16$ in red, green, orange, and blue, respectively.}}{}{pvgrfig2.jpg}%
%{\special{ language "Scientific Word";  type "GRAPHIC";
%maintain-aspect-ratio TRUE;  display "USEDEF";  valid_file "F";
%width 3.039in;  height 2.0228in;  depth 0in;  original-width 3.9167in;
%original-height 2.5936in;  cropleft "0";  croptop "1";  cropright "1";
%cropbottom "0";  filename 'PVGRFig2.jpg';file-properties "XNPEU";}} }%
%BeginExpansion
{\parbox[b]{3.039in}{\begin{center}
\includegraphics[
natheight=2.593600in,
natwidth=3.916700in,
height=2.0228in,
width=3.039in
]%
{C:/Users/Thomas Curtright/OneDrive - University of Miami/Desktop/arXiv attempts/graphics/PVGRFig2__2.pdf}%
\\
$\left.  R_{0}^{2}U\left(  w\right)  \right\vert _{N=3,\ \ell=1}$ versus
$x=w/R_{0}$ for $n=2,4,8,16$ in red, green, orange, and blue, respectively.
\end{center}}}
%EndExpansion


\noindent As $n$ increases, the effective potential is similar to a $\delta
$-shell \cite{Gottfried} but with potential spikes at \emph{both} top and
bottom edges of the wormhole. \ Also, in the vicinity of the wormhole's
throat, for $\ell>0$ there is a positive but \emph{finite} effective angular
momentum barrier of maximum height $\ell\left(  \ell+1\right)  /R_{0}^{2}$.
\ That is to say, the asymptotic form of the $N=3$ effective potential for the
equivalent particle on the line $-\infty<x<+\infty$, as $n\rightarrow\infty$,
is%
\begin{equation}
U_{\infty}=\frac{\ell\left(  \ell+1\right)  }{R_{0}^{2}}\left(  \Theta\left(
1-x^{2}\right)  +\frac{\Theta\left(  x^{2}-1\right)  }{x^{2}}\right)
+\frac{1}{R_{0}^{2}}\left(  \delta\left(  x-1\right)  +\delta\left(
x+1\right)  \right)
\end{equation}
where $x=w/R_{0}$, $\Theta$ is the Heaviside step function, and $\delta$ is a
Dirac delta. \ The finite part of $R_{0}^{2}U_{\infty}$ has a simple
\textquotedblleft\href{https://en.wikipedia.org/wiki/Devils_Tower}{Devil's
Tower}\textquotedblright\ profile, with height $\ell\left(  \ell+1\right)  $.%
%TCIMACRO{\FRAME{dtbpFU}{4.5688in}{3.039in}{0pt}{\Qcb{$\ell\left(
%\ell+1\right)  \left(  \Theta\left(  1-x^{2}\right)  +\frac{1}{x^{2}}%
%~\Theta\left(  x^{2}-1\right)  \right)  $ for $\ell=1,2$, in black, compared
%to $\left.  \frac{\ell\left(  \ell+1\right)  }{P\left(  x,n\right)  ^{2}%
%}\right\vert _{n=8}$ in red.}}{}{devilstower.jpg}%
%{\special{ language "Scientific Word";  type "GRAPHIC";
%maintain-aspect-ratio TRUE;  display "USEDEF";  valid_file "F";
%width 4.5688in;  height 3.039in;  depth 0pt;  original-width 5.8747in;
%original-height 3.896in;  cropleft "0";  croptop "1";  cropright "1";
%cropbottom "0";  filename 'DevilsTower.jpg';file-properties "XNPEU";}} }%
%BeginExpansion
\begin{center}
\includegraphics[
natheight=3.896000in,
natwidth=5.874700in,
height=3.039in,
width=4.5688in
]%
{C:/Users/Thomas Curtright/OneDrive - University of Miami/Desktop/arXiv attempts/graphics/DevilsTower__3.pdf}%
\\
$\ell\left(  \ell+1\right)  \left(  \Theta\left(  1-x^{2}\right)  +\frac
{1}{x^{2}}~\Theta\left(  x^{2}-1\right)  \right)  $ for $\ell=1,2$, in black,
compared to $\left.  \frac{\ell\left(  \ell+1\right)  }{P\left(  x,n\right)
^{2}}\right\vert _{n=8}$ in red.
\end{center}
%EndExpansion


At first sight the Dirac deltas appearing in $U_{\infty}$ might be surprising,
since the invariant Laplacian only involves first derivatives of a finite
continuous metric, even as $n\rightarrow\infty$, hence at most step functions
due to those first derivatives. \ But second derivatives of $R\left(
w\right)  $ occur from the Laplacian acting on the prefactor in $\Psi_{\ell
}\left(  w\right)  =\left(  R\left(  w\right)  \right)  ^{\left(  1-N\right)
/2}\psi_{\ell}\left(  w\right)  $. \ That is to say, $R\left(  w\right)
^{\prime\prime}$ gives Dirac deltas as $n\rightarrow\infty$ since in that
limit the manifold has infinite curvature at $w/R_{0}=\pm1$. \ So indeed, as
$n\rightarrow\infty$ there are Dirac deltas in the unique energy eigenvalue
equation for $\psi_{\ell}$, namely, $0=\left(  \partial_{w}^{2}+\left(
\varepsilon-U_{\infty}\right)  \right)  \psi_{\ell}\left(  w\right)  $, but
there are only finite terms in the corresponding eigenvalue equation for
$\Psi_{\ell}\left(  w\right)  $ in that limit, albeit now the differential
equation for $\Psi_{\ell}$ must be expressed piecewise. \ Thus $\lim
_{n\rightarrow\infty}P\left(  x,n\right)  =1$ if $x^{2}<1$ and $\lim
_{n\rightarrow\infty}P\left(  x,n\right)  =\left\vert x\right\vert $ if
$x^{2}>1$ leads to%
\begin{equation}
\lim_{n\rightarrow\infty}\frac{1}{R\left(  w\right)  ^{N-1}}~\partial
_{w}\left(  R\left(  w\right)  ^{N-1}\partial_{w}\Psi_{\ell}\left(  w\right)
\right)  =\left\{
\begin{array}
[c]{ccc}%
\partial_{w}^{2}\Psi_{\ell}\left(  w\right)  & \text{if} & w^{2}<R_{0}^{2}\\
&  & \\
\frac{1}{w^{N-1}}\partial_{w}\left(  w^{N-1}\partial_{w}\Psi_{\ell}\left(
w\right)  \right)  & \text{if} & w^{2}>R_{0}^{2}%
\end{array}
\right.
\end{equation}
and therefore%
\begin{align}
&  \lim_{n\rightarrow\infty}\frac{1}{R\left(  w\right)  ^{N-1}}~\partial
_{w}\left(  R\left(  w\right)  ^{N-1}\partial_{w}\Psi_{\ell}\left(  w\right)
\right)  +\left(  \varepsilon-\frac{\ell\left(  \ell+N-2\right)  }{R\left(
w\right)  ^{2}}\right)  \Psi_{\ell}\left(  w\right) \nonumber\\
&  =\left\{
\begin{array}
[c]{ccc}%
\partial_{w}^{2}\Psi_{\ell}\left(  w\right)  +\left(  \varepsilon-\frac
{\ell\left(  \ell+N-2\right)  }{R_{0}^{2}}\right)  \Psi_{\ell}\left(  w\right)
& \text{if} & w^{2}<R_{0}^{2}\\
&  & \\
\frac{1}{w^{N-1}}\partial_{w}\left(  w^{N-1}\partial_{w}\Psi_{\ell}\left(
w\right)  \right)  +\left(  \varepsilon-\frac{\ell\left(  \ell+N-2\right)
}{w^{2}}\right)  \Psi_{\ell}\left(  w\right)  & \text{if} & w^{2}>R_{0}^{2}%
\end{array}
\right.
\end{align}
Nevertheless, even in the infinite $n$ limit the solution of either eigenvalue
equation leads to exactly the same results, upon realizing that
\begin{equation}
\lim_{n\rightarrow\infty}\Psi_{\ell}\left(  w\right)  =\left\{
\begin{array}
[c]{ccc}%
R_{0}^{\left(  1-N\right)  /2}\psi_{\ell}\left(  w\right)  & \text{if} &
w^{2}<R_{0}^{2}\\
&  & \\
w^{\left(  1-N\right)  /2}\psi_{\ell}\left(  w\right)  & \text{if} &
w^{2}>R_{0}^{2}%
\end{array}
\right.
\end{equation}


More explicitly for $N=3$, while a potential with a unit strength Dirac delta
$\delta\left(  x-1\right)  $ implies a discontinuity in $\psi^{\prime}$ at
$x=1$, there is no such discontinuity in the derivative of $\Psi\left(
x\right)  =\psi\left(  x\right)  \Theta\left(  1-x\right)  +\frac{1}{x}%
~\psi\left(  x\right)  \Theta\left(  x-1\right)  $. \ Thus
\begin{align}
0  &  =\lim_{\epsilon\rightarrow0}\int_{1-\epsilon}^{1+\epsilon}\left(
\delta\left(  x-1\right)  \psi\left(  x\right)  -\frac{d^{2}}{dx^{2}}%
\psi\left(  x\right)  \right)  dx\nonumber\\
&  =\psi\left(  1\right)  -\lim_{\epsilon\rightarrow0}\left(  \psi^{\prime
}\left(  1+\epsilon\right)  -\psi^{\prime}\left(  x-\epsilon\right)  \right)
\nonumber\\
&  =\psi\left(  1\right)  -\lim_{\epsilon\rightarrow0}\left(  \left.  \frac
{d}{dx}\left(  x\Psi\left(  x\right)  \right)  \right\vert _{x=1+\epsilon
}-\left.  \frac{d}{dx}\Psi\left(  x\right)  \right\vert _{x=1-\epsilon}\right)
\nonumber\\
&  =\psi\left(  1\right)  -\psi\left(  1\right)  -\lim_{\epsilon\rightarrow
0}\left(  \left.  \frac{d}{dx}\Psi\left(  x\right)  \right\vert _{x=1+\epsilon
}-\left.  \frac{d}{dx}\Psi\left(  x\right)  \right\vert _{x=1-\epsilon
}\right)
\end{align}
Hence $\Psi$ has a continuous first derivative at $x=1$:
\begin{equation}
\lim_{\epsilon\rightarrow0}\left(  \left.  \frac{d}{dx}\Psi\left(  x\right)
\right\vert _{x=1+\epsilon}-\left.  \frac{d}{dx}\Psi\left(  x\right)
\right\vert _{x=1-\epsilon}\right)  =0
\end{equation}
For the particular $N=3$ eigenvalue equation at hand, with energy
$\varepsilon=k^{2}$, $\Psi_{\ell}\left(  w\right)  $ is a linear combination
of $h_{\ell}^{\left(  1,2\right)  }\left(  kw\right)  $ for $w>R_{0}$, while
$\psi_{\ell}\left(  w\right)  $ is a linear combination of $\sqrt{kw}H_{\ell
}^{\left(  1,2\right)  }\left(  kw\right)  $ for $w>R_{0}$, but both
$\Psi_{\ell}\left(  w\right)  $ and $\psi_{\ell}\left(  w\right)  $ are linear
combinations of $\exp\left(  \pm i\kappa w\right)  $ for $-R_{0}<w<R_{0}$,
where $\kappa^{2}=k^{2}-\ell\left(  \ell+1\right)  /R_{0}^{2}$.

So then, the limit $n\rightarrow\infty$ reduces to two parallel copies of flat
Euclidean space, $\mathbb{E}_{N}$, punctured by holes of radius $R_{0}$, and
connected by a flat hypercylinder whose \textquotedblleft
top\textquotedblright\ and \textquotedblleft bottom\textquotedblright%
\ boundaries consist of the two holes, as shown graphically in the Appendix.
\ Scattering on this geometry is straightforward to analyze simply by invoking
the boundary conditions of both continuous $\Psi$ and continuous normal
derivative $\Psi^{\prime}$ at each of the holes. \ The results may be
expressed analytically in terms of Gegenbauer polynomials and Bessel
functions, and may be computed numerically without great effort, as
demonstrated in the next section.

\newpage

\subsection{Wormhole scattering amplitudes}

In $N$ spatial Euclidean dimensions, $\mathbb{E}_{N}$, outside a localized,
rotationally invariant scattering center, the time-independent, incoming,
scattered, and total wave functions are given by partial wave expansions
\cite{Sommerfeld}.%
\begin{gather}
\Psi_{\text{in}}\left(  \overrightarrow{r}\right)  =\frac{2^{\left(
N-2\right)  /2}\Gamma\left(  \frac{N-2}{2}\right)  }{4\left(  kr\right)
^{\frac{N-2}{2}}}\sum_{n=0}^{\infty}i^{n}\left(  2n+N-2\right)  C_{n}^{\left(
\frac{N-2}{2}\right)  }\left(  \cos\theta\right)  ~\left(  H_{n+\frac{N-2}{2}%
}^{\left(  1\right)  }\left(  kr\right)  +H_{n+\frac{N-2}{2}}^{\left(
2\right)  }\left(  kr\right)  \right) \\
\nonumber\\
\Psi_{\text{sc}}\left(  \overrightarrow{r}\right)  =\frac{2^{\left(
N-2\right)  /2}\Gamma\left(  \frac{N-2}{2}\right)  }{4\left(  kr\right)
^{\frac{N-2}{2}}}\sum_{n=0}^{\infty}i^{n}\left(  2n+N-2\right)  C_{n}^{\left(
\frac{N-2}{2}\right)  }\left(  \cos\theta\right)  ~\left(  S_{n}-1\right)
~H_{n+\frac{N-2}{2}}^{\left(  1\right)  }\left(  kr\right) \\
\nonumber\\
\Psi_{\text{total}}\left(  \overrightarrow{r}\right)  =\Psi_{\text{in}}\left(
\overrightarrow{r}\right)  +\Psi_{\text{sc}}\left(  \overrightarrow{r}\right)
\end{gather}
Here $H_{n+\frac{N-2}{2}}^{\left(  1,2\right)  }\left(  kr\right)
=J_{n+\frac{N-2}{2}}\left(  kr\right)  \pm iY_{n+\frac{N-2}{2}}\left(
kr\right)  $ are \href{https://en.wikipedia.org/wiki/Bessel_function}{Bessel
functions} and $C_{n}^{\left(  \frac{N-2}{2}\right)  }$ are conventionally
normalized,
orthogonal\ \href{https://en.wikipedia.org/wiki/Gegenbauer_polynomials}{Gegenbauer
polynomials}. \ Of course, $\Psi_{\text{in}}\left(  \overrightarrow{r}\right)
$\ is just a series expansion for the usual plane wave expressed in spherical
coordinates, $\Psi_{\text{in}}\left(  \overrightarrow{r}\right)  =\exp\left(
ikr\cos\theta\right)  $.

The differential cross-section, $d\sigma/d\Omega_{N}$, follows from the radial
probability flux of $\Psi_{\text{sc}}\left(  \overrightarrow{r}\right)  $,
while the integrated cross-section is $\sigma=\int\frac{d\sigma}{d\Omega_{N}%
}~d\Omega_{N}$. \ The result for $\sigma$ in $N$ spatial dimensions is%
\begin{equation}
\sigma=\frac{1}{\Omega_{N}}\left(  \frac{2\pi}{k}\right)  ^{N-1}\sum
_{n=0}^{\infty}\dim\left(  n,N\right)  \left\vert S_{n}-1\right\vert ^{2}
\label{SigmaN}%
\end{equation}
where $\dim\left(  n,N\right)  $ is the dimension of a totally symmetric,
traceless, rank $n$ tensor representation of $SO\left(  N\right)  $, namely,%
\begin{equation}
\dim\left(  n,N\right)  =\frac{\left(  2n+N-2\right)  }{\Gamma\left(
N-1\right)  }\frac{\Gamma\left(  n+N-2\right)  }{\Gamma\left(  n+1\right)  }%
\end{equation}
and $\Omega_{N}$ is the total \textquotedblleft solid angle\textquotedblright%
\ in $N$ spatial dimensions, i.e. the \textquotedblleft surface
area\textquotedblright\ of a unit radius hyper-sphere, namely,%
\begin{equation}
\Omega_{N}=\frac{2\pi^{\frac{N}{2}}}{\Gamma\left(  \frac{N}{2}\right)  }%
\end{equation}
Other ways to write the pre-factor in $\sigma$ are
\begin{equation}
\frac{1}{\Omega_{N}}\left(  \frac{2\pi}{k}\right)  ^{N-1}=\frac{1}{k^{N-1}%
}~\left(  4\pi\right)  ^{\frac{N-2}{2}}\Gamma\left(  \frac{N}{2}\right)
=\frac{1}{2k^{N-1}}~\Omega_{N-1}~\Gamma\left(  N-1\right)  =\frac
{V_{N-1}\left(  R\right)  }{\left(  kR\right)  ^{N-1}}\frac{\Gamma\left(
N\right)  }{2}%
\end{equation}
where $V_{N-1}\left(  R\right)  $ is the (hyper)volume of a ball of radius $R$
embedded in $N-1$ dimensions.%
\begin{equation}
V_{N-1}\left(  R\right)  =\frac{\Omega_{N-1}R^{N-1}}{N-1}=\frac{\pi
^{\frac{N-1}{2}}R^{N-1}}{\Gamma\left(  \tfrac{N+1}{2}\right)  }%
\end{equation}
Thus we obtain the dimensionless ratio%
\begin{equation}
\frac{\sigma}{V_{N-1}\left(  R\right)  }=\frac{2\Gamma\left(  N\right)
}{\left(  kR\right)  ^{N-1}}\sum_{n=0}^{\infty}\dim\left(  n,N\right)
\left\vert \frac{S_{n}-1}{2i}\right\vert ^{2} \label{SigmaNDimensionless}%
\end{equation}
For example, when $N=3$, $\dim\left(  n,3\right)  =\left(  2n+1\right)  $, and
$V_{2}\left(  R\right)  =\pi R^{2}$, i.e. the maximum area of a
cross-sectional circular disk for a sphere of radius $R$. \ 

Consider scattering for the $n\rightarrow\infty$ wormhole geometry described
above, with an incident plane wave $\Psi_{\text{in}}\left(  \overrightarrow{r}%
\right)  $ only on the upper copy of $\mathbb{E}_{N}$. \ The corresponding
\textquotedblleft elastic\textquotedblright\ cross section $\sigma$ would be
measurable by a \textquotedblleft top-side\textquotedblright\ observer with
detection apparatus only in the upper copy of $\mathbb{E}_{N}$, while
probability flux emerging on the lower copy of $\mathbb{E}_{N}$\ would appear
as an \textquotedblleft inelastic\textquotedblright\ contribution to the cross
section and not directly detected by that top-side observer.

Imposing continuity of $\Psi_{\text{total}}$ and $d\Psi_{\text{total}}/dw$ at
the top and bottom boundaries of the hypercylinder connecting the two copies
of $\mathbb{E}_{N}$, with cylinder radius $R_{0}$ and length $L_{0}$, the
partial wave scattering amplitudes for wave functions on the upper Euclidean
space are then given by%
\begin{equation}
S_{n}=-\frac{\left(  \kappa^{2}h_{1}h_{2}-k^{2}h_{1}^{\prime}h_{2}^{\prime
}\right)  \sin\left(  \kappa L_{0}\right)  +k\kappa\left(  h_{1}h_{2}^{\prime
}+h_{2}h_{1}^{\prime}\right)  \cos\left(  \kappa L_{0}\right)  }{\left(
\kappa^{2}h_{1}^{2}-k^{2}(h_{1}^{\prime})^{2}\right)  \sin\left(  \kappa
L_{0}\right)  +2k\kappa h_{1}h_{1}^{\prime}\cos\left(  \kappa L_{0}\right)  }
\label{WormholeScatteringAmplitudes}%
\end{equation}
where the abbreviations are as follows. \ For $N\geq2$, \
\begin{equation}
h_{1,2}=\frac{1}{\left(  kR_{0}\right)  ^{\frac{N-2}{2}}}~H_{n+\frac{N-2}{2}%
}^{\left(  1,2\right)  }\left(  kR_{0}\right)  \ ,\ \ \ \kappa=\sqrt
{k^{2}-n\left(  n+N-2\right)  /R_{0}^{2}}%
\end{equation}
It follows that $\left\vert S_{n}\right\vert \neq1$ for generic values of $k$,
so the scattering has an inelastic component as seen by the top-side observer. \ 

For example, consider the cases $N=2$ and $3$, the former being more readily
visualized as two punctured Euclidean planes joined by a right circular
cylinder, as shown here for $L_{0}=2R_{0}$.%
%TCIMACRO{\FRAME{dtbpFU}{6.8562in}{4.2168in}{0pt}{\Qcb{$N=2$ wormhole geometry
%in the limit $n\rightarrow\infty$: \ A cylinder of constant radius $R_{0}%
%$\protect\linebreak and height $2R_{0}$ bridging two punctured Euclidean
%planes.}}{}{cylinderwormhole.jpg}{\special{ language "Scientific Word";
%type "GRAPHIC";  maintain-aspect-ratio TRUE;  display "USEDEF";
%valid_file "F";  width 6.8562in;  height 4.2168in;  depth 0pt;
%original-width 6.7914in;  original-height 4.1667in;  cropleft "0";
%croptop "1";  cropright "1";  cropbottom "0";
%filename 'CylinderWormhole.jpg';file-properties "XNPEU";}} }%
%BeginExpansion
\begin{center}
\includegraphics[
natheight=4.166700in,
natwidth=6.791400in,
height=4.2168in,
width=6.8562in
]%
{C:/Users/Thomas Curtright/OneDrive - University of Miami/Desktop/arXiv attempts/graphics/CylinderWormhole__4.pdf}%
\\
$N=2$ wormhole geometry in the limit $n\rightarrow\infty$: \ A cylinder of
constant radius $R_{0}$\protect\linebreak and height $2R_{0}$ bridging two
punctured Euclidean planes.
\end{center}
%EndExpansion


\noindent For these two cases the differential and integrated cross sections
for plane waves incident and scattered on the upper Euclidean space are given
by the following. \ 

For $N=2$,%
\begin{equation}
\frac{d\sigma}{d\theta}=\frac{1}{2\pi k}\left\vert \sum_{m=-\infty}^{\infty
}\left(  S_{m}-1\right)  e^{im\theta}\right\vert ^{2}\ ,\ \ \ \sigma=\frac
{1}{k}\sum_{m=-\infty}^{\infty}\left\vert S_{m}-1\right\vert ^{2}%
\end{equation}
where the $N=2$ wormhole scattering amplitudes are given by
(\ref{WormholeScatteringAmplitudes}) with $h_{1,2}=H_{m}^{\left(  1,2\right)
}\left(  kR_{0}\right)  \ $and $\kappa=\sqrt{k^{2}-m^{2}/R_{0}^{2}}$. \ That
is to say, no matter if $k^{2}R^{2}\geq m^{2}$ so that $\kappa R=\sqrt
{k^{2}R^{2}-m^{2}}$ is real, or if $k^{2}R^{2}\leq m^{2}$ so that $\kappa
R=i\sqrt{m^{2}-k^{2}R^{2}}$ is imaginary, either way $\left\vert
S_{m}\right\vert \neq1$. \ This means there is not only elastic scattering on
the upper plane but also inelastic scattering, i.e. outward flux on the bottom plane.

For $N=3$,
\begin{equation}
\frac{d\sigma}{d\Omega}=\frac{1}{4k^{2}}\left\vert \sum_{l=0}^{\infty}\left(
2l+1\right)  \left(  S_{l}-1\right)  P_{l}\left(  \cos\theta\right)
\right\vert ^{2}\ ,\text{ \ \ }\sigma=\frac{\pi}{k^{2}}\sum_{l=0}^{\infty
}\left(  2l+1\right)  \left\vert S_{l}-1\right\vert ^{2}%
\end{equation}
where the $N=3$ wormhole scattering amplitudes are given by
(\ref{WormholeScatteringAmplitudes}) with \ $h_{1,2}=h_{l+1/2}^{\left(
1,2\right)  }\left(  kR_{0}\right)  \ $and $\kappa=\sqrt{k^{2}-l\left(
l+1\right)  /R_{0}^{2}}$. Again, $\left\vert S_{l}\right\vert \neq1$. \ So
again there is not only elastic scattering on the upper $\mathbb{E}_{3}$ but
also inelastic scattering, i.e. outward flux on the bottom $\mathbb{E}_{3}$.

Here are representative s-wave ($m=0=l$) and p-wave ($m=1=l$) amplitudes for
$N=2$ and $3$, plotted parametrically as functions of $kR_{0}$.

\noindent%
%TCIMACRO{\FRAME{itbpFU}{3.2621in}{3.0139in}{0in}{\Qcb{2D \& 3D Wormhole
%$\left(  \operatorname{Re}\frac{S_{0}-1}{2i},\operatorname{Im}\frac{S_{0}%
%-1}{2i}\right)  $ \protect\linebreak for $L_{0}=2R_{0}$ and $0\leq kR_{0}%
%\leq5$, in blue \& red.}}{}{wormhole0amp.jpg}%
%{\special{ language "Scientific Word";  type "GRAPHIC";
%maintain-aspect-ratio TRUE;  display "USEDEF";  valid_file "F";
%width 3.2621in;  height 3.0139in;  depth 0in;  original-width 4.2186in;
%original-height 3.896in;  cropleft "0";  croptop "1";  cropright "1";
%cropbottom "0";  filename 'Wormhole0Amp.jpg';file-properties "XNPEU";}} }%
%BeginExpansion
{\parbox[b]{3.2621in}{\begin{center}
\includegraphics[
natheight=3.896000in,
natwidth=4.218600in,
height=3.0139in,
width=3.2621in
]%
{C:/Users/Thomas Curtright/OneDrive - University of Miami/Desktop/arXiv attempts/graphics/Wormhole0Amp__5.pdf}%
\\
2D \& 3D Wormhole $\left(  \operatorname{Re}\frac{S_{0}-1}{2i}%
,\operatorname{Im}\frac{S_{0}-1}{2i}\right)  $ \protect\linebreak for
$L_{0}=2R_{0}$ and $0\leq kR_{0}\leq5$, in blue \& red.
\end{center}}}
%EndExpansion
%TCIMACRO{\FRAME{itbpFU}{3.2621in}{3.0139in}{0in}{\Qcb{2D \& 3D Wormhole
%$\left(  \operatorname{Re}\frac{S_{1}-1}{2i},\operatorname{Im}\frac{S_{1}%
%-1}{2i}\right)  $ \protect\linebreak for $L_{0}=2R_{0}$ and $0\leq kR_{0}%
%\leq5$, in blue \& red.}}{}{wormhole1amp.jpg}%
%{\special{ language "Scientific Word";  type "GRAPHIC";
%maintain-aspect-ratio TRUE;  display "USEDEF";  valid_file "F";
%width 3.2621in;  height 3.0139in;  depth 0in;  original-width 4.2186in;
%original-height 3.896in;  cropleft "0";  croptop "1";  cropright "1";
%cropbottom "0";  filename 'Wormhole1Amp.jpg';file-properties "XNPEU";}} }%
%BeginExpansion
{\parbox[b]{3.2621in}{\begin{center}
\includegraphics[
natheight=3.896000in,
natwidth=4.218600in,
height=3.0139in,
width=3.2621in
]%
{C:/Users/Thomas Curtright/OneDrive - University of Miami/Desktop/arXiv attempts/graphics/Wormhole1Amp__6.pdf}%
\\
2D \& 3D Wormhole $\left(  \operatorname{Re}\frac{S_{1}-1}{2i}%
,\operatorname{Im}\frac{S_{1}-1}{2i}\right)  $ \protect\linebreak for
$L_{0}=2R_{0}$ and $0\leq kR_{0}\leq5$, in blue \& red.
\end{center}}}
%EndExpansion


\noindent

\noindent Note the amplitudes lie within the unitarity circle given by
$i\left(  1-\exp\left(  -2ikR_{0}\right)  \right)  /2$, thereby evincing
inelastic scattering. \ And their asymptotic limits are the same, namely
$S_{0,1}\underset{kR_{0}\rightarrow\infty}{\longrightarrow}0$, albeit with
non-trivial but diminishing oscillations, such that waves incident on the
upper branch of the wormhole give rise to probability flux emergent only on
the lower branch, in this large $k$ limit. \ That is to say, for these angular
momenta there are no outgoing reflected waves on the upper branch as
$kR_{0}\rightarrow\infty$.\footnote{It is noteworthy that the limiting
behavior $S_{n}\underset{kR_{0}\rightarrow\infty}{\longrightarrow}0$ for any
fixed $n$ is what would be expected from classical trajectories for particles
moving freely on the manifold. \ In the large momentum limit, classical
particles with any fixed angular momentum will always strike and then traverse
the tube, going from one branch of the ambient flat space to the other branch,
as illustrated in the following Figure.%
%TCIMACRO{\FRAME{dtbpFU}{2.2442in}{1.4935in}{0pt}{\Qcb{A classical trajectory,
%with winding number 2, for a particle transiting the $n\rightarrow\infty$
%wormhole.}}{}{spiral2cylinderforward.jpg}%
%{\special{ language "Scientific Word";  type "GRAPHIC";
%maintain-aspect-ratio TRUE;  display "USEDEF";  valid_file "F";
%width 2.2442in;  height 1.4935in;  depth 0pt;  original-width 11.0004in;
%original-height 7.2809in;  cropleft "0";  croptop "1";  cropright "1";
%cropbottom "0";
%filename 'Spiral2CylinderForward.jpg';file-properties "XNPEU";}} }%
%BeginExpansion
\begin{center}
\includegraphics[
natheight=7.280900in,
natwidth=11.000400in,
height=1.4935in,
width=2.2442in
]%
{C:/Users/Thomas Curtright/OneDrive - University of Miami/Desktop/arXiv attempts/graphics/Spiral2CylinderForward__7.pdf}%
\\
A classical trajectory, with winding number 2, for a particle transiting the
$n\rightarrow\infty$ wormhole.
\end{center}
%EndExpansion
}

\newpage

However, summing \emph{all} the partial wave contributions to the integrated
cross section leads to the following graphs.%
%TCIMACRO{\FRAME{dtbpFU}{5.9352in}{3.9444in}{0pt}{\Qcb{$L_{0}=2R_{0}$ Wormhole
%$\sigma/V_{N-1}\left(  R\right)  $ for $N=2$ \& $3$ in black \& red.}}%
%{}{wormhole2d3d.jpg}{\special{ language "Scientific Word";  type "GRAPHIC";
%maintain-aspect-ratio TRUE;  display "USEDEF";  valid_file "F";
%width 5.9352in;  height 3.9444in;  depth 0pt;  original-width 5.8747in;
%original-height 3.896in;  cropleft "0";  croptop "1";  cropright "1";
%cropbottom "0";  filename 'Wormhole2D3D.jpg';file-properties "XNPEU";}} }%
%BeginExpansion
\begin{center}
\includegraphics[
natheight=3.896000in,
natwidth=5.874700in,
height=3.9444in,
width=5.9352in
]%
{C:/Users/Thomas Curtright/OneDrive - University of Miami/Desktop/arXiv attempts/graphics/Wormhole2D3D__8.pdf}%
\\
$L_{0}=2R_{0}$ Wormhole $\sigma/V_{N-1}\left(  R\right)  $ for $N=2$ \& $3$ in
black \& red.
\end{center}
%EndExpansion
Asymptotically, for the full sum over all partial waves, $\sigma
\underset{kR_{0}\rightarrow\infty}{\longrightarrow}2R_{0}$ for $N=2$ and
$\sigma\underset{kR_{0}\rightarrow\infty}{\longrightarrow}\pi R_{0}^{2}$ for
$N=3$, even though any fixed angular momentum gives no contribution to
$\sigma$ in this limit. \ Note that these asymptotic results are each exactly
one half of the elastic scattering cross-sections for impenetrable disk or
sphere scattering, respectively.

\section{Summary}

The paper has exhibited various relationships between non-relativistic quantum
systems involving a potential, in flat space, and systems without a potential
but defined on curved manifolds. \ The main general results are encoded in
(\ref{NDGeomFromPotl}), (\ref{NDMetric}), (\ref{NDPotlFromGeom}), and
(\ref{LinePotl}) for $N$-dimensional spatial manifolds. \ More specific
examples have been presented involving $1/r$ potentials and regularized
$1/R^{4}$ potentials on wormhole manifolds. \ Perhaps our discussion should be
viewed as a contemporary reconsideration of Riemann's ideas\ (\textit{pre}%
-relativity) about universal time Newtonian dynamics in terms of geometry
\cite{Riemann} but in the context of quantum mechanics. \ Or perhaps not.
\ The reader can decide.\bigskip

\noindent\textbf{Acknowledgement \ }TC thanks T. S. Van Kortryk for
discussions about this work.\ \ This research was initiated when TC was the
Clark Way Harrison Visiting Professor at Washington University in St.
Louis.\bigskip

\newpage

\section{Appendix: \ Wormhole Equatorial Slice Profiles}

The following Figures show 3D embeddings for 2D equatorial slice $\left(
R,Z\right)  $ profiles of various \textquotedblleft$p$-norm
wormholes\textquotedblright\ \cite{CA et al.,AC} where on the equatorial
plane
\begin{equation}
\left(  ds\right)  ^{2}=\left(  dw\right)  ^{2}+R^{2}\left(  w\right)  \left(
d\theta\right)  ^{2}=\left(  dx\right)  ^{2}+\left(  dy\right)  ^{2}+\left(
dZ\right)  ^{2} \tag{F1}%
\end{equation}%
\begin{equation}
x\left(  w,\theta\right)  =R\left(  w\right)  \cos\theta\ ,\ \ \ y\left(
w,\theta\right)  =R\left(  w\right)  \sin\theta\ ,\ \ \ R\left(  w\right)
=\left(  R_{0}^{p}+\left(  w^{2}\right)  ^{p/2}\right)  ^{1/p} \tag{F2}%
\end{equation}%
\begin{equation}
Z\left(  w\right)  =\int_{0}^{w}\sqrt{1-\left(  dR\left(  \varpi\right)
/d\varpi\right)  ^{2}}d\varpi=\int_{0}^{w}\sqrt{1-\left(  \varpi^{2}\right)
^{p-1}\left(  R_{0}^{p}+\left(  \varpi^{2}\right)  ^{p/2}\right)  ^{\frac
{2}{p}-2}}\,d\varpi\tag{F3}%
\end{equation}
For example, for $p=2$,%
\begin{equation}
Z\left(  w\right)  =R_{0}\ln\left(  \frac{w+\sqrt{R_{0}^{2}+w^{2}}}{R_{0}%
}\right)  =R_{0}\operatorname{arcsinh}\left(  \frac{w}{R_{0}}\right)  \tag{F4}%
\end{equation}
For generic $p$, it is easiest to obtain $Z\left(  w\right)  $ by numerical
solution of
\begin{equation}
\frac{dZ\left(  w\right)  }{dw}=\sqrt{1-\left(  w^{2}\right)  ^{p-1}\left(
R_{0}^{p}+\left(  w^{2}\right)  ^{p/2}\right)  ^{\frac{2}{p}-2}} \tag{F5}%
\end{equation}
with initial condition $Z\left(  0\right)  =0$.%

%TCIMACRO{\FRAME{itbpFU}{3.0113in}{3.0104in}{0in}{\Qcb{$\left(  R,Z\right)  $
%profile for $p=2$ (black)\protect\linebreak compared to $p=4$ (red).}}%
%{}{p2vsp4.jpg}{\special{ language "Scientific Word";  type "GRAPHIC";
%display "USEDEF";  valid_file "F";  width 3.0113in;  height 3.0104in;
%depth 0in;  original-width 3.9271in;  original-height 3.9167in;
%cropleft "0";  croptop "1";  cropright "1.0028";  cropbottom "0";
%filename 'p2vsp4.jpg';file-properties "XNPEU";}} }%
%BeginExpansion
{\parbox[b]{3.0113in}{\begin{center}
\includegraphics[
trim=0.000000in 0.000000in -0.010996in 0.000000in,
natheight=3.916700in,
natwidth=3.927100in,
height=3.0104in,
width=3.0113in
]%
{C:/Users/Thomas Curtright/OneDrive - University of Miami/Desktop/arXiv attempts/graphics/p2vsp4__9.pdf}%
\\
$\left(  R,Z\right)  $ profile for $p=2$ (black)\protect\linebreak compared to
$p=4$ (red).
\end{center}}}
%EndExpansion
%TCIMACRO{\FRAME{itbpFU}{3.0113in}{3.0104in}{0in}{\Qcb{$\left(  R,Z\right)  $
%profile for $p=2$ (black)\protect\linebreak compared to $p=8$ (red).}}%
%{}{p2vsp8.jpg}{\special{ language "Scientific Word";  type "GRAPHIC";
%maintain-aspect-ratio TRUE;  display "USEDEF";  valid_file "F";
%width 3.0113in;  height 3.0104in;  depth 0in;  original-width 3.9271in;
%original-height 3.9167in;  cropleft "0";  croptop "1";  cropright "0.9976";
%cropbottom "0";  filename 'p2vsp8.jpg';file-properties "XNPEU";}} }%
%BeginExpansion
{\parbox[b]{3.0113in}{\begin{center}
\includegraphics[
trim=0.000000in 0.000000in 0.009425in 0.000000in,
natheight=3.916700in,
natwidth=3.927100in,
height=3.0104in,
width=3.0113in
]%
{C:/Users/Thomas Curtright/OneDrive - University of Miami/Desktop/arXiv attempts/graphics/p2vsp8__10.pdf}%
\\
$\left(  R,Z\right)  $ profile for $p=2$ (black)\protect\linebreak compared to
$p=8$ (red).
\end{center}}}
%EndExpansion
%

%TCIMACRO{\FRAME{itbpFU}{3.0113in}{3.0096in}{0in}{\Qcb{$\left(  R,Z\right)  $
%profile for $p=2$ (black)\protect\linebreak compared to $p=16$ (red).}}%
%{}{p2vsp16.jpg}{\special{ language "Scientific Word";  type "GRAPHIC";
%maintain-aspect-ratio TRUE;  display "USEDEF";  valid_file "F";
%width 3.0113in;  height 3.0096in;  depth 0in;  original-width 3.9271in;
%original-height 3.9167in;  cropleft "0";  croptop "1";  cropright "0.9976";
%cropbottom "0";  filename 'p2vsp16.jpg';file-properties "XNPEU";}} }%
%BeginExpansion
{\parbox[b]{3.0113in}{\begin{center}
\includegraphics[
trim=0.000000in 0.000000in 0.009425in 0.000000in,
natheight=3.916700in,
natwidth=3.927100in,
height=3.0096in,
width=3.0113in
]%
{C:/Users/Thomas Curtright/OneDrive - University of Miami/Desktop/arXiv attempts/graphics/p2vsp16__11.pdf}%
\\
$\left(  R,Z\right)  $ profile for $p=2$ (black)\protect\linebreak compared to
$p=16$ (red).
\end{center}}}
%EndExpansion
%TCIMACRO{\FRAME{itbpFU}{3.0113in}{3.0113in}{0.0069in}{\Qcb{$\left(
%R,Z\right)  $ profile for $p=2$ (black)\protect\linebreak compared to $p=32$
%(red).}}{}{p2vsp32.jpg}{\special{ language "Scientific Word";
%type "GRAPHIC";  maintain-aspect-ratio TRUE;  display "USEDEF";
%valid_file "F";  width 3.0113in;  height 3.0113in;  depth 0.0069in;
%original-width 3.9271in;  original-height 3.9167in;  cropleft "0";
%croptop "1.0025";  cropright "1";  cropbottom "0";
%filename 'p2vsp32.jpg';file-properties "XNPEU";}} }%
%BeginExpansion
\raisebox{-0.0069in}{\parbox[b]{3.0113in}{\begin{center}
\includegraphics[
trim=0.000000in 0.000000in 0.000000in -0.009792in,
natheight=3.916700in,
natwidth=3.927100in,
height=3.0113in,
width=3.0113in
]%
{C:/Users/Thomas Curtright/OneDrive - University of Miami/Desktop/arXiv attempts/graphics/p2vsp32__12.pdf}%
\\
$\left(  R,Z\right)  $ profile for $p=2$ (black)\protect\linebreak compared to
$p=32$ (red).
\end{center}}}
%EndExpansion
%

%TCIMACRO{\FRAME{itbpFU}{3.0113in}{3.0096in}{0in}{\Qcb{$\left(  R,Z\right)  $
%profile for $p=2$ (black)\protect\linebreak compared to $p=64$ (red).}}%
%{}{p2vsp64.jpg}{\special{ language "Scientific Word";  type "GRAPHIC";
%maintain-aspect-ratio TRUE;  display "USEDEF";  valid_file "F";
%width 3.0113in;  height 3.0096in;  depth 0in;  original-width 3.9271in;
%original-height 3.9167in;  cropleft "0";  croptop "1";  cropright "0.9976";
%cropbottom "0";  filename 'p2vsp64.jpg';file-properties "XNPEU";}} }%
%BeginExpansion
{\parbox[b]{3.0113in}{\begin{center}
\includegraphics[
trim=0.000000in 0.000000in 0.009425in 0.000000in,
natheight=3.916700in,
natwidth=3.927100in,
height=3.0096in,
width=3.0113in
]%
{C:/Users/Thomas Curtright/OneDrive - University of Miami/Desktop/arXiv attempts/graphics/p2vsp64__13.pdf}%
\\
$\left(  R,Z\right)  $ profile for $p=2$ (black)\protect\linebreak compared to
$p=64$ (red).
\end{center}}}
%EndExpansion
%TCIMACRO{\FRAME{itbpFU}{3.0113in}{3.0096in}{0in}{\Qcb{$\left(  R,Z\right)  $
%profile for $p=2$ (black)\protect\linebreak compared to $p=128$ (red).}}%
%{}{p2vsp128.jpg}{\special{ language "Scientific Word";  type "GRAPHIC";
%maintain-aspect-ratio TRUE;  display "USEDEF";  valid_file "F";
%width 3.0113in;  height 3.0096in;  depth 0in;  original-width 3.9271in;
%original-height 3.9167in;  cropleft "0";  croptop "1";  cropright "0.9976";
%cropbottom "0";  filename 'p2vsp128.jpg';file-properties "XNPEU";}} }%
%BeginExpansion
{\parbox[b]{3.0113in}{\begin{center}
\includegraphics[
trim=0.000000in 0.000000in 0.009425in 0.000000in,
natheight=3.916700in,
natwidth=3.927100in,
height=3.0096in,
width=3.0113in
]%
{C:/Users/Thomas Curtright/OneDrive - University of Miami/Desktop/arXiv attempts/graphics/p2vsp128__14.pdf}%
\\
$\left(  R,Z\right)  $ profile for $p=2$ (black)\protect\linebreak compared to
$p=128$ (red).
\end{center}}}
%EndExpansion


The Figures evince the limit $p\rightarrow\infty$ very directly, for any $N$.
\ In that limit the wormhole profile consists of two flat but punctured
Euclidean spaces $\mathbb{E}_{N}$ connected by an intrinsically flat
(hyper)cylinder tube whose boundaries are the (hyper)spherical holes of radius
$R_{0}$ in each of the $\mathbb{E}_{N}$ --- a manifold most easily visualized
in its entirety for $N=2$.

\newpage

\begin{thebibliography}{99}                                                                                               %


\bibitem {MTW}C.W. Misner, K.S. Thorne, and J.A. Wheeler,
\textit{\href{https://www.amazon.com/Gravitation-Charles-W-Misner/dp/0691177791/ref=sr_1_1?dchild=1&hvadid=78065396245556&hvbmt=bp&hvdev=c&hvqmt=p&keywords=misner+thorne+and+wheeler&qid=1609378280&sr=8-1&tag=mh0b-20}{\textit{Gravitation}%
}}, Princeton University Press (2017), Chapter 25.

\bibitem {Ince}E.L. Ince,
\textit{\href{https://www.google.com/books/edition/Ordinary_Differential_Equations/mbyqAAAAQBAJ?hl=en&gbpv=1&dq=E.L.+Ince,+Ordinary+Differential+Equations&printsec=frontcover}{\textit{Ordinary
Differential Equations}}}, Dover, New York (1956).

\bibitem {Green}G. Green, \textquotedblleft Mathematical investigations
concerning the laws of the equilibrium of fluids analogous to the electric
fluid, with other similar researches\textquotedblright%
\ \href{https://babel.hathitrust.org/cgi/pt?id=nyp.33433004518365&view=1up&seq=13}{Trans.
Cambridge Phil. Soc. 5 (1835) 1--63}.

\bibitem {Sommerfeld}A. Sommerfeld,
\textit{\href{https://www.amazon.com/Partial-Differential-Equations-Lectures-Theoretical/dp/0126546584/ref=pd_sbs_14_5/144-0603225-6367369?_encoding=UTF8&pd_rd_i=0126546584&pd_rd_r=a17dd993-2c43-41cb-8229-41ae39f67895&pd_rd_w=HpjAM&pd_rd_wg=UwgN7&pf_rd_p=ed1e2146-ecfe-435e-b3b5-d79fa072fd58&pf_rd_r=2BD716DFW2V8N64XHVXE&psc=1&refRID=2BD716DFW2V8N64XHVXE}{\textit{Partial
Differential Equations in Physics}}}, Academic Press (1964), Appendix IV.

\bibitem {ER}A. Einstein and N. Rosen, \textquotedblleft The Particle Problem
in the General Theory of Relativity\textquotedblright%
\ \href{https://doi.org/10.1103/PhysRev.48.73}{Phys. Rev. 48 (1935) 73-77}.

\bibitem {Thorne}M.S. Morris and K.S. Thorne, \textquotedblleft Wormholes in
spacetime and their use for interstellar travel: A tool for teaching general
relativity\textquotedblright\ \href{https://doi.org/10.1119/1.15620}{Am. J.
Phys. 56 (1988) 395-412}.

\bibitem {Ellis}H.G. Ellis, \textquotedblleft Ether flow through a drainhole:
A particle model in general relativity\textquotedblright%
\ \href{https://doi.org/10.1063/1.1666161}{J. Math. Phys. 14 (1973) 104--118}.

\bibitem {Singular1950}K.M. Case, \textquotedblleft Singular
Potentials\textquotedblright%
\ \href{https://doi.org/10.1103/PhysRev.80.797}{Phys. Rev. 80 (1950) 797-806}.

\bibitem {Singular1971}W.M. Frank, D.J. Land, and R.M. Spector,
\textquotedblleft Singular Potentials\textquotedblright%
\ \href{https://doi.org/10.1103/RevModPhys.43.36}{Rev. Mod. Phys. 43 (1971)
36-98}.

\bibitem {C et al.}T.L. Curtright, et al., \textquotedblleft Charged line
segments and ellipsoidal equipotentials\textquotedblright%
\ \href{https://doi.org/10.1088/0143-0807/37/3/035201}{Eur. J. Phys. 37 (2016)
035201}.

\bibitem {DLMF}H. Volkmer, \href{https://dlmf.nist.gov/30}{Chapter 30:
\ Spheroidal Wave Functions}, \href{http://dlmf.nist.gov/}{NIST Digital
Library of Mathematical Functions}, Release 1.1.0 of 2020-12-15. F. W. J.
Olver, A. B. Olde Daalhuis, D. W. Lozier, B. I. Schneider, R. F. Boisvert, C.
W. Clark, B. R. Miller, B. V. Saunders, H. S. Cohl, and M. A. McClain, eds.

\bibitem {CS}T.L. Curtright and S. Subedi, \textquotedblleft Holistic
Scattering\textquotedblright\ \emph{unpublished, in preparation}.

\bibitem {CS2}T.L. Curtright and S. Subedi, \textquotedblleft Regge Hole
Theory\textquotedblright\ \emph{unpublished, in preparation}.

\bibitem {CA et al.}T. Curtright, H. Alshal, et al., \textquotedblleft The
Conducting Ring Viewed as a Wormhole\textquotedblright%
\ \href{https://iopscience.iop.org/article/10.1088/1361-6404/aae3cd}{Eur. J.
Phys. 40 (2019) 015206}.

\bibitem {AC}H. Alshal and T. Curtright, \textquotedblleft Grounded
Hyperspheres as Squashed Wormholes\textquotedblright%
\ \href{https://aip.scitation.org/doi/10.1063/1.5044432}{J. Math. Phys. 60
(2019) 032901}.

\bibitem {Riemann}Riemann's pre-relativity thoughts on gravity as curved
space, and related ideas by several others, are discussed
\href{https://hsm.stackexchange.com/questions/9585/what-are-the-references-for-riemanns-discussion-of-gravity#:~:text=Similar%20to%20Newton%2C%20but%20mathematically%20in%20greater%20detail%2C,and%20normal%20matter%20represents%20sinks%20in%20this%20aether.}{here}%
.

\bibitem {Gottfried}K. Gottfried, \textit{Quantum Mechanics Volume I:
Fundamentals}, Chapter III, Section 15 (1966).
\end{thebibliography}


\end{document}